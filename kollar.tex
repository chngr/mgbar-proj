\IfFileExists{stacks-project.cls}{%
\documentclass{stacks-project}
}{%
\documentclass{amsart}
}

% The following AMS packages are automatically loaded with
% the amsart documentclass:
%\usepackage{amsmath}
%\usepackage{amssymb}
%\usepackage{amsthm}

% For dealing with references we use the comment environment
\usepackage{verbatim}
\newenvironment{reference}{\comment}{\endcomment}
%\newenvironment{reference}{}{}
\newenvironment{slogan}{\comment}{\endcomment}
\newenvironment{history}{\comment}{\endcomment}

% For commutative diagrams you can use
% \usepackage{amscd}
\usepackage[all]{xy}

% We use 2cell for 2-commutative diagrams.
\xyoption{2cell}
\UseAllTwocells

% To put source file link in headers.
% Change "template.tex" to "this_filename.tex"
% \usepackage{fancyhdr}
% \pagestyle{fancy}
% \lhead{}
% \chead{}
% \rhead{Source file: \url{template.tex}}
% \lfoot{}
% \cfoot{\thepage}
% \rfoot{}
% \renewcommand{\headrulewidth}{0pt}
% \renewcommand{\footrulewidth}{0pt}
% \renewcommand{\headheight}{12pt}

\usepackage{multicol}

% For cross-file-references
\usepackage{xr-hyper}

% Package for hypertext links:
\usepackage{hyperref}

% For any local file, say "hello.tex" you want to link to please
% use \externaldocument[hello-]{hello}
\externaldocument[introduction-]{introduction}
\externaldocument[conventions-]{conventions}
\externaldocument[sets-]{sets}
\externaldocument[categories-]{categories}
\externaldocument[topology-]{topology}
\externaldocument[sheaves-]{sheaves}
\externaldocument[sites-]{sites}
\externaldocument[stacks-]{stacks}
\externaldocument[fields-]{fields}
\externaldocument[algebra-]{algebra}
\externaldocument[brauer-]{brauer}
\externaldocument[homology-]{homology}
\externaldocument[derived-]{derived}
\externaldocument[simplicial-]{simplicial}
\externaldocument[more-algebra-]{more-algebra}
\externaldocument[smoothing-]{smoothing}
\externaldocument[modules-]{modules}
\externaldocument[sites-modules-]{sites-modules}
\externaldocument[injectives-]{injectives}
\externaldocument[cohomology-]{cohomology}
\externaldocument[sites-cohomology-]{sites-cohomology}
\externaldocument[dga-]{dga}
\externaldocument[dpa-]{dpa}
\externaldocument[hypercovering-]{hypercovering}
\externaldocument[schemes-]{schemes}
\externaldocument[constructions-]{constructions}
\externaldocument[properties-]{properties}
\externaldocument[morphisms-]{morphisms}
\externaldocument[coherent-]{coherent}
\externaldocument[divisors-]{divisors}
\externaldocument[limits-]{limits}
\externaldocument[varieties-]{varieties}
\externaldocument[topologies-]{topologies}
\externaldocument[descent-]{descent}
\externaldocument[perfect-]{perfect}
\externaldocument[more-morphisms-]{more-morphisms}
\externaldocument[flat-]{flat}
\externaldocument[groupoids-]{groupoids}
\externaldocument[more-groupoids-]{more-groupoids}
\externaldocument[etale-]{etale}
\externaldocument[chow-]{chow}
\externaldocument[intersection-]{intersection}
\externaldocument[pic-]{pic}
\externaldocument[adequate-]{adequate}
\externaldocument[dualizing-]{dualizing}
\externaldocument[duality-]{duality}
\externaldocument[discriminant-]{discriminant}
\externaldocument[local-cohomology-]{local-cohomology}
\externaldocument[curves-]{curves}
\externaldocument[resolve-]{resolve}
\externaldocument[models-]{models}
\externaldocument[pione-]{pione}
\externaldocument[etale-cohomology-]{etale-cohomology}
\externaldocument[proetale-]{proetale}
\externaldocument[crystalline-]{crystalline}
\externaldocument[spaces-]{spaces}
\externaldocument[spaces-properties-]{spaces-properties}
\externaldocument[spaces-morphisms-]{spaces-morphisms}
\externaldocument[decent-spaces-]{decent-spaces}
\externaldocument[spaces-cohomology-]{spaces-cohomology}
\externaldocument[spaces-limits-]{spaces-limits}
\externaldocument[spaces-divisors-]{spaces-divisors}
\externaldocument[spaces-over-fields-]{spaces-over-fields}
\externaldocument[spaces-topologies-]{spaces-topologies}
\externaldocument[spaces-descent-]{spaces-descent}
\externaldocument[spaces-perfect-]{spaces-perfect}
\externaldocument[spaces-more-morphisms-]{spaces-more-morphisms}
\externaldocument[spaces-flat-]{spaces-flat}
\externaldocument[spaces-groupoids-]{spaces-groupoids}
\externaldocument[spaces-more-groupoids-]{spaces-more-groupoids}
\externaldocument[bootstrap-]{bootstrap}
\externaldocument[spaces-pushouts-]{spaces-pushouts}
\externaldocument[groupoids-quotients-]{groupoids-quotients}
\externaldocument[spaces-more-cohomology-]{spaces-more-cohomology}
\externaldocument[spaces-simplicial-]{spaces-simplicial}
\externaldocument[spaces-duality-]{spaces-duality}
\externaldocument[formal-spaces-]{formal-spaces}
\externaldocument[restricted-]{restricted}
\externaldocument[spaces-resolve-]{spaces-resolve}
\externaldocument[formal-defos-]{formal-defos}
\externaldocument[defos-]{defos}
\externaldocument[cotangent-]{cotangent}
\externaldocument[examples-defos-]{examples-defos}
\externaldocument[algebraic-]{algebraic}
\externaldocument[examples-stacks-]{examples-stacks}
\externaldocument[stacks-sheaves-]{stacks-sheaves}
\externaldocument[criteria-]{criteria}
\externaldocument[artin-]{artin}
\externaldocument[quot-]{quot}
\externaldocument[stacks-properties-]{stacks-properties}
\externaldocument[stacks-morphisms-]{stacks-morphisms}
\externaldocument[stacks-limits-]{stacks-limits}
\externaldocument[stacks-cohomology-]{stacks-cohomology}
\externaldocument[stacks-perfect-]{stacks-perfect}
\externaldocument[stacks-introduction-]{stacks-introduction}
\externaldocument[stacks-more-morphisms-]{stacks-more-morphisms}
\externaldocument[stacks-geometry-]{stacks-geometry}
\externaldocument[moduli-]{moduli}
\externaldocument[moduli-curves-]{moduli-curves}
\externaldocument[examples-]{examples}
\externaldocument[exercises-]{exercises}
\externaldocument[guide-]{guide}
\externaldocument[desirables-]{desirables}
\externaldocument[coding-]{coding}
\externaldocument[obsolete-]{obsolete}
\externaldocument[fdl-]{fdl}
\externaldocument[index-]{index}

% Theorem environments.
%
\theoremstyle{plain}
\newtheorem{theorem}[subsection]{Theorem}
\newtheorem{proposition}[subsection]{Proposition}
\newtheorem{lemma}[subsection]{Lemma}

\theoremstyle{definition}
\newtheorem{definition}[subsection]{Definition}
\newtheorem{example}[subsection]{Example}
\newtheorem{exercise}[subsection]{Exercise}
\newtheorem{situation}[subsection]{Situation}

\theoremstyle{remark}
\newtheorem{remark}[subsection]{Remark}
\newtheorem{remarks}[subsection]{Remarks}

\numberwithin{equation}{subsection}

% Macros
%
\def\lim{\mathop{\rm lim}\nolimits}
\def\colim{\mathop{\rm colim}\nolimits}
\def\Spec{\mathop{\rm Spec}}
\def\Hom{\mathop{\rm Hom}\nolimits}
\def\Ext{\mathop{\rm Ext}\nolimits}
\def\SheafHom{\mathop{\mathcal{H}\!{\it om}}\nolimits}
\def\SheafExt{\mathop{\mathcal{E}\!{\it xt}}\nolimits}
\def\Sch{\textit{Sch}}
\def\Mor{\mathop{\rm Mor}\nolimits}
\def\Ob{\mathop{\rm Ob}\nolimits}
\def\Sh{\mathop{\textit{Sh}}\nolimits}
\def\NL{\mathop{N\!L}\nolimits}
\def\proetale{{pro\text{-}\acute{e}tale}}
\def\etale{{\acute{e}tale}}
\def\QCoh{\textit{QCoh}}
\def\Ker{\mathop{\rm Ker}}
\def\Im{\mathop{\rm Im}}
\def\Coker{\mathop{\rm Coker}}
\def\Coim{\mathop{\rm Coim}}

%
% Macros for moduli stacks/spaces
%
\def\QCohstack{\mathcal{QC}\!{\it oh}}
\def\Cohstack{\mathcal{C}\!{\it oh}}
\def\Spacesstack{\mathcal{S}\!{\it paces}}
\def\Quotfunctor{{\rm Quot}}
\def\Hilbfunctor{{\rm Hilb}}
\def\Curvesstack{\mathcal{C}\!{\it urves}}
\def\Polarizedstack{\mathcal{P}\!{\it olarized}}
\def\Complexesstack{\mathcal{C}\!{\it omplexes}}
% \Pic is the operator that assigns to X its picard group, usage \Pic(X)
% \Picardstack_{X/B} denotes the Picard stack of X over B
% \Picardfunctor_{X/B} denotes the Picard functor of X over B
\def\Pic{\mathop{\rm Pic}\nolimits}
\def\Picardstack{\mathcal{P}\!{\it ic}}
\def\Picardfunctor{{\rm Pic}}
\def\Deformationcategory{\mathcal{D}\!{\it ef}}


\newcommand{\todo}[1]{\footnote{\textbf{TODO.} #1}}

\begin{document}
\title{Koll\'ar's Argument}
\maketitle

\section{Related Results}

\begin{lemma}[{\cite[\href{http://stacks.math.columbia.edu/tag/01E6}{Tag 01E6}]{stacks-project}}]
\label{lemma-projection-formula}
Let $f : X \to Y$ be a morphism of ringed spaces.
Let $\mathcal{F}$ be an $\mathcal{O}_X$-module.
Let $\mathcal{E}$ be an $\mathcal{O}_Y$-module.
Assume $\mathcal{E}$ is finite locally free on $Y$, see
Modules, Definition \ref{modules-definition-locally-free}.
Then there exist isomorphisms
$$
\mathcal{E} \otimes_{\mathcal{O}_Y} R^qf_*\mathcal{F}
\longrightarrow
R^qf_*(f^*\mathcal{E} \otimes_{\mathcal{O}_X} \mathcal{F})
$$
for all $q \geq 0$. In fact there exists an isomorphism
$$
\mathcal{E} \otimes_{\mathcal{O}_Y} Rf_*\mathcal{F}
\longrightarrow
Rf_*(f^*\mathcal{E} \otimes_{\mathcal{O}_X} \mathcal{F})
$$
in $D^{+}(Y)$ functorial in $\mathcal{F}$.
\end{lemma}

\begin{lemma}[{\cite[\href{http://stacks.math.columbia.edu/tag/0BAV}{Tag 0BAV}]{stacks-project}}]
\label{lemma-ubiquity-nagata}
The following types of algebraic spaces are Nagata.
\begin{enumerate}
\item Any algebraic space locally of finite type over a Nagata scheme.
\item Any algebraic space locally of finite type over a field.
\item Any algebraic space locally of finite type over a
Noetherian complete local ring.
\item Any algebraic space locally of finite type over $\mathbf{Z}$.
\item Any algebraic space locally of finite type over a Dedekind ring of
characteristic zero.
\item And so on.
\end{enumerate}
\end{lemma}

\begin{lemma}[{\cite[\href{http://stacks.math.columbia.edu/tag/0BB5}{Tag 0BB5}]{stacks-project}}]
\label{lemma-nagata-normalization}
Let $S$ be a scheme. Let $X$ be a Nagata algebraic space over $S$.
The normalization $\nu : X^\nu \to X$ is a finite morphism.
\end{lemma}

\begin{lemma}[{\cite[\href{http://stacks.math.columbia.edu/tag/089I}{Tag 089I}]{stacks-project}}]
\label{lemma-weak-chow}
Let $A$ be a ring. Let $X$ be an algebraic space over $\Spec(A)$
whose structure morphism $X \to \Spec(A)$ is separated of finite type.
Then there exists a proper surjective morphism $X' \to X$
where $X'$ is a scheme which is H-quasi-projective over $\Spec(A)$.
\end{lemma}

\begin{lemma}[Chow's lemma][{\cite[\href{http://stacks.math.columbia.edu/tag/088U}{Tag 088U}]{stacks-project}}]
\label{lemma-chow-noetherian-separated}
\begin{reference}
\cite[IV Theorem 3.1]{Kn}
\end{reference}
Let $S$ be a scheme. Let $f : X \to Y$ be a morphism of algebraic spaces
over $S$. Assume $f$ separated of finite type, and $Y$ separated and
Noetherian. Then there exists a commutative diagram
$$
\xymatrix{
X \ar[rd] & X' \ar[l] \ar[d] \ar[r] & \mathbf{P}^n_Y \ar[ld] \\
& Y
}
$$
where $X' \to X$ is a $U$-admissible blowup for some dense open
$U \subset X$ and the morphism $X' \to \mathbf{P}^n_Y$ is an immersion.
\end{lemma}

\section{Positivity Lemma}

\begin{lemma}
Let $X$ be a scheme over a field $k$, and let $\mathcal{E}$ be a%semipositive
locally free sheaf on $X$ of rank $m$.
Let $\mathcal{F} = \mathrm{Sym}^d(\mathcal{E})$ for some $d \geq 1$ and let $n$
be the rank of $\mathcal{F}$.
Let $q\colon \mathcal{F} \to \mathcal{Q}$ be a locally free quotient of rank $k$. 
Let $G = \mathrm{PGL}_m$ be the projective general linear group over $k$.
Then,
\begin{enumerate}
\item There exists a closed subscheme $D \subset
\mathbf{P}((\mathcal{E}^\vee)^{\oplus m})$ such that
$$
\mathbf{P}((\mathcal{E}^\vee)^{\oplus m}) \setminus D \longrightarrow X
$$
is a principal $G$-bundle.
\item There is a $G$-action on $\mathrm{Gr}(n,k)$, and 
a $G$-equivariant map
$$
\mathbf{P}((\mathcal{E}^\vee)^{\oplus m}) \setminus D \longrightarrow
\mathrm{Gr}(n,k).
$$
\item There exists a map of stacks
$$
X \longrightarrow [\mathrm{Gr}(n,k)/G].
$$
\end{enumerate}
\end{lemma}
\begin{proof}
We first prove (1).
Let $\mathbf{P} = \mathbf{P}( (\mathcal{E}^\vee)^{\oplus m})$ be the projective
bundle of matrices whose columns belong to $\mathcal{E}$, and write $\pi :
\mathbf{P} \to X$ for the projection.
Using the projection formula (Lemma \ref{lemma-projection-formula}),
\begin{align*}
H^0(\mathbf{P}, \pi^*\mathcal{E}(1))
& \cong H^0(X, \pi_*(\pi^*\mathcal{E} \otimes_{\mathcal{O}_{\mathbf{P}}}
\mathcal{O}_{\mathbf{P}}(1)))\\
& \cong H^0(X,\mathcal{E} \otimes_{\mathcal{O}_X}
\pi_*\mathcal{O}_{\mathbf{P}}(1))\\
& \cong H^0(X,\mathcal{E} \otimes (\mathcal{E}^\vee)^{\oplus m})\\
& \cong H^0(X,\mathrm{End}(\mathcal{E})^{\oplus m})
\end{align*}
has $m$ sections, corresponding to the identity on each summand.
We therefore have an evaluation morphism
$$
B : \mathcal{O}_{\mathbf{P}}^{\oplus m} \longrightarrow \pi^*\mathcal{E}(1).
$$
Taking determinants, this gives a map
$$
\det B : \mathcal{O}_{\mathbf{P}} \longrightarrow \det(\pi^*\mathcal{E}) (m).
$$
Let $\mathcal{I} \otimes \det(\pi^*\mathcal{E}) (m)$ be the image of $\det B$,
and let $D$ be the subscheme of $\mathbf{P}$ defined by $\mathcal{I}$.
Then, $D$ defines the locus for which $\det B$ vanishes, which is the locus of
$\mathbf{P}$ where the corresponding matrices are not invertible. We therefore
see that $\mathbf{P} \setminus D$ corresponds to those matrices that are
invertible, i.e., are elements of $\mathrm{PGL}_m$. This turns $\mathbf{P}
\setminus D$ into a principal $\mathrm{PGL}_m$-bundle over $X$.

For (2), note that $\mathrm{PGL}_m$ acts on $\mathrm{Gr}(n,k)$ via
$$
\mathrm{GL}_m \longrightarrow (\mathrm{GL}_m)^{\times d} \longrightarrow
\mathrm{Aut}(\mathrm{Sym}^d (k^{\oplus m})),
$$
which descends to their quotients by $k^*$; note that
$\mathrm{Aut}(\mathrm{Sym}^d (k^{\oplus m}))/k^*$ is a subgroup of the
automorphism group of the Grassmannian $\mathrm{Gr}(n,k)$.
To show the morphism claimed exists, we fist consider what happens when we
apply $\mathrm{Sym}^d$ to $B$, giving a morphism
$$
  \mathrm{Sym}^d(B) : \mathcal{O}_{\mathbf{P}}^{\oplus n} \to
    \pi^*\mathrm{Sym}^d(\mathcal{E})(d) = \pi^*\mathcal{F}(d).
$$
Composing this will the given surjection $\mathcal{F} \to \mathcal{Q}$, we
obtain a morphism
$$
  U : \mathcal{O}_{\mathbf{P}}^{\oplus n} \to
        \pi^*\mathcal{F}(d) \to
        \pi^*\mathcal{Q}(d)
$$
which is surjective away from $D$.
By the universal property of the Grassmannian, we obtain a morphism
$\mathbf{P} \setminus D \to \mathrm{Gr}(n,k)$. It is $G$-equivariant
\end{proof}

\begin{lemma}
\label{lemma-positive-self-intersection}
Let $X$ be a normal projective variety over a field $k$.
Let $\mathcal{E}$ be a semipositive locally free sheaf on $X$ with structure
group $G$.\todo{Make sense of this?}
Set $\mathcal{F} = \mathrm{Sym}^d(\mathcal{E})$ for some $d \geq 1$ and let $n$
be the rank.
Let $\mathcal{Q}$ be a locally free quotient of $\mathcal{F}$ of rank $k$.
Assume that the map\todo{Replace all occurrences of $G$ by $\mathrm{GL}$?}
$$
u : X \to [\mathrm{Gr}(n,k)/G]
$$
is quasi-finite on a nonempty open subset of $X$.
Then $(\det(\mathcal{Q})^{\dim(Y)}) > 0$.
\end{lemma}

\begin{proof}
Consider the given surjection $\mathcal{F} \to \mathcal{Q}$ of locally free
$\mathcal{O}_X$-modules over $X$.
Taking $k$\textsuperscript{th} exterior powers, we obtain a surjection
$$
  \bigwedge\nolimits^k \mathcal{F} \to \det(\mathcal{Q})
$$
from $\bigwedge^k\mathcal{F}$ to the invertible module $\det(\mathcal{Q})$.
Thus we obtain a morphism
$$
  u : X \to \mathbf{P}\big(\bigwedge\nolimits^k \mathcal{F}\big)
$$
such that
$u^*\mathcal{O}_{\mathbf{P}(\bigwedge^k\mathcal{F})}(1) = \det(\mathcal{Q})$.

Let $\mathbf{P} = \mathbf{P}((\mathcal{F}^\vee)^{\oplus n})$ be the projective
bundle associated with $(\mathcal{F}^\vee)^{\oplus n}$ and write
$\pi : \mathbf{P} \to X$ for the projection.
Using the projection formula, Lemma \ref{lemma-projection-formula},
\begin{align*}
H^0(\mathbf{P}, \pi^*\mathcal{F}(1))
  & \cong
H^0(X, \pi_*(\pi^*\mathcal{F} \otimes_{\mathcal{O}_{\mathbf{P}}}
  \mathcal{O}_{\mathbf{P}}(1))) \\
  & \cong
H^0(X,\mathcal{F} \otimes_{\mathcal{O}_X} \pi_*\mathcal{O}_{\mathbf{P}}(1)) \\
  & \cong
H^0(X,\mathcal{F} \otimes (\mathcal{F}^\vee)^{\oplus n}) \\
  & \cong
H^0(X,\mathrm{End}(\mathcal{F})^{\oplus n}).
\end{align*}
The $n$ canonical sections of $H^0(X,\mathrm{End}(\mathcal{F})^{\oplus n})$
give $n$ sections of
$H^0(\mathbf{P}, \pi^*\mathcal{F}
  \otimes_{\mathcal{O}_{\mathbf{P}}} \mathcal{O}_{\mathbf{P}}(1))$ and hence
a morphism
$$
  B : \mathcal{O}_{\mathbf{P}}^{\oplus n} \to \pi^*\mathcal{F}(1)
$$
of sheaves over $\mathbf{P}$.
Let $D \subset \mathbf{P}$ be the support of the cokernel of $B$.
Composing $B$ with the pullback of the given surjection
$\mathcal{F} \to \mathcal{Q}$, we obtain a morphism
$$
  U : \mathcal{O}_{\mathbf{P}}^{\oplus n} \to
        \pi^*\mathcal{F}(1) \to
        \pi^*\mathcal{Q}(1).
$$
By construction, this morphism is surjective away from $D$.
Taking $k$\textsuperscript{th} exterior powers of this morphism, we obtain a
morphism
$$
  \bigwedge^k U : \mathcal{O}_{\mathbf{P}}^{\oplus \binom{n}{k}} \to
    \pi^*\big(\bigwedge\nolimits^k W\big)(k) \to
    \pi^*\det(Q)(k)
$$
which is surjective away from $D$.

Choose a $G$-orbit closure $P$ in $\mathbf{P}$ and write $i : P \to \mathbf{P}$
for the inclusion.\todo{Is it really important that we take closure of the
  projective fibres rather than the entire orbit?}
Write $\overline{\pi} : P \to X$ for the restriction of $\pi$ to $P$.
Pulling back $B$ via $i$, we obtain a map
$$
  \overline{B} : \mathcal{O}_P^{\oplus n} \to \overline{\pi}^* \mathcal{F}(1)
$$
of sheaves of $\mathcal{O}_P$-modules over $P$.
Composing $\overline{B}$ with the pullback of the given surjection
$\mathcal{F} \to \mathcal{Q}$, we obtain a map
$$
  \overline{U} : \mathcal{O}_P^{\oplus n} \to
                  \overline{\pi}^*\mathcal{F}(1) \to
                  \overline{\pi}^*\mathcal{Q}(1).
$$
Being the composition of $\overline{B}$ with a surjection, $\overline{U}$ is
surjective away from $P \cap D$.

Let $H$ be a ample divisor on $Y$. Since $u'$ is generically finite and $\mathcal{O}_{Gr}(1)$ is big, the pullback $u'^*  \mathcal{O}_{Gr}(1)$ is big on $\mathbf{P}'$. \todo{Explain why pullback of big is big} Thus, by Kollar's definition of big, there is a natural number $m>>0$ such that we have a nontrivial section of $u'^*\mathcal{O}_{Gr}(m)\otimes g^*p^*H^{-1}$ i.e. there's a nontrivial morphism 

\begin{align*}
{O}_{\mathbf{P}'}\to & u'^*\mathcal{O}_{Gr}(m)\otimes g^*(p^*H^{-1})=g^*(p^*(\det Q)^{\otimes m}\otimes p^*H^{-1}\otimes \mathcal{O}_{\mathbf{P}}(km))\otimes \mathcal{O}_{\mathbf{P}'}(-mE)\\&\to g^*(p^*(\det Q)^{\otimes m}\otimes p^*H^{-1}\otimes \mathcal{O}_{\mathbf{P}}(km))
\end{align*} 

Then when we pushforward to $Y$ we get $$\mathcal{O}_Y\to p_*g_*\mathcal{O}_{\mathbf{P}'}\to (\det Q)^{\otimes m}\otimes H^{-1}\otimes p_*\mathcal{O}_{\mathbf{P}}(km)$$ \todo{Explain why the composite is nonzero}.

After rearanging we get $$H\otimes (p_*\mathcal{O}_{\mathbf{P}}(km))^{\vee}\stackrel{\tau}\to (\det Q)^{\otimes m}$$ 

If $m$ is sufficiently large, then the natural map $$\pi_*\mathcal{O}_{\mathbb{P}}(mk)\to p_*\mathcal{O}_{\mathbf{P}}(mk)$$ is surjective and so the dual is injective $$p_*\mathcal{O}_{\mathbf{P}}(mk)^*\hookrightarrow \pi_*\mathcal{O}_{\mathbb{P}}(mk)^*$$ Thus, $(p_*\mathcal{O}_{\mathbf{P}}(km))^{\vee}$ is a $G$-subbundle  of $(\pi_*\mathcal{O}_{\mathbb{P}}(km))^*=(Sym^{km}(\sum W^*))^*$, which is semipositive by \todo{Add reference}. Hence, $(p_*\mathcal{O}_{\mathbf{P}}(km))^{\vee}$  itself is semipostive. 

If we blow up the image sheaf of $\tau$, then we get a birational map $s:Y'\to Y$ within the commuative diagram
 
$$\xymatrix{
&\mathbb{P}((p_*\mathcal{O}_{\mathbf{P}}(km)^{\vee})) \ar[d]^{\cong}\\
Y'  \ar[r]|-{v'} \ar[ru]^{\tilde{v}'} \ar[d]^s &\mathbb{P}(H\otimes (p_*\mathcal{O}_{\mathbf{P}}(km)^{\vee})) \\
Y \ar@{.>}[ru]^v &
}$$ 

such that 

$$s^*(\det Q)^{\otimes m}\cong \mathcal{O}_Y(F)\otimes \tilde{v}^*\mathcal{O}_{\mathbb{P}(p_*\mathcal{O}_{\mathbf{P}}(km))^{\vee})}(1)\otimes s^*H$$  Here $F$ is the exceptional divisor of the blow up.

By \todo{add reference} $\tilde{v}^*\mathcal{O}_{\mathbb{P}(p_*\mathcal{O}_{\mathbf{P}}(km))^{\vee})}(1)$ is nef. 
  
So we get   
$$s^*\det(Q)^*=s^*H\otimes \mathcal{O}_{Y'}\otimes \mathcal{O}_{Y'}(F)$$   

Then computing the $t=\dim Y$-fold self-intersection we get 

\begin{align*}
c_1(\det(Q)) & =c_1(s^*\det(Q)^{\otimes m})=(s^*\det(Q)^{\otimes m})^{(t)}\\ & =(H+N+F)^{(t)}=H^{(t)}+\sum\limits_{i=1}^{t} H^{(t-i)}(H+N+F)^{(i-1)}(N+F)
\end{align*} 
 Here $H$ is the pullback of an ample invertible sheaf, $N$ is nef, and $F$ is effective. Then $H+N+F$ is nef because $\det(Q)^{\otimes m}$ is nef using that $Q$ is a quotient of a semipositive sheaf.  Similarly, $N+F$ is effective because it's a sum of effective and nef. Combing we get that the sum $\sum\limits_{i=1}^{(t)} H^{(t-i)}(H+N+F)^{(i-1)}(N+F)$ is at least 0. Additionally, $H^{(t)}>0$ is positive as $H$ is a pullback of an ample sheaf. Overall, the total sum $H^{(t)}+\sum\limits_{i=1}^{(t)} H^{(t-i)}(H+N+F)^{(i-1)}(N+F)$ is positive. 


\end{proof}

\begin{lemma}
Let $k$ be a field.
Let $X$ be a proper algebraic space over $k$.
Let $V$ be a semipositive vector bundle with structure group $G$.
Set $W = \mathrm{Sym}^d(V)$ for some $d \geq 1$ and let $n$ be the rank.
Let $Q$ be a quotient vector bundle of $W$ of rank $k$.
Assume that the map
$$
u : X \to [\mathrm{Gr}(n,k)/G]
$$
is quasi-finite.
Then $\det(Q)$ is ample.\todo{Rework this in terms of locally free sheaves?}
\end{lemma}

\begin{proof}
First we reduce to the case $X$ is normal.
Since $X$ is proper over a field, Lemma \ref{lemma-ubiquity-nagata} shows $X$
is a Nagata algebraic space.
By Lemma \ref{lemma-nagata-normalization}, the normalization
$\nu : X^\nu \to X$ is a finite morphism.
Since $\det(Q)$ is ample if and only if $\nu^*\det(Q)$ is ample by
Lemma ...\todo{Get this lemma from \texttt{ampleness-spaces.tex}.}, we may
assume $X$ is normal.

So assume $X$ is normal.
By the Nakai--Moishezon Criterion for spaces\todo{Reference this somehow.},
we must show that $\det(Q)$ has positive top intersection with
every irreducible subspace $Z \subset X$.
So let $Z \subset X$ be any irreducible subspace and write $i : Z \to X$ for
the inclusion.
By Lemma \ref{lemma-chow-noetherian-separated}, there is a projective
scheme $Z'$ and a proper birational morphism $g : Z' \to Z$.
Now
$$
  (\det(Q)^{\dim(Z)} \cdot Z)
    = (\det(i^*Q)^{\dim(Z)})
    = (\det(g^*i^*Q)^{\dim(Z')})
$$
so we need to show that the top self-intersection of $\det(g^*i^*Q)$ on $Z'$
is positive.
But now $g^*i^*V$ is semipositive by ...\todo{Reference.},
$g^*i^*W \cong \mathrm{Sym}^d(g^*i^*V)$\todo{Obvious?}, and $g^*i^*Q$ is a
quotient bundle of $g^*i^*W$ of rank $k$ on $Z'$.
Also, the map $u \circ i \circ g : Z' \to [\mathrm{Gr}(W,k)/G]$ is quasi-finite
on the dense open on which $g$ is an isomorphism.
Thus Lemma~\ref{lemma-positive-self-intersection},
$(\det(g^*i^*Q)^{\dim(Z')}) > 0$ and we are done.\todo{Do we normalize $Z'$ or
  $X$, actually? Does $X$ normal imply $Z'$ normal? That doesn't seem right...}
  


  
\end{proof}

\bibliographystyle{unsrt}
\bibliography{references}

\end{document}

\input{preamble}
\theoremstyle{definition}
\newtheorem{question}[subsection]{Question}
\newtheorem{steps}[subsection]{Steps}

\newcommand{\todo}[1]{\footnote{\textbf{TODO.} #1}}
\begin{document}
\title{Projectivity of $\overline{M}_g$}
\maketitle

\section{Motivation}
We start with the following question:
\begin{question}
  How does one classify algebraic varieties?
\end{question}
There are two complementary approaches:
\begin{enumerate}
  \item Vertically, discretely: Birational transformations and the minimal model
    program (Italians, Mori, \ldots).
  \item Horizontally, continuously: Moduli theory (Riemann, Teichm\"uller,
    Mumford, \ldots).
\end{enumerate}
Our project is about moduli spaces, specifically:
\begin{example}
  $\overline{\mathcal{M}}_g$, $g \ge 2$, the moduli space of \emph{stable}
  curves.
  \begin{enumerate}
    \item $\overline{\mathcal{M}}_g$ is a proper stack;
    \item{}[Keel--Mori] There is a coarse moduli space $\overline{\mathcal{M}}_g
      \to \overline{M}_g$ (proper algebraic space).
  \end{enumerate}
\end{example}
Our main goal:
\begin{theorem}
  $\overline{M}_g$ is a projective \emph{scheme} over $\Spec \mathbf{Z}$. 
\end{theorem}
Three approaches:
\begin{enumerate}
  \item Hodge theory ($/\mathbf{C}$, Mumford): use the Jacobian map
    $\overline{M}_g \to \overline{A}_g$ and study $\overline{A}_g$.
  \item GIT (Mumford, Gieseker): translate to representation theory.
  \item Koll\'ar: ``Just'' find an ample line bundle.
\end{enumerate}
The advantage of Koll\'ar's approach is that it is applicable to more moduli
problems: there is no hard GIT involved, and it works over arbitrary fields or
$\mathbf{Z}$.
\begin{situation}
  For us, we have the universal family
  $$
  \xymatrix{\mathcal{C}\ar[d]^\pi\\
  \overline{\mathcal{M}}_g}
  $$
  We consider
  $$
  \lambda_k = \det(\pi_*
  \omega_{\mathcal{C}/\overline{\mathcal{M}}_g}^{\otimes k}).
  $$
  It is a bundle, and descends to a $\mathbf{Q}$-bundle on $\overline{M}_g$.
\end{situation}
For other moduli problems, Koll\'ar's approach would be to find a suitable guess
for what is an ample bundle.
\begin{steps}\leavevmode
  \begin{enumerate}
    \item {\bf Semipositivity.} Let $k$ be a field, and
      $$
      \xymatrix{S \ar[d]^f\\C}
      $$
      a family of stable curves. Then, $f_*(\omega_{S/C}^{\otimes m})$ is
      semipositive for all $m \ge 2$, i.e., it has no quotients of negative
      degree.
      \begin{itemize}
        \item A vanishing theorem for $H^1$ (Ekedahl);
        \item Classification of surfaces (Kodaira--Enriques, Bombieri--Mumford).
      \end{itemize}
    \item {\bf Ampleness.} 
      Let $\mathcal{E}$ be a semipositive vector bundle of
      rank $m$ 
  \end{enumerate}
\end{steps}
\end{document}

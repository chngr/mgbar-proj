\IfFileExists{stacks-project.cls}{%
\documentclass{stacks-project}
}{%
\documentclass{amsart}
}

% The following AMS packages are automatically loaded with
% the amsart documentclass:
%\usepackage{amsmath}
%\usepackage{amssymb}
%\usepackage{amsthm}

% For dealing with references we use the comment environment
\usepackage{verbatim}
\newenvironment{reference}{\comment}{\endcomment}
%\newenvironment{reference}{}{}
\newenvironment{slogan}{\comment}{\endcomment}
\newenvironment{history}{\comment}{\endcomment}

% For commutative diagrams you can use
% \usepackage{amscd}
\usepackage[all]{xy}

% We use 2cell for 2-commutative diagrams.
\xyoption{2cell}
\UseAllTwocells

% To put source file link in headers.
% Change "template.tex" to "this_filename.tex"
% \usepackage{fancyhdr}
% \pagestyle{fancy}
% \lhead{}
% \chead{}
% \rhead{Source file: \url{template.tex}}
% \lfoot{}
% \cfoot{\thepage}
% \rfoot{}
% \renewcommand{\headrulewidth}{0pt}
% \renewcommand{\footrulewidth}{0pt}
% \renewcommand{\headheight}{12pt}

\usepackage{multicol}

% For cross-file-references
\usepackage{xr-hyper}

% Package for hypertext links:
\usepackage{hyperref}

% For any local file, say "hello.tex" you want to link to please
% use \externaldocument[hello-]{hello}
\externaldocument[introduction-]{introduction}
\externaldocument[conventions-]{conventions}
\externaldocument[sets-]{sets}
\externaldocument[categories-]{categories}
\externaldocument[topology-]{topology}
\externaldocument[sheaves-]{sheaves}
\externaldocument[sites-]{sites}
\externaldocument[stacks-]{stacks}
\externaldocument[fields-]{fields}
\externaldocument[algebra-]{algebra}
\externaldocument[brauer-]{brauer}
\externaldocument[homology-]{homology}
\externaldocument[derived-]{derived}
\externaldocument[simplicial-]{simplicial}
\externaldocument[more-algebra-]{more-algebra}
\externaldocument[smoothing-]{smoothing}
\externaldocument[modules-]{modules}
\externaldocument[sites-modules-]{sites-modules}
\externaldocument[injectives-]{injectives}
\externaldocument[cohomology-]{cohomology}
\externaldocument[sites-cohomology-]{sites-cohomology}
\externaldocument[dga-]{dga}
\externaldocument[dpa-]{dpa}
\externaldocument[hypercovering-]{hypercovering}
\externaldocument[schemes-]{schemes}
\externaldocument[constructions-]{constructions}
\externaldocument[properties-]{properties}
\externaldocument[morphisms-]{morphisms}
\externaldocument[coherent-]{coherent}
\externaldocument[divisors-]{divisors}
\externaldocument[limits-]{limits}
\externaldocument[varieties-]{varieties}
\externaldocument[topologies-]{topologies}
\externaldocument[descent-]{descent}
\externaldocument[perfect-]{perfect}
\externaldocument[more-morphisms-]{more-morphisms}
\externaldocument[flat-]{flat}
\externaldocument[groupoids-]{groupoids}
\externaldocument[more-groupoids-]{more-groupoids}
\externaldocument[etale-]{etale}
\externaldocument[chow-]{chow}
\externaldocument[intersection-]{intersection}
\externaldocument[pic-]{pic}
\externaldocument[adequate-]{adequate}
\externaldocument[dualizing-]{dualizing}
\externaldocument[duality-]{duality}
\externaldocument[discriminant-]{discriminant}
\externaldocument[local-cohomology-]{local-cohomology}
\externaldocument[curves-]{curves}
\externaldocument[resolve-]{resolve}
\externaldocument[models-]{models}
\externaldocument[pione-]{pione}
\externaldocument[etale-cohomology-]{etale-cohomology}
\externaldocument[proetale-]{proetale}
\externaldocument[crystalline-]{crystalline}
\externaldocument[spaces-]{spaces}
\externaldocument[spaces-properties-]{spaces-properties}
\externaldocument[spaces-morphisms-]{spaces-morphisms}
\externaldocument[decent-spaces-]{decent-spaces}
\externaldocument[spaces-cohomology-]{spaces-cohomology}
\externaldocument[spaces-limits-]{spaces-limits}
\externaldocument[spaces-divisors-]{spaces-divisors}
\externaldocument[spaces-over-fields-]{spaces-over-fields}
\externaldocument[spaces-topologies-]{spaces-topologies}
\externaldocument[spaces-descent-]{spaces-descent}
\externaldocument[spaces-perfect-]{spaces-perfect}
\externaldocument[spaces-more-morphisms-]{spaces-more-morphisms}
\externaldocument[spaces-flat-]{spaces-flat}
\externaldocument[spaces-groupoids-]{spaces-groupoids}
\externaldocument[spaces-more-groupoids-]{spaces-more-groupoids}
\externaldocument[bootstrap-]{bootstrap}
\externaldocument[spaces-pushouts-]{spaces-pushouts}
\externaldocument[groupoids-quotients-]{groupoids-quotients}
\externaldocument[spaces-more-cohomology-]{spaces-more-cohomology}
\externaldocument[spaces-simplicial-]{spaces-simplicial}
\externaldocument[spaces-duality-]{spaces-duality}
\externaldocument[formal-spaces-]{formal-spaces}
\externaldocument[restricted-]{restricted}
\externaldocument[spaces-resolve-]{spaces-resolve}
\externaldocument[formal-defos-]{formal-defos}
\externaldocument[defos-]{defos}
\externaldocument[cotangent-]{cotangent}
\externaldocument[examples-defos-]{examples-defos}
\externaldocument[algebraic-]{algebraic}
\externaldocument[examples-stacks-]{examples-stacks}
\externaldocument[stacks-sheaves-]{stacks-sheaves}
\externaldocument[criteria-]{criteria}
\externaldocument[artin-]{artin}
\externaldocument[quot-]{quot}
\externaldocument[stacks-properties-]{stacks-properties}
\externaldocument[stacks-morphisms-]{stacks-morphisms}
\externaldocument[stacks-limits-]{stacks-limits}
\externaldocument[stacks-cohomology-]{stacks-cohomology}
\externaldocument[stacks-perfect-]{stacks-perfect}
\externaldocument[stacks-introduction-]{stacks-introduction}
\externaldocument[stacks-more-morphisms-]{stacks-more-morphisms}
\externaldocument[stacks-geometry-]{stacks-geometry}
\externaldocument[moduli-]{moduli}
\externaldocument[moduli-curves-]{moduli-curves}
\externaldocument[examples-]{examples}
\externaldocument[exercises-]{exercises}
\externaldocument[guide-]{guide}
\externaldocument[desirables-]{desirables}
\externaldocument[coding-]{coding}
\externaldocument[obsolete-]{obsolete}
\externaldocument[fdl-]{fdl}
\externaldocument[index-]{index}

% Theorem environments.
%
\theoremstyle{plain}
\newtheorem{theorem}[subsection]{Theorem}
\newtheorem{proposition}[subsection]{Proposition}
\newtheorem{lemma}[subsection]{Lemma}

\theoremstyle{definition}
\newtheorem{definition}[subsection]{Definition}
\newtheorem{example}[subsection]{Example}
\newtheorem{exercise}[subsection]{Exercise}
\newtheorem{situation}[subsection]{Situation}

\theoremstyle{remark}
\newtheorem{remark}[subsection]{Remark}
\newtheorem{remarks}[subsection]{Remarks}

\numberwithin{equation}{subsection}

% Macros
%
\def\lim{\mathop{\rm lim}\nolimits}
\def\colim{\mathop{\rm colim}\nolimits}
\def\Spec{\mathop{\rm Spec}}
\def\Hom{\mathop{\rm Hom}\nolimits}
\def\Ext{\mathop{\rm Ext}\nolimits}
\def\SheafHom{\mathop{\mathcal{H}\!{\it om}}\nolimits}
\def\SheafExt{\mathop{\mathcal{E}\!{\it xt}}\nolimits}
\def\Sch{\textit{Sch}}
\def\Mor{\mathop{\rm Mor}\nolimits}
\def\Ob{\mathop{\rm Ob}\nolimits}
\def\Sh{\mathop{\textit{Sh}}\nolimits}
\def\NL{\mathop{N\!L}\nolimits}
\def\proetale{{pro\text{-}\acute{e}tale}}
\def\etale{{\acute{e}tale}}
\def\QCoh{\textit{QCoh}}
\def\Ker{\mathop{\rm Ker}}
\def\Im{\mathop{\rm Im}}
\def\Coker{\mathop{\rm Coker}}
\def\Coim{\mathop{\rm Coim}}

%
% Macros for moduli stacks/spaces
%
\def\QCohstack{\mathcal{QC}\!{\it oh}}
\def\Cohstack{\mathcal{C}\!{\it oh}}
\def\Spacesstack{\mathcal{S}\!{\it paces}}
\def\Quotfunctor{{\rm Quot}}
\def\Hilbfunctor{{\rm Hilb}}
\def\Curvesstack{\mathcal{C}\!{\it urves}}
\def\Polarizedstack{\mathcal{P}\!{\it olarized}}
\def\Complexesstack{\mathcal{C}\!{\it omplexes}}
% \Pic is the operator that assigns to X its picard group, usage \Pic(X)
% \Picardstack_{X/B} denotes the Picard stack of X over B
% \Picardfunctor_{X/B} denotes the Picard functor of X over B
\def\Pic{\mathop{\rm Pic}\nolimits}
\def\Picardstack{\mathcal{P}\!{\it ic}}
\def\Picardfunctor{{\rm Pic}}
\def\Deformationcategory{\mathcal{D}\!{\it ef}}



\newcommand{\todo}[1]{\footnote{\textbf{TODO.} #1}}
\begin{document}
\title{Ampleness Criteria for algebraic spaces}
\maketitle

\section{Facts from schemes}
\begin{lemma}[{\cite[\href{http://stacks.math.columbia.edu/tag/0B5V}{Tag 0B5V}]{stacks-project}}]
\label{cohomology-schemes-lemma-surjective-finite-morphism-ample}
Let $R$ be a Noetherian ring. Let $f : Y \to X$ be a morphism of
schemes proper over $R$. Let $\mathcal{L}$ be an
invertible $\mathcal{O}_X$-module. Assume $f$ is finite and surjective.
Then $\mathcal{L}$ is ample if and only if $f^*\mathcal{L}$ is ample.
\end{lemma}

\section{Basic properties of algebraic spaces}

\begin{lemma}[{\cite[\href{http://stacks.math.columbia.edu/tag/025W}{Tag 025W}]{stacks-project}}]
\label{lemma-representable-diagonal}
Let $S$ be a scheme contained in $\Sch_{fppf}$.
Let $F$ be a presheaf of sets on $(\Sch/S)_{fppf}$.
The following are equivalent:
\begin{enumerate}
\item the diagonal $F \to F \times F$ is representable,
\item for $U \in \Ob((\Sch/S)_{fppf})$ and any $a \in F(U)$
the map $a : h_U \to F$ is representable,
\item for every pair $U, V \in \Ob((\Sch/S)_{fppf})$
and any $a \in F(U)$, $b \in F(V)$ the fibre product
$h_U \times_{a, F, b} h_V$ is representable.
\end{enumerate}
\end{lemma}

\begin{lemma}[{\cite[\href{http://stacks.math.columbia.edu/tag/03WG}{Tag 03WG}]{stacks-project}}]
\label{lemma-affine-local}
Let $S$ be a scheme.
Let $f : X \to Y$ be a morphism of algebraic spaces over $S$.
The following are equivalent:
\begin{enumerate}
\item $f$ is representable and affine,
\item $f$ is affine,
\item for every affine scheme $V$ and \'etale morphism $V \to Y$
the scheme $X \times_Y V$ is affine,
\item there exists a scheme $V$ and a surjective \'etale morphism
$V \to Y$ such that $V \times_Y X \to V$ is affine, and
\item there exists a Zariski covering $Y = \bigcup Y_i$ such
that each of the morphisms $f^{-1}(Y_i) \to Y_i$ is affine.
\end{enumerate}
\end{lemma}

\begin{lemma}[{\cite[\href{http://stacks.math.columbia.edu/tag/07UT}{Tag 07UT}]{stacks-project}}]
\label{lemma-property-higher-rank-cohomological}
Let $S$ be a scheme. Let $X$ be a Noetherian algebraic space over $S$.
Let $\mathcal{P}$ be a property of coherent sheaves on $X$. Assume
\begin{enumerate}
\item For any short exact sequence of coherent sheaves
$$
0 \to \mathcal{F}_1 \to \mathcal{F} \to \mathcal{F}_2 \to 0
$$
if $\mathcal{F}_i$, $i = 1, 2$ have property $\mathcal{P}$
then so does $\mathcal{F}$.
\item If $\mathcal{P}$ holds for $\mathcal{F}^{\oplus r}$ for
some $r \geq 1$, then it holds for $\mathcal{F}$.
\item For every reduced closed subspace $i : Z \to X$ with
$|Z|$ irreducible there exists a coherent sheaf $\mathcal{G}$ on $Z$
such that
\begin{enumerate}
\item $\text{Supp}(\mathcal{G}) = Z$,
\item for every nonzero quasi-coherent sheaf of ideals
$\mathcal{I} \subset \mathcal{O}_Z$ there exists a quasi-coherent
subsheaf $\mathcal{G}' \subset \mathcal{I}\mathcal{G}$ such that
$\text{Supp}(\mathcal{G}/\mathcal{G}')$ is proper closed in $Z$
and such that $\mathcal{P}$ holds for $i_*\mathcal{G}'$.
\end{enumerate}
\end{enumerate}
Then property $\mathcal{P}$ holds for every coherent sheaf on $X$.
\end{lemma}

\section{Ampleness for spaces}
\begin{definition}[{\cite[\href{http://stacks.math.columbia.edu/tag/0D31}{Tag 0D31}]{stacks-project}}]\label{definition-relatively-ample}
Let $S$ be a scheme.
Let $f : X \to Y$ be a morphism of algebraic spaces over $S$.
Let $\mathcal{L}$ be an invertible $\mathcal{O}_X$-module.
We say $\mathcal{L}$ is {\it relatively ample}, or {\it $f$-relatively ample},
or {\it ample on $X/Y$}, or {\it $f$-ample} if $f : X \to Y$
is representable and for every morphism $Z \to Y$
where $Z$ is a scheme, the pullback $\mathcal{L}_T$\todo{Report typo} of $\mathcal{L}$
to $X_Z = Z \times_Y X$ is ample on $X_Z/Z$.
\end{definition}

\begin{definition}[{\cite[II, Def.\ 7.9]{Kn}}]\label{definition-knutson-ample}
  Let $f : X \to Y$ be a map of algebraic spaces. Let $\mathcal{L}$ be an
  invertible sheaf on $X$. We say $\mathcal{L}$ is {\it $f$-ample} if there
  exists a factorization
  $$
  \xymatrix{
    X \ar@{^{(}->}[r]^i\ar[dr]_f & \mathbf{P}^n_Y \ar[d]\\
    & Y
  }
  $$
  where $i$ is an immersion, such that for some integer $k$, we have
  $\mathcal{L}^{\otimes k} \simeq i^*\mathcal{O}(1)$, where $\mathcal{O}(1)$ is
  the canonical sheaf on $\mathbf{P}^n_k$.\todo{Is this equivalent to Definition
  \ref{definition-relatively-ample}?}
\end{definition}

\begin{lemma}[{\cite[\href{http://stacks.math.columbia.edu/tag/0D32}{Tag 0D32}]{stacks-project}}]
\label{lemma-relatively-ample-sanity-check}
Let $S$ be a scheme.
Let $f : X \to Y$ be a morphism of algebraic spaces over $S$.
Let $\mathcal{L}$ be an invertible $\mathcal{O}_X$-module.
Assume $Y$ is a scheme. The following are equivalent
\begin{enumerate}
\item $\mathcal{L}$ is ample on $X/Y$ in the sense of
Definition \ref{definition-relatively-ample}, and
\item $X$ is a scheme and $\mathcal{L}$ is ample on $X/Y$
in the sense of
Morphisms, Definition \ref{morphisms-definition-relatively-ample}.
\end{enumerate}
\end{lemma}

\begin{lemma}[{\cite[\href{http://stacks.math.columbia.edu/tag/0D34}{Tag 0D34}]{stacks-project}}]
\label{lemma-relatively-ample-properties}
Let $S$ be a scheme.
Let $f : X \to Y$ be a morphism of algebraic spaces over $S$.
If there exists an $f$-ample invertible sheaf, then
$f$ is representable, quasi-compact, and separated.
\end{lemma}

\begin{lemma}[Serre vanishing for algebraic spaces, {\cite[\href{http://stacks.math.columbia.edu/tag/0D38}{Tag 0D38}]{stacks-project}}]
Let $R$ be a Noetherian ring. Let $X$ be an algebraic space over $R$
such that the structure morphism $f : X \to \Spec(R)$ is proper.
Let $\mathcal{L}$ be an invertible $\mathcal{O}_X$-module.
The following are equivalent
\begin{enumerate}
\item $\mathcal{L}$ is ample on $X/R$
(Definition \ref{definition-relatively-ample}),
\item for every coherent $\mathcal{O}_X$-module $\mathcal{F}$
there exists an $n_0 \geq 0$ such that
$H^p(X, \mathcal{F} \otimes \mathcal{L}^{\otimes n}) = 0$
for all $n \geq n_0$ and $p > 0$.
\end{enumerate}
\end{lemma}

\begin{lemma}[{\cite[\href{http://stacks.math.columbia.edu/tag/0D2W}{Tag 0D2W}]{stacks-project}}]
\label{lemma-Noetherian-h1-zero-invertible}
Let $S$ be a scheme. Let $X$ be a Noetherian algebraic space over $S$.
Let $\mathcal{L}$ be an invertible $\mathcal{O}_X$-module.
Assume that for every coherent $\mathcal{O}_X$-module
$\mathcal{F}$ there exists an $n \geq 1$ such that
$H^1(X, \mathcal{F} \otimes_{\mathcal{O}_X} \mathcal{L}^{\otimes n}) = 0$.
Then $X$ is a scheme and $\mathcal{L}$ is ample on $X$.
\end{lemma}

\section{Euler characteristics for spaces}
\begin{lemma}[{cf.\ \cite[\href{http://stacks.math.columbia.edu/tag/0BEK}{Tag 0BEK}]{stacks-project}}]
\label{lemma-euler-characteristic-morphism}
Let $k$ be a field. Let $f : Y \to X$ be a morphism of proper algebraic spaces
over $k$. Let $\mathcal{G}$ be a coherent $\mathcal{O}_Y$-module. Then
$$
\chi(Y, \mathcal{G}) = \sum (-1)^i \chi(X, R^if_*\mathcal{G})
$$
\end{lemma}

\begin{proof}
The formula makes sense: the sheaves $R^if_*\mathcal{G}$ are coherent
and only a finite number of them are nonzero, see
\cite[\href{http://stacks.math.columbia.edu/tag/08AR}{Tag 08AR}]{stacks-project}
and \cite[\href{http://stacks.math.columbia.edu/tag/073G}{Tag
073G}]{stacks-project}. By the Leray spectral sequence
\cite[\href{http://stacks.math.columbia.edu/tag/0732}{Tag 0732}]{stacks-project}
there is a spectral
sequence with
$$
E_2^{p, q} = H^p(X, R^qf_*\mathcal{G})
$$
converging to $H^{p + q}(Y, \mathcal{G})$. By finiteness of cohomology
on $X$ we see that only a finite number of $E_2^{p, q}$ are nonzero
and each $E_2^{p, q}$ is a finite dimensional vector space. It follows
that the same is true for $E_r^{p, q}$ for $r \geq 2$ and that
$$
\sum (-1)^{p + q} \dim_k E_r^{p, q}
$$
is independent of $r$. Since for $r$ large enough we have
$E_r^{p, q} = E_\infty^{p, q}$ and since convergence means there
is a filtration on $H^n(Y, \mathcal{G})$ whose graded pieces are
$E_\infty^{p, q}$ with $p + 1 = n$ (this is the meaning of convergence
of the spectral sequence), we conclude.
\end{proof}

\section{Numerical intersection theory for spaces}
\begin{lemma}[{\cite[\href{http://stacks.math.columbia.edu/tag/0DN4}{Tag 0DN4}]{stacks-project}}]\label{lemma-numerical-polynomial-from-euler}
Let $k$ be a field. Let $X$ be a proper algebraic space over $k$.
Let $\mathcal{F}$ be a coherent $\mathcal{O}_X$-module. Let
$\mathcal{L}_1, \ldots, \mathcal{L}_r$ be invertible $\mathcal{O}_X$-modules.
The map
$$
(n_1, \ldots, n_r) \longmapsto
\chi(X, \mathcal{F} \otimes
\mathcal{L}_1^{\otimes n_1} \otimes \ldots \otimes
\mathcal{L}_r^{\otimes n_r})
$$
is a numerical polynomial in $n_1, \ldots, n_r$ of total degree at
most the dimension of the scheme theoretic support of $\mathcal{F}$.
\end{lemma}
We can then define:
\begin{definition}[{cf.\ \cite[\href{http://stacks.math.columbia.edu/tag/0BEP}{Tag 0BEP}]{stacks-project}}]\label{tag:0BEP}
Let $k$ be a field. Let $X$ be a proper algebraic space over $k$. Let
$i : Z \to X$ be a closed subspace of dimension $d$. Let
$\mathcal{L}_1, \ldots, \mathcal{L}_d$ be invertible
$\mathcal{O}_X$-modules. We define the {\it intersection number}
$(\mathcal{L}_1 \cdots \mathcal{L}_d \cdot Z)$
as the coefficient of $n_1 \ldots n_d$ in the numerical polynomial
$$
\chi(X, i_*\mathcal{O}_Z \otimes \mathcal{L}_1^{\otimes n_1} \otimes
\ldots \otimes \mathcal{L}_d^{\otimes n_d}) =
\chi(Z, \mathcal{L}_1^{\otimes n_1} \otimes
\ldots \otimes \mathcal{L}_d^{\otimes n_d}|_Z)
$$
In the special
case that $\mathcal{L}_1 = \ldots = \mathcal{L}_d = \mathcal{L}$
we write $(\mathcal{L}^d \cdot Z)$.
\end{definition}
\begin{lemma}[{cf.\ \cite[\href{http://stacks.math.columbia.edu/tag/0BET}{Tag 0BET}]{stacks-project}}]
\label{lemma-intersection-number-and-pullback}
Let $k$ be a field. Let $f : Y \to X$ be a morphism of proper algebraic spaces
over $k$.
Let $Z \subset Y$ be an integral closed subspace of dimension $d$ and let
$\mathcal{L}_1, \ldots, \mathcal{L}_d$ be invertible $\mathcal{O}_X$-modules.
Then
$$
(f^*\mathcal{L}_1 \cdots f^*\mathcal{L}_d \cdot Z) =
\deg(f|_Z : Z \to f(Z)) (\mathcal{L}_1 \cdots \mathcal{L}_d \cdot f(Z))
$$
where $\deg(Z \to f(Z))$ is as in
{\it \cite[\href{http://stacks.math.columbia.edu/tag/0AD6}{Tag 0AD6}]{stacks-project}}
or $0$ if $\dim(f(Z)) < d$.\todo{Prove this.}
\end{lemma}
%\begin{proof}
%The left hand side is computed using the coefficient of $n_1 \ldots n_d$
%in the function
%$$
%\chi(Y, \mathcal{O}_Z \otimes f^*\mathcal{L}_1^{\otimes n_1} \otimes
%\ldots \otimes f^*\mathcal{L}_d^{\otimes n_d}) =
%\sum (-1)^i
%\chi(X, R^if_*\mathcal{O}_Z \otimes
%\mathcal{L}_1^{\otimes n_1} \otimes \ldots \otimes
%\mathcal{L}_d^{\otimes n_d})
%$$
%The equality follows from Lemma \ref{lemma-euler-characteristic-morphism}
%and the projection formula
%\cite[\href{http://stacks.math.columbia.edu/tag/0944}{Tag 0944}]{stacks-project}.
%If $f(Z)$ has dimension $< d$, then the right hand side
%is a polynomial of total degree $<d$ by
%Lemma \ref{lemma-numerical-polynomial-from-euler}
%and the result is true. Assume $\dim(f(Z)) = d$. Let
%$\xi \in \lvert f(Z) \rvert$ be the generic point. By
%dimension theory
%the generic point of $Z$ is the unique point of $Z$ mapping to $\xi$.
%Then $f : Z \to f(Z)$ is finite over a nonempty open of $f(Z)$, see
%Morphisms, Lemma \ref{morphisms-lemma-generically-finite}.
%Thus $\deg(f : Z \to f(Z))$ is defined and in fact it is equal
%to the length of the stalk of $f_*\mathcal{O}_Z$ at $\xi$
%over $\mathcal{O}_{X, \xi}$. Moreover, the stalk of
%$R^if_*\mathcal{O}_X$ at $\xi$ is zero for $i > 0$ because
%we just saw that $f|_Z$ is finite in a neighbourhood of $\xi$
%(so that Cohomology of Schemes, Lemma
%\ref{coherent-lemma-finite-pushforward-coherent} gives the vanishing).
%Thus the terms $\chi(X, R^if_*\mathcal{O}_Z \otimes
%\mathcal{L}_1^{\otimes n_1} \otimes \ldots \otimes
%\mathcal{L}_d^{\otimes n_d})$ with $i > 0$ have total
%degree $< d$ and
%$$
%\chi(X, f_*\mathcal{O}_Z \otimes
%\mathcal{L}_1^{\otimes n_1} \otimes \ldots \otimes
%\mathcal{L}_d^{\otimes n_d})
%=
%\deg(f : Z \to f(Z)) \chi(f(Z),
%\mathcal{L}_1^{\otimes n_1} \otimes \ldots \otimes
%\mathcal{L}_d^{\otimes n_d}|_{f(Z)})
%$$
%modulo a polynomial of total degree $< d$ by
%Lemma \ref{lemma-numerical-polynomial-leading-term}.
%The desired result follows.
%\end{proof}

\section{Finiteness and cohomology}
This is a generalization of \cite[Lem.\ 2.8]{Ko90}.
\begin{proposition}[{\cite[\href{http://stacks.math.columbia.edu/tag/09YC}{Tag 09YC}]{stacks-project}}]
\label{proposition-there-is-a-scheme-finite-over}
Let $S$ be a scheme. Let $X$ be a quasi-compact and quasi-separated
algebraic space over $S$.
\begin{enumerate}
\item There exists a surjective finite morphism $Y \to X$
of finite presentation where $Y$ is a scheme,
\item given a surjective \'etale morphism $U \to X$ we may choose
$Y \to X$ such that for every $y \in Y$ there is an open neighbourhood
$V \subset Y$ such that $V \to X$ factors through $U$.
\end{enumerate}
\end{proposition}

\begin{lemma}[{\cite[\href{http://stacks.math.columbia.edu/tag/0A4K}{Tag 0A4K}]{stacks-project}}]
\label{lemma-finite-higher-direct-image-zero}
Let $S$ be a scheme. Let $f : X \to Y$ be an integral (for example finite)
morphism of algebraic spaces. Then
$f_* : \textit{Ab}(X_\etale) \to \textit{Ab}(Y_\etale)$
is an exact functor and $R^pf_* = 0$ for $p > 0$.
\end{lemma}

\begin{lemma}[{\cite[\href{http://stacks.math.columbia.edu/tag/0733}{Tag 0733}]{stacks-project}}]
\label{lemma-apply-Leray}
Let $f : (\Sh(\mathcal{C}), \mathcal{O}_\mathcal{C}) \to
(\Sh(\mathcal{D}), \mathcal{O}_\mathcal{D})$ be a morphism of ringed topoi.
Let $\mathcal{F}$ be an $\mathcal{O}_\mathcal{C}$-module.
\begin{enumerate}
\item If $R^qf_*\mathcal{F} = 0$ for $q > 0$, then
$H^p(\mathcal{C}, \mathcal{F}) = H^p(\mathcal{D}, f_*\mathcal{F})$ for all $p$.
\item If $H^p(\mathcal{D}, R^qf_*\mathcal{F}) = 0$ for all $q$ and $p > 0$,
then $H^q(\mathcal{C}, \mathcal{F}) = H^0(\mathcal{D}, R^qf_*\mathcal{F})$
for all $q$.
\end{enumerate}
\end{lemma}



\section{Nakai--Moishezon Criterion}

\begin{lemma}
\label{lemma-surjective-finite-morphism-ample}
Let $R$ be a Noetherian ring.
Let $X$ and $Y$ be algebraic spaces over $R$.
Let $f : Y \to X$ be a proper morphism of algebraic spaces over $R$.
Let $\mathcal{L}$ be an invertible $\mathcal{O}_X$-module.
Assume $f$ is finite and surjective.
Then $\mathcal{L}$ is ample if and only if $f^*\mathcal{L}$ is ample.
\end{lemma}

\begin{proof}
  Suppose that $\mathcal{L}$ is ample.
  Then $X \to \Spec(R)$ is representable and hence $X$ is a scheme.
  But $f : Y \to X$ is finite and hence affine, so, by
  Lemma \ref{lemma-affine-local}, $f$ is representable.
  Therefore $Y$ is a scheme.
  Then $f^*\mathcal{L}$ is ample by the schemes case,
  Lemma \ref{cohomology-schemes-lemma-surjective-finite-morphism-ample}.

\medskip\noindent
Assume that $f^*\mathcal{L}$ is ample.
Let $P$ be the following property on coherent $\mathcal{O}_X$-modules
$\mathcal{F}$:
there exists an $n_0$ such that $H^p(X, \mathcal{F} \otimes
\mathcal{L}^{\otimes n}) = 0$ for all $n \geq n_0$ and $p > 0$.
We will prove that $P$ holds for any coherent $\mathcal{O}_X$-module
$\mathcal{F}$, which suffices to prove that $\mathcal{L}$ is ample.
We are going to apply Lemma \ref{lemma-property-higher-rank-cohomological}.
Thus we have to verify (1), (2) and (3) of that lemma for $P$.
Property (1) follows from the long exact cohomology sequence associated
to a short exact sequence of sheaves and the fact that tensoring with
an invertible sheaf is an exact functor.
Property (2) follows since $H^p(X, -)$ is an additive functor.
To see (3) let $Z \subset X$ be an integral closed subscheme with generic point
$\xi$.
Let $\mathcal{F}$ be a coherent sheaf on $Y$ such that the support of
$f_*\mathcal{F}$ is equal to $Z$ and $(f_*\mathcal{F})_\xi$ is annihilated by
$\mathfrak m_\xi$,
see Lemma \ref{lemma-finite-morphism-Noetherian}. We claim that
taking $\mathcal{G} = f_*\mathcal{F}$ works. We only have to verify
part (3)(c) of Lemma \ref{lemma-property-higher-rank-cohomological}.
Hence assume that $\mathcal{J} \subset \mathcal{O}_X$ is a
quasi-coherent sheaf of ideals such that
$\mathcal{J}_\xi = \mathcal{O}_{X, \xi}$.
A finite morphism is affine hence by
Lemma \ref{lemma-affine-morphism-projection-ideal} we see that
$\mathcal{J}\mathcal{G} = f_*(f^{-1}\mathcal{J}\mathcal{F})$.
Also, as pointed out in the proof of
Lemma \ref{lemma-affine-morphism-projection-ideal} the sheaf
$f^{-1}\mathcal{J}\mathcal{F}$ is a coherent $\mathcal{O}_Y$-module.
By assumption we see that there exists an $n_0$ such that
$$
H^p(Y, f^{-1}\mathcal{J}\mathcal{F}
\otimes_{\mathcal{O}_Y} f^*\mathcal{L}^{\otimes n}) = 0,
$$
for $n \geq n_0$ and $p > 0$. Since $f$ is finite, hence affine, we see that
\begin{align*}
H^p(X, \mathcal{J}\mathcal{G} \otimes_{\mathcal{O}_X}
\mathcal{L}^{\otimes n})
& =
H^p(X, f_*(f^{-1}\mathcal{J}\mathcal{F}) \otimes_{\mathcal{O}_X}
\mathcal{L}^{\otimes n}) \\
& =
H^p(X, f_*(f^{-1}\mathcal{J}\mathcal{F}) \otimes_{\mathcal{O}_Y}
f^*\mathcal{L}^{\otimes n})) \\
& =
H^p(Y, f^{-1}\mathcal{J}\mathcal{F} \otimes_{\mathcal{O}_Y}
f^*\mathcal{L}^{\otimes n}) = 0
\end{align*}
by references cited earlier in this proof.
Hence the quasi-coherent subsheaf $\mathcal{G}' = \mathcal{J}\mathcal{G}$
satisfies $P$. This verifies property (3)(c) of
Lemma \ref{lemma-property-higher-rank-cohomological} as desired.
\end{proof}

\begin{theorem}[Nakai--Moishezon Criterion for algebraic spaces {\cite[Thm.\ 3.11]{Ko90}}]
  Let $k$ be a field. Let $X$ be a proper algebraic space over $k$. Let
  $\mathcal{L}$ be an invertible $\mathcal{O}_X$-module. Then the following are
  equivalent:
  \begin{enumerate}
    \item $\mathcal{L}$ is ample on $X/k$.
    \item For every integral closed subspace $Y$ of $X$, $(\mathcal{L}^{\dim(Y)}
      \cdot Y) > 0$.
  \end{enumerate}
\end{theorem}
\begin{proof}
  Assume (1) holds. Then, $X$ is a scheme, and $\mathcal{L}$ is ample on $X/k$
  in the scheme-theoretic sense by Lemma
  \ref{lemma-relatively-ample-sanity-check}.
  The Nakai--Moishezon criterion for schemes implies (2).

  Now suppose (2) holds.
  By Lemma~\ref{proposition-there-is-a-scheme-finite-over}, we can choose
  a finite surjective map $p : X' \to X$ from a scheme $X'$.
  By Lemma~\ref{lemma-surjective-finite-morphism-ample}, $\mathcal{L}$ is
  ample if and only if $p^*\mathcal{L}$ is ample, so by the Nakai--Moishezon
  criterion for schemes, it suffices to show that for every integral closed
  subscheme $Y'$ in $X'$, we have $(p^*\mathcal{L}^{\dim(Y')} \cdot Y') > 0$.
  Let $Y$ be the image of $Y'$ in $Y$; by Lemma
  \ref{lemma-intersection-number-and-pullback}, we then have
  $$
  ( (p^*\mathcal{L})^{\dim Y'} \cdot Y') = \deg(p\rvert_{Y'} : Y' \to
  Y)(\mathcal{L}^{\dim Y'} \cdot Y).
  $$
  Note $\dim Y' = \dim Y$ since $p\rvert_{Y'} : Y' \to Y$ is finite surjective.
  Since $(\mathcal{L}^{\dim Y} \cdot Y) > 0$ by (2), we therefore see that
  $( (p^*\mathcal{L})^{\dim Y'} \cdot Y') > 0$.
\end{proof}

\bibliographystyle{unsrt}
\bibliography{references}

\end{document}

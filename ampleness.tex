\input{preamble}


\newcommand{\todo}[1]{\footnote{\textbf{TODO.} #1}}
\begin{document}
\title{Ampleness Critera}
\maketitle

\section{Ampleness}
\begin{definition}[{\cite[\href{http://stacks.math.columbia.edu/tag/0D31}{Tag 0D31}]{stacks-project}}]\label{definition-relatively-ample}
Let $S$ be a scheme.
Let $f : X \to Y$ be a morphism of algebraic spaces over $S$.
Let $\mathcal{L}$ be an invertible $\mathcal{O}_X$-module.
We say $\mathcal{L}$ is {\it relatively ample}, or {\it $f$-relatively ample},
or {\it ample on $X/Y$}, or {\it $f$-ample} if $f : X \to Y$
is representable and for every morphism $Z \to Y$
where $Z$ is a scheme, the pullback $\mathcal{L}_T$ of $\mathcal{L}$
to $X_Z = Z \times_Y X$ is ample on $X_Z/Z$.
\end{definition}

\begin{lemma}[Serre vanishing for algebraic spaces, {\cite[\href{http://stacks.math.columbia.edu/tag/0D38}{Tag 0D38}]{stacks-project}}]\label{lemma-vanshing-gives-ample}
Let $R$ be a Noetherian ring. Let $X$ be an algebraic space over $R$
such that the structure morphism $f : X \to \Spec(R)$ is proper.
Let $\mathcal{L}$ be an invertible $\mathcal{O}_X$-module.
The following are equivalent
\begin{enumerate}
\item $\mathcal{L}$ is ample on $X/R$
(Definition \ref{definition-relatively-ample}),
\item for every coherent $\mathcal{O}_X$-module $\mathcal{F}$
there exists an $n_0 \geq 0$ such that
$H^p(X, \mathcal{F} \otimes \mathcal{L}^{\otimes n}) = 0$
for all $n \geq n_0$ and $p > 0$.
\end{enumerate}
\end{lemma}

\section{Intersection theory}
\begin{lemma}[{\cite[\href{http://stacks.math.columbia.edu/tag/0DN4}{Tag 0DN4}]{stacks-project}}]\label{lemma-numerical-polynomial-from-euler}
Let $k$ be a field. Let $X$ be a proper algebraic space over $k$.
Let $\mathcal{F}$ be a coherent $\mathcal{O}_X$-module. Let
$\mathcal{L}_1, \ldots, \mathcal{L}_r$ be invertible $\mathcal{O}_X$-modules.
The map
$$
(n_1, \ldots, n_r) \longmapsto
\chi(X, \mathcal{F} \otimes
\mathcal{L}_1^{\otimes n_1} \otimes \ldots \otimes
\mathcal{L}_r^{\otimes n_r})
$$
is a numerical polynomial in $n_1, \ldots, n_r$ of total degree at
most the dimension of the scheme theoretic support of $\mathcal{F}$.
\end{lemma}
We can then define:
\begin{definition}[{cf.\ \cite[\href{http://stacks.math.columbia.edu/tag/0BEP}{Tag 0BEP}]{stacks-project}}]
Let $k$ be a field. Let $X$ be a proper algebraic space over $k$. Let
$i : Z \to X$ be a closed subscheme of dimension $d$. Let
$\mathcal{L}_1, \ldots, \mathcal{L}_d$ be invertible
$\mathcal{O}_X$-modules. We define the {\it intersection number}
$(\mathcal{L}_1 \cdots \mathcal{L}_d \cdot Z)$
as the coefficient of $n_1 \ldots n_d$ in the numerical polynomial
$$
\chi(X, i_*\mathcal{O}_Z \otimes \mathcal{L}_1^{\otimes n_1} \otimes
\ldots \otimes \mathcal{L}_d^{\otimes n_d}) =
\chi(Z, \mathcal{L}_1^{\otimes n_1} \otimes
\ldots \otimes \mathcal{L}_d^{\otimes n_d}|_Z)
$$
In the special
case that $\mathcal{L}_1 = \ldots = \mathcal{L}_d = \mathcal{L}$
we write $(\mathcal{L}^d \cdot Z)$.
\end{definition}

\section{Introduction}
In this section, we work out some ampleness critera for schemes over
algebraically closed fields.

\begin{theorem}[Nakai--Moishezon Criterion]
\label{theorem-nakai-moishezon}
Let $X$ be a proper scheme over an algebraically closed field $k$.
Let $\mathcal{L}$ be an invertible $\mathcal{O}_X$-module.
Then the following are equivalent:
\begin{enumerate}
  \item $\mathcal{L}$ is ample.
  \item For every closed integral subscheme $Y$ of $X$,
    $(\mathcal{L}^{\dim(Y)} \cdot Y) > 0$.
\end{enumerate}
\end{theorem}

\begin{proof}
Assume (1) holds.
Then there is some $n > 0$ such that $\mathcal{L}^{\otimes n}$ is very ample.
Replacing $\mathcal{L}$ by $\mathcal{L}^{\otimes n}$, we may assume
$\mathcal{L}$ is very ample.
Let $f : X \hookrightarrow \mathbf{P}^n$ be a closed immersion such that
$f^*\mathcal{O}_{\mathbf{P}^n}(1) = \mathcal{L}$.
If $Y$ is any integral subscheme of $X$,
$(\mathcal{L}^{\dim(Y)} \cdot Y) = \deg(Y) > 0$.\todo{Find reference for this or something.}

Assume (2) holds.
Reduce to the integral case.\todo{Figure out how to package this reduction.}
We proceed by induction on $\mathrm{dim}(X)$.
When $\mathrm{dim}(X) = 1$, (2) says that $\deg(\mathcal{L}) > 0$ and hence
by ...\todo{Find reference} $L$ is ample.

Now assume $\dim(X) > 1$ and the theorem is true for proper schemes of lower
dimension.
Let $\mathcal{K}_X$ be the sheaf of meromorphic functions on $X$.
Then $\mathcal{L}$ is a subsheaf of $\mathcal{K}_X$.\todo{Explain. Maybe need to fix
an identification with $\mathcal{O}_X(D)$ to show that the ideals have positive
codimension.}
Set $\mathcal{I}_1 = \mathcal{O}_X \cap \mathcal{L}$
and $\mathcal{I}_2 = \mathcal{O}_X \cap \mathcal{L}^{-1}$ with intersections
taken inside $\mathcal{K}_X$.
Then $\mathcal{I}_1$ and $\mathcal{I}_2$ are ideal sheaves on $X$.
Let $Y_j$ be the closed subscheme of $X$ defined by $\mathcal{I}_j$, $j = 1,2$.
Note that the $Y_j$ have positive codimension in $X$.\todo{Justify.}
Then, for all $m \geq $, we the following commutative diagram with exact rows:
$$
\xymatrix{
  0 \ar[r]
    & \mathcal{I}_1 \otimes \mathcal{L}^{\otimes m} \ar[r] \ar@{=}[d]
    & \mathcal{L}^{\otimes m} \ar[r]
    & \mathcal{L}^{\otimes m}\rvert_{Y_1} \ar[r]
    & 0 \\
  0 \ar[r]
    & \mathcal{I}_2 \otimes \mathcal{L}^{\otimes (m-1)} \ar[r]
    & \mathcal{L}^{\otimes (m-1)} \ar[r]
    & \mathcal{L}^{\otimes (m-1)}\rvert_{Y_2} \ar[r]
    & 0.
}
$$
By induction, $\mathcal{L}\rvert_{Y_j}$ is ample on $Y_j$ for $j = 1,2$.
Hence by Serre vanishing\todo{Find reference}, there is some $m_0 > 0$
such that for all $m \geq m_0$,
$H^i(Y_j,\mathcal{L}^{\otimes m}\rvert_{Y_j}) = 0$ for all $i > 0$.
Thus, taking the long exact sequence in cohomology of the sequences above,
for $i \geq 2$,
$$
  h^i(X,\mathcal{L}^{\otimes m})
    = h^i(X,\mathcal{I}_1 \otimes \mathcal{L}^{\otimes m})
    = h^i(X,\mathcal{I}_2 \otimes \mathcal{L}^{\otimes (m - 1)})
    = h^i(X,\mathcal{L}^{\otimes (m-1)})
$$
for all $m > m_0$.
Hence, for all $m > m_0$,
$$
 \sum\nolimits_{i = 2}^{\dim(X)} h^i(X,\mathcal{L}^{\otimes m})
$$
is a constant, say, $N$.
Since
$$
  \chi(X,\mathcal{L}^{\otimes m})
    = h^0(X,\mathcal{L}^{\otimes m}) - h^1(X,\mathcal{L}^{\otimes m})
      + N \to \infty
$$
as $m \to \infty$ just by the hypothesis that the leading coefficient
$(\mathcal{L}^{\dim X} \cdot X)$ of the Euler characteristic
$\chi(X,\mathcal{L}^{\otimes m})$ is positive\todo{Reference a lemma. OK},
we have that
$h^0(X,\mathcal{L}^{\otimes m}) - h^1(X,\mathcal{L}^{\otimes m}) \to \infty$
as $m \to \infty$.
Hence $h^0(X,\mathcal{L}^{\otimes m}) \to \infty$ as $m \to \infty$.
Thus we may replace $\mathcal{L}$ by a multiple $\mathcal{L}^{\otimes m}$
so that $\mathcal{L} = \mathcal{O}_X(D)$ for some effective Cartier divisor
$D$.

The short exact sequence\todo{Put pushforward along inclusion \(i \colon D
    \hookrightarrow X\)? OK}
$$
  0 \to \mathcal{O}_X((m-1)D) \to \mathcal{O}_X(mD) \to \mathcal{O}_D(mD) \to 0
$$
together with Serre vanishing (Lemma \ref{lemma-vanshing-gives-ample}) applied to $\mathcal{O}_D(mD)$,
we have a surjection
$$
\rho_m \colon H^1(X,\mathcal{O}_X((m -1)D)) \to H^1(X,\mathcal{O}_X(mD))
$$
for all $m$ sufficiently large.
Hence the $h^1(X,\mathcal{O}_X(mD))$ form a nonincreasing sequence and thus
stabilizes for $m$ sufficiently large, in which time the map
$$
H^0(X,\mathcal{O}_X(mD)) \to H^0(D,\mathcal{O}_D(mD))
$$
is surjective.
By induction, $\mathcal{O}_D(D)$ is ample on $D$ and hence $\mathcal{O}_D(mD)$
is generated by global sections for $m$ large.
By Nakayama's Lemma, $\mathcal{O}_X(mD)$ is generated by global sections\todo{How do.}.
Hence there is a morphism $f : X \to \mathbf{P}^n_k$ such that
$f^*\mathcal{O}_{\mathbf{P}^n}(1) = \mathcal{O}_X(mD)$.
As $X$ is proper, $f$ is proper.\todo{Maybe reference the required property.}
Since $\mathcal{O}_X(mD)$ restricted to any fibre is trivial, (2) implies that
every fibre must be zero dimensional\todo{Be more careful with this; split out to lemma?}.
Hence $f$ is quasi-finite, and therefore finite by ...\todo{Reference.}.
By ...\todo{Pullback by finite of ample is ample.}, $\mathcal{O}_X(mD)$ is ample
and so $\mathcal{O}_X(D)$ is ample.
\end{proof}

\bibliographystyle{unsrt}
\bibliography{references}

\end{document}

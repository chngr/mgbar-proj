\IfFileExists{stacks-project.cls}{%
\documentclass{stacks-project}
}{%
\documentclass{amsart}
}

% The following AMS packages are automatically loaded with
% the amsart documentclass:
%\usepackage{amsmath}
%\usepackage{amssymb}
%\usepackage{amsthm}

% For dealing with references we use the comment environment
\usepackage{verbatim}
\newenvironment{reference}{\comment}{\endcomment}
%\newenvironment{reference}{}{}
\newenvironment{slogan}{\comment}{\endcomment}
\newenvironment{history}{\comment}{\endcomment}

% For commutative diagrams you can use
% \usepackage{amscd}
\usepackage[all]{xy}

% We use 2cell for 2-commutative diagrams.
\xyoption{2cell}
\UseAllTwocells

% To put source file link in headers.
% Change "template.tex" to "this_filename.tex"
% \usepackage{fancyhdr}
% \pagestyle{fancy}
% \lhead{}
% \chead{}
% \rhead{Source file: \url{template.tex}}
% \lfoot{}
% \cfoot{\thepage}
% \rfoot{}
% \renewcommand{\headrulewidth}{0pt}
% \renewcommand{\footrulewidth}{0pt}
% \renewcommand{\headheight}{12pt}

\usepackage{multicol}

% For cross-file-references
\usepackage{xr-hyper}

% Package for hypertext links:
\usepackage{hyperref}

% For any local file, say "hello.tex" you want to link to please
% use \externaldocument[hello-]{hello}
\externaldocument[introduction-]{introduction}
\externaldocument[conventions-]{conventions}
\externaldocument[sets-]{sets}
\externaldocument[categories-]{categories}
\externaldocument[topology-]{topology}
\externaldocument[sheaves-]{sheaves}
\externaldocument[sites-]{sites}
\externaldocument[stacks-]{stacks}
\externaldocument[fields-]{fields}
\externaldocument[algebra-]{algebra}
\externaldocument[brauer-]{brauer}
\externaldocument[homology-]{homology}
\externaldocument[derived-]{derived}
\externaldocument[simplicial-]{simplicial}
\externaldocument[more-algebra-]{more-algebra}
\externaldocument[smoothing-]{smoothing}
\externaldocument[modules-]{modules}
\externaldocument[sites-modules-]{sites-modules}
\externaldocument[injectives-]{injectives}
\externaldocument[cohomology-]{cohomology}
\externaldocument[sites-cohomology-]{sites-cohomology}
\externaldocument[dga-]{dga}
\externaldocument[dpa-]{dpa}
\externaldocument[hypercovering-]{hypercovering}
\externaldocument[schemes-]{schemes}
\externaldocument[constructions-]{constructions}
\externaldocument[properties-]{properties}
\externaldocument[morphisms-]{morphisms}
\externaldocument[coherent-]{coherent}
\externaldocument[divisors-]{divisors}
\externaldocument[limits-]{limits}
\externaldocument[varieties-]{varieties}
\externaldocument[topologies-]{topologies}
\externaldocument[descent-]{descent}
\externaldocument[perfect-]{perfect}
\externaldocument[more-morphisms-]{more-morphisms}
\externaldocument[flat-]{flat}
\externaldocument[groupoids-]{groupoids}
\externaldocument[more-groupoids-]{more-groupoids}
\externaldocument[etale-]{etale}
\externaldocument[chow-]{chow}
\externaldocument[intersection-]{intersection}
\externaldocument[pic-]{pic}
\externaldocument[adequate-]{adequate}
\externaldocument[dualizing-]{dualizing}
\externaldocument[duality-]{duality}
\externaldocument[discriminant-]{discriminant}
\externaldocument[local-cohomology-]{local-cohomology}
\externaldocument[curves-]{curves}
\externaldocument[resolve-]{resolve}
\externaldocument[models-]{models}
\externaldocument[pione-]{pione}
\externaldocument[etale-cohomology-]{etale-cohomology}
\externaldocument[proetale-]{proetale}
\externaldocument[crystalline-]{crystalline}
\externaldocument[spaces-]{spaces}
\externaldocument[spaces-properties-]{spaces-properties}
\externaldocument[spaces-morphisms-]{spaces-morphisms}
\externaldocument[decent-spaces-]{decent-spaces}
\externaldocument[spaces-cohomology-]{spaces-cohomology}
\externaldocument[spaces-limits-]{spaces-limits}
\externaldocument[spaces-divisors-]{spaces-divisors}
\externaldocument[spaces-over-fields-]{spaces-over-fields}
\externaldocument[spaces-topologies-]{spaces-topologies}
\externaldocument[spaces-descent-]{spaces-descent}
\externaldocument[spaces-perfect-]{spaces-perfect}
\externaldocument[spaces-more-morphisms-]{spaces-more-morphisms}
\externaldocument[spaces-flat-]{spaces-flat}
\externaldocument[spaces-groupoids-]{spaces-groupoids}
\externaldocument[spaces-more-groupoids-]{spaces-more-groupoids}
\externaldocument[bootstrap-]{bootstrap}
\externaldocument[spaces-pushouts-]{spaces-pushouts}
\externaldocument[groupoids-quotients-]{groupoids-quotients}
\externaldocument[spaces-more-cohomology-]{spaces-more-cohomology}
\externaldocument[spaces-simplicial-]{spaces-simplicial}
\externaldocument[spaces-duality-]{spaces-duality}
\externaldocument[formal-spaces-]{formal-spaces}
\externaldocument[restricted-]{restricted}
\externaldocument[spaces-resolve-]{spaces-resolve}
\externaldocument[formal-defos-]{formal-defos}
\externaldocument[defos-]{defos}
\externaldocument[cotangent-]{cotangent}
\externaldocument[examples-defos-]{examples-defos}
\externaldocument[algebraic-]{algebraic}
\externaldocument[examples-stacks-]{examples-stacks}
\externaldocument[stacks-sheaves-]{stacks-sheaves}
\externaldocument[criteria-]{criteria}
\externaldocument[artin-]{artin}
\externaldocument[quot-]{quot}
\externaldocument[stacks-properties-]{stacks-properties}
\externaldocument[stacks-morphisms-]{stacks-morphisms}
\externaldocument[stacks-limits-]{stacks-limits}
\externaldocument[stacks-cohomology-]{stacks-cohomology}
\externaldocument[stacks-perfect-]{stacks-perfect}
\externaldocument[stacks-introduction-]{stacks-introduction}
\externaldocument[stacks-more-morphisms-]{stacks-more-morphisms}
\externaldocument[stacks-geometry-]{stacks-geometry}
\externaldocument[moduli-]{moduli}
\externaldocument[moduli-curves-]{moduli-curves}
\externaldocument[examples-]{examples}
\externaldocument[exercises-]{exercises}
\externaldocument[guide-]{guide}
\externaldocument[desirables-]{desirables}
\externaldocument[coding-]{coding}
\externaldocument[obsolete-]{obsolete}
\externaldocument[fdl-]{fdl}
\externaldocument[index-]{index}

% Theorem environments.
%
\theoremstyle{plain}
\newtheorem{theorem}[subsection]{Theorem}
\newtheorem{proposition}[subsection]{Proposition}
\newtheorem{lemma}[subsection]{Lemma}

\theoremstyle{definition}
\newtheorem{definition}[subsection]{Definition}
\newtheorem{example}[subsection]{Example}
\newtheorem{exercise}[subsection]{Exercise}
\newtheorem{situation}[subsection]{Situation}

\theoremstyle{remark}
\newtheorem{remark}[subsection]{Remark}
\newtheorem{remarks}[subsection]{Remarks}

\numberwithin{equation}{subsection}

% Macros
%
\def\lim{\mathop{\rm lim}\nolimits}
\def\colim{\mathop{\rm colim}\nolimits}
\def\Spec{\mathop{\rm Spec}}
\def\Hom{\mathop{\rm Hom}\nolimits}
\def\Ext{\mathop{\rm Ext}\nolimits}
\def\SheafHom{\mathop{\mathcal{H}\!{\it om}}\nolimits}
\def\SheafExt{\mathop{\mathcal{E}\!{\it xt}}\nolimits}
\def\Sch{\textit{Sch}}
\def\Mor{\mathop{\rm Mor}\nolimits}
\def\Ob{\mathop{\rm Ob}\nolimits}
\def\Sh{\mathop{\textit{Sh}}\nolimits}
\def\NL{\mathop{N\!L}\nolimits}
\def\proetale{{pro\text{-}\acute{e}tale}}
\def\etale{{\acute{e}tale}}
\def\QCoh{\textit{QCoh}}
\def\Ker{\mathop{\rm Ker}}
\def\Im{\mathop{\rm Im}}
\def\Coker{\mathop{\rm Coker}}
\def\Coim{\mathop{\rm Coim}}

%
% Macros for moduli stacks/spaces
%
\def\QCohstack{\mathcal{QC}\!{\it oh}}
\def\Cohstack{\mathcal{C}\!{\it oh}}
\def\Spacesstack{\mathcal{S}\!{\it paces}}
\def\Quotfunctor{{\rm Quot}}
\def\Hilbfunctor{{\rm Hilb}}
\def\Curvesstack{\mathcal{C}\!{\it urves}}
\def\Polarizedstack{\mathcal{P}\!{\it olarized}}
\def\Complexesstack{\mathcal{C}\!{\it omplexes}}
% \Pic is the operator that assigns to X its picard group, usage \Pic(X)
% \Picardstack_{X/B} denotes the Picard stack of X over B
% \Picardfunctor_{X/B} denotes the Picard functor of X over B
\def\Pic{\mathop{\rm Pic}\nolimits}
\def\Picardstack{\mathcal{P}\!{\it ic}}
\def\Picardfunctor{{\rm Pic}}
\def\Deformationcategory{\mathcal{D}\!{\it ef}}

\newcommand{\citeSP}[1]{\cite[\href{http://stacks.math.columbia.edu/tag/#1}{Tag #1}]{stacks-project}}
\newcommand{\todo}[1]{\footnote{\textbf{TODO.} #1}}

\begin{document}
\title{Ampleness Criteria}
\maketitle

\section{Nakai--Moishezon}
In this section, we prove an ampleness criterion for invertible sheaves on
schemes over a field using intersection theory.
The next result can be seen as a converse to~\citeSP{0BEV}.

\begin{theorem}[Nakai--Moishezon Criterion]
\label{theorem-nakai-moishezon}
Let $X$ be a proper scheme over an algebraically closed field $k$.
Let $\mathcal{L}$ be an invertible $\mathcal{O}_X$-module.
Assume that for every closed integral subscheme $Y$ of $X$,
$(\mathcal{L}^{\dim(Y)} \cdot Y) > 0$.
Then $\mathcal{L}$ is ample.
\end{theorem}

\begin{proof}
Using~\citeSP{0B5V} and~\citeSP{09MS}, we reduce to the case $X$ is integral.
We proceed by induction on $\dim(X)$.
When $\dim(X) = 1$, our assumption says that $\deg(\mathcal{L}) > 0$ and hence
$\mathcal{L}$ is ample by~\citeSP{0B5X}.

Now suppose $\dim(X) > 1$ and that the theorem is true for all proper schemes
of lower dimension.
Since $X$ is integral, $\mathcal{L}$ has a regular meromorphic section
by~\citeSP{02OZ}.
Let $\mathcal{I}_1$ be the sheaf of denominators of $\mathcal{L}$ and set
$\mathcal{I}_2 = \mathcal{I}_1 \otimes \mathcal{L}$.
Let $Y_j$ be the closed subschemes defined by $\mathcal{I}_j$ with $j = 1,2$.
By~\citeSP{02P0}, $\dim(Y_j) < \dim(X)$.
Then, for all $m \geq 0$, we the following commutative diagram with exact rows:
$$
\xymatrix{
  0 \ar[r]
    & \mathcal{I}_1 \otimes \mathcal{L}^{\otimes m} \ar[r] \ar@{=}[d]
    & \mathcal{L}^{\otimes m} \ar[r]
    & \mathcal{L}^{\otimes m}\rvert_{Y_1} \ar[r]
    & 0 \\
  0 \ar[r]
    & \mathcal{I}_2 \otimes \mathcal{L}^{\otimes (m-1)} \ar[r]
    & \mathcal{L}^{\otimes (m-1)} \ar[r]
    & \mathcal{L}^{\otimes (m-1)}\rvert_{Y_2} \ar[r]
    & 0.
}
$$
By induction, $\mathcal{L}\rvert_{Y_j}$ is ample on $Y_j$ for $j = 1,2$.
Hence by~\citeSP{0B5U}, there is some $m_0 \geq 0$
such that for all $m \geq m_0$,
$H^i(Y_j,\mathcal{L}^{\otimes m}\rvert_{Y_j}) = 0$ for all $i > 0$.
Thus, taking the long exact sequence in cohomology of the sequences above,
for $i \geq 2$,
$$
  h^i(X,\mathcal{L}^{\otimes m})
    = h^i(X,\mathcal{I}_1 \otimes \mathcal{L}^{\otimes m})
    = h^i(X,\mathcal{I}_2 \otimes \mathcal{L}^{\otimes (m - 1)})
    = h^i(X,\mathcal{L}^{\otimes (m-1)})
$$
for all $m > m_0$.
Hence, for all $m > m_0$,
$$
N := \sum\nolimits_{i = 2}^{\dim(X)} (-1)^i\,h^i(X,\mathcal{L}^{\otimes m})
$$
is a constant.
Now since $\chi(X,\mathcal{L}^{\otimes m})$ has leading coefficient
$(\mathcal{L}^{\dim X} \cdot X)$, which is positive by assumption,
we see that
$$
  \chi(X,\mathcal{L}^{\otimes m})
    = h^0(X,\mathcal{L}^{\otimes m}) - h^1(X,\mathcal{L}^{\otimes m})
      + N \to \infty
$$
as $m \to \infty$. Thus,
$h^0(X,\mathcal{L}^{\otimes m}) - h^1(X,\mathcal{L}^{\otimes m}) \to \infty$
as $m \to \infty$; in particular,
$h^0(X,\mathcal{L}^{\otimes m}) \to \infty$ as $m \to \infty$.
By Lemma~\citeSP{01PS}, we may replace $\mathcal{L}$ by
$\mathcal{L}^{\otimes m}$ to assume $\mathcal{L} = \mathcal{O}_X(D)$ for some
effective Cartier divisor $D$.

We now claim that $\mathcal{L}^{\otimes m} = \mathcal{O}_X(mD)$ is generated by
its global sections for $m \gg 0$. Consider the short exact sequence
$$
0 \longrightarrow \mathcal{O}_X((m-1)D) \longrightarrow \mathcal{O}_X(mD)
\longrightarrow \mathcal{O}_D(mD) \longrightarrow 0.
$$
Since $\mathcal{O}_D(mD)$ is ample by the inductive hypothesis,
Serre vanishing~\citeSP{0B5U} implies
$H^1(D,\mathcal{O}_D(mD)) = 0$ for $m \gg 0$, hence the maps
$$
\rho_m : H^1(X,\mathcal{O}_X((m -1)D)) \longrightarrow H^1(X,\mathcal{O}_X(mD))
$$
arising from the long exact sequence on cohomology
are surjective for all $m \gg 0$. Since the vector spaces
$H^1(X,\mathcal{O}_X(mD))$ are all finite-dimensional, we have that for $m \gg
0$, the sequence
$$h^1(X,\mathcal{O}_X(mD)) \ge h^1(X,\mathcal{O}_X((m+1)D)) \ge \cdots$$
is a nonincreasing sequence of nonnegative integers, and thus stabilizes for $m
\gg 0$. The maps $\rho_m$ are therefore bijective for $m \gg 0$, hence
\begin{equation}\label{eq:restriction-surjects}
  H^0(X,\mathcal{O}_X(mD)) \longrightarrow H^0(D,\mathcal{O}_D(mD))
\end{equation}
is surjective for $m \gg 0$. We can now show $\mathcal{O}_X(mD)$ is generated by
global sections. If $x \in X \setminus D$, then a global section in
$H^0(X,\mathcal{O}_X(mD))$ defining $mD$ generates $(\mathcal{O}_X(mD))_x$,
since $\mathcal{O}_X(mD)$ restricted to $X \setminus D$ is trivial.
Otherwise, suppose $x \in D$. By inductive hypothesis, since $\mathcal{O}_D(D)$
is ample, the invertible sheaf $\mathcal{O}_D(mD)$ is generated by global
sections for all $m \gg 0$, i.e., the morphism
$$
H^0(D,\mathcal{O}_D(mD)) \otimes_k \mathcal{O}_D \longrightarrow
\mathcal{O}_D(mD)
$$
is surjective. Thus, in the commutative diagram
$$
\xymatrix{
  H^0(X,\mathcal{O}_X(mD)) \otimes_k \mathcal{O}_X/\mathfrak{m}_x
  \ar[r]\ar@{->>}[d] & \mathcal{O}_X(mD) \otimes_{\mathcal{O}_X}
  \mathcal{O}_X/\mathfrak{m}_x \ar[d]^\simeq\\
  H^0(D,\mathcal{O}_D(mD)) \otimes_k \mathcal{O}_D/\mathfrak{m}_x \ar@{->>}[r] &
  \mathcal{O}_D(mD) \otimes_{\mathcal{O}_D} \mathcal{O}_D/\mathfrak{m}_x
}
$$
the bottom arrow is surjective. By \eqref{eq:restriction-surjects}, the left
arrow is surjective. The commutativity of the diagram implies the top row is
surjective, hence
$$
H^0(X,\mathcal{O}_X(mD)) \otimes_k \mathcal{O}_{X,x} \longrightarrow
(\mathcal{O}_X(mD))_x
$$
is surjective by Nakayama's lemma, i.e., $\mathcal{O}_X(mD)$ is generated by
global sections.

We therefore see that $\mathcal{O}_X(mD)$ induces a morphism $f : X \to
\mathbf{P}^n_k$ such that $f^*\mathcal{O}_{\mathbf{P}^n}(1) =
\mathcal{O}_X(mD)$ for $m \gg 0$. As $X$ is proper, $f$ is proper.
We now claim that for all $y \in \mathbf{P}^n_k$, the fiber $X_y$ has finitely
many points. If not, we have a commutative diagram
$$
\xymatrix{
  X \ar[r]^f & \mathbf{P}^n_k\\
  X_y \ar@{^{(}->}[u] \ar[r] & \Spec(k(y)) \ar@{^{(}->}[u]\\
  C \ar@{^{(}->}[u]\ar[ur]_\pi
}
$$
where the top square is cartesian, and $C$ is a complete curve in $X_y$. By
commutativity of the diagram, we see that
$$
  \mathcal{L}\rvert_C =
    (f^*\mathcal{O}_{\mathbf{P}^n}(1))\rvert_C \simeq
    \pi^*\mathcal{O}_{\Spec(k(y))} =
    \mathcal{O}_C,
$$
which is a contradiction, since $(\mathcal{L}^{\dim X - 1} \cdot C) > 0$ by (2),
whereas $\mathcal{O}_C$ has constant Euler characteristic. Thus, $f$ is
quasi-finite; since it is proper by the fact that both $X$ and $\mathbf{P}^n_k$
are proper, we moreover have that $f$ is finite by~\citeSP{02OG}.
Since the pullback of an ample invertible sheaf via a quasi-finite morphism is
ample by~\citeSP{0892} and finite
morphisms are quasi-finite by definition,
$\mathcal{O}_X(mD)$ is ample for $m \gg 0$.
Hence $\mathcal{O}_X(D)$ is ample by~\citeSP{01PS}.
\end{proof}

\bibliographystyle{unsrt}
\bibliography{references}

\end{document}

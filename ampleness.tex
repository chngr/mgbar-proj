\IfFileExists{stacks-project.cls}{%
\documentclass{stacks-project}
}{%
\documentclass{amsart}
}

% The following AMS packages are automatically loaded with
% the amsart documentclass:
%\usepackage{amsmath}
%\usepackage{amssymb}
%\usepackage{amsthm}

% For dealing with references we use the comment environment
\usepackage{verbatim}
\newenvironment{reference}{\comment}{\endcomment}
%\newenvironment{reference}{}{}
\newenvironment{slogan}{\comment}{\endcomment}
\newenvironment{history}{\comment}{\endcomment}

% For commutative diagrams you can use
% \usepackage{amscd}
\usepackage[all]{xy}

% We use 2cell for 2-commutative diagrams.
\xyoption{2cell}
\UseAllTwocells

% To put source file link in headers.
% Change "template.tex" to "this_filename.tex"
% \usepackage{fancyhdr}
% \pagestyle{fancy}
% \lhead{}
% \chead{}
% \rhead{Source file: \url{template.tex}}
% \lfoot{}
% \cfoot{\thepage}
% \rfoot{}
% \renewcommand{\headrulewidth}{0pt}
% \renewcommand{\footrulewidth}{0pt}
% \renewcommand{\headheight}{12pt}

\usepackage{multicol}

% For cross-file-references
\usepackage{xr-hyper}

% Package for hypertext links:
\usepackage{hyperref}

% For any local file, say "hello.tex" you want to link to please
% use \externaldocument[hello-]{hello}
\externaldocument[introduction-]{introduction}
\externaldocument[conventions-]{conventions}
\externaldocument[sets-]{sets}
\externaldocument[categories-]{categories}
\externaldocument[topology-]{topology}
\externaldocument[sheaves-]{sheaves}
\externaldocument[sites-]{sites}
\externaldocument[stacks-]{stacks}
\externaldocument[fields-]{fields}
\externaldocument[algebra-]{algebra}
\externaldocument[brauer-]{brauer}
\externaldocument[homology-]{homology}
\externaldocument[derived-]{derived}
\externaldocument[simplicial-]{simplicial}
\externaldocument[more-algebra-]{more-algebra}
\externaldocument[smoothing-]{smoothing}
\externaldocument[modules-]{modules}
\externaldocument[sites-modules-]{sites-modules}
\externaldocument[injectives-]{injectives}
\externaldocument[cohomology-]{cohomology}
\externaldocument[sites-cohomology-]{sites-cohomology}
\externaldocument[dga-]{dga}
\externaldocument[dpa-]{dpa}
\externaldocument[hypercovering-]{hypercovering}
\externaldocument[schemes-]{schemes}
\externaldocument[constructions-]{constructions}
\externaldocument[properties-]{properties}
\externaldocument[morphisms-]{morphisms}
\externaldocument[coherent-]{coherent}
\externaldocument[divisors-]{divisors}
\externaldocument[limits-]{limits}
\externaldocument[varieties-]{varieties}
\externaldocument[topologies-]{topologies}
\externaldocument[descent-]{descent}
\externaldocument[perfect-]{perfect}
\externaldocument[more-morphisms-]{more-morphisms}
\externaldocument[flat-]{flat}
\externaldocument[groupoids-]{groupoids}
\externaldocument[more-groupoids-]{more-groupoids}
\externaldocument[etale-]{etale}
\externaldocument[chow-]{chow}
\externaldocument[intersection-]{intersection}
\externaldocument[pic-]{pic}
\externaldocument[adequate-]{adequate}
\externaldocument[dualizing-]{dualizing}
\externaldocument[duality-]{duality}
\externaldocument[discriminant-]{discriminant}
\externaldocument[local-cohomology-]{local-cohomology}
\externaldocument[curves-]{curves}
\externaldocument[resolve-]{resolve}
\externaldocument[models-]{models}
\externaldocument[pione-]{pione}
\externaldocument[etale-cohomology-]{etale-cohomology}
\externaldocument[proetale-]{proetale}
\externaldocument[crystalline-]{crystalline}
\externaldocument[spaces-]{spaces}
\externaldocument[spaces-properties-]{spaces-properties}
\externaldocument[spaces-morphisms-]{spaces-morphisms}
\externaldocument[decent-spaces-]{decent-spaces}
\externaldocument[spaces-cohomology-]{spaces-cohomology}
\externaldocument[spaces-limits-]{spaces-limits}
\externaldocument[spaces-divisors-]{spaces-divisors}
\externaldocument[spaces-over-fields-]{spaces-over-fields}
\externaldocument[spaces-topologies-]{spaces-topologies}
\externaldocument[spaces-descent-]{spaces-descent}
\externaldocument[spaces-perfect-]{spaces-perfect}
\externaldocument[spaces-more-morphisms-]{spaces-more-morphisms}
\externaldocument[spaces-flat-]{spaces-flat}
\externaldocument[spaces-groupoids-]{spaces-groupoids}
\externaldocument[spaces-more-groupoids-]{spaces-more-groupoids}
\externaldocument[bootstrap-]{bootstrap}
\externaldocument[spaces-pushouts-]{spaces-pushouts}
\externaldocument[groupoids-quotients-]{groupoids-quotients}
\externaldocument[spaces-more-cohomology-]{spaces-more-cohomology}
\externaldocument[spaces-simplicial-]{spaces-simplicial}
\externaldocument[spaces-duality-]{spaces-duality}
\externaldocument[formal-spaces-]{formal-spaces}
\externaldocument[restricted-]{restricted}
\externaldocument[spaces-resolve-]{spaces-resolve}
\externaldocument[formal-defos-]{formal-defos}
\externaldocument[defos-]{defos}
\externaldocument[cotangent-]{cotangent}
\externaldocument[examples-defos-]{examples-defos}
\externaldocument[algebraic-]{algebraic}
\externaldocument[examples-stacks-]{examples-stacks}
\externaldocument[stacks-sheaves-]{stacks-sheaves}
\externaldocument[criteria-]{criteria}
\externaldocument[artin-]{artin}
\externaldocument[quot-]{quot}
\externaldocument[stacks-properties-]{stacks-properties}
\externaldocument[stacks-morphisms-]{stacks-morphisms}
\externaldocument[stacks-limits-]{stacks-limits}
\externaldocument[stacks-cohomology-]{stacks-cohomology}
\externaldocument[stacks-perfect-]{stacks-perfect}
\externaldocument[stacks-introduction-]{stacks-introduction}
\externaldocument[stacks-more-morphisms-]{stacks-more-morphisms}
\externaldocument[stacks-geometry-]{stacks-geometry}
\externaldocument[moduli-]{moduli}
\externaldocument[moduli-curves-]{moduli-curves}
\externaldocument[examples-]{examples}
\externaldocument[exercises-]{exercises}
\externaldocument[guide-]{guide}
\externaldocument[desirables-]{desirables}
\externaldocument[coding-]{coding}
\externaldocument[obsolete-]{obsolete}
\externaldocument[fdl-]{fdl}
\externaldocument[index-]{index}

% Theorem environments.
%
\theoremstyle{plain}
\newtheorem{theorem}[subsection]{Theorem}
\newtheorem{proposition}[subsection]{Proposition}
\newtheorem{lemma}[subsection]{Lemma}

\theoremstyle{definition}
\newtheorem{definition}[subsection]{Definition}
\newtheorem{example}[subsection]{Example}
\newtheorem{exercise}[subsection]{Exercise}
\newtheorem{situation}[subsection]{Situation}

\theoremstyle{remark}
\newtheorem{remark}[subsection]{Remark}
\newtheorem{remarks}[subsection]{Remarks}

\numberwithin{equation}{subsection}

% Macros
%
\def\lim{\mathop{\rm lim}\nolimits}
\def\colim{\mathop{\rm colim}\nolimits}
\def\Spec{\mathop{\rm Spec}}
\def\Hom{\mathop{\rm Hom}\nolimits}
\def\Ext{\mathop{\rm Ext}\nolimits}
\def\SheafHom{\mathop{\mathcal{H}\!{\it om}}\nolimits}
\def\SheafExt{\mathop{\mathcal{E}\!{\it xt}}\nolimits}
\def\Sch{\textit{Sch}}
\def\Mor{\mathop{\rm Mor}\nolimits}
\def\Ob{\mathop{\rm Ob}\nolimits}
\def\Sh{\mathop{\textit{Sh}}\nolimits}
\def\NL{\mathop{N\!L}\nolimits}
\def\proetale{{pro\text{-}\acute{e}tale}}
\def\etale{{\acute{e}tale}}
\def\QCoh{\textit{QCoh}}
\def\Ker{\mathop{\rm Ker}}
\def\Im{\mathop{\rm Im}}
\def\Coker{\mathop{\rm Coker}}
\def\Coim{\mathop{\rm Coim}}

%
% Macros for moduli stacks/spaces
%
\def\QCohstack{\mathcal{QC}\!{\it oh}}
\def\Cohstack{\mathcal{C}\!{\it oh}}
\def\Spacesstack{\mathcal{S}\!{\it paces}}
\def\Quotfunctor{{\rm Quot}}
\def\Hilbfunctor{{\rm Hilb}}
\def\Curvesstack{\mathcal{C}\!{\it urves}}
\def\Polarizedstack{\mathcal{P}\!{\it olarized}}
\def\Complexesstack{\mathcal{C}\!{\it omplexes}}
% \Pic is the operator that assigns to X its picard group, usage \Pic(X)
% \Picardstack_{X/B} denotes the Picard stack of X over B
% \Picardfunctor_{X/B} denotes the Picard functor of X over B
\def\Pic{\mathop{\rm Pic}\nolimits}
\def\Picardstack{\mathcal{P}\!{\it ic}}
\def\Picardfunctor{{\rm Pic}}
\def\Deformationcategory{\mathcal{D}\!{\it ef}}



\newcommand{\todo}[1]{\footnote{\textbf{TODO.} #1}}
\begin{document}
\title{Ampleness Critera}
\maketitle

\section{Ampleness}
\begin{definition}[{\cite[\href{http://stacks.math.columbia.edu/tag/0D31}{Tag 0D31}]{stacks-project}}]
Let $S$ be a scheme.
Let $f : X \to Y$ be a morphism of algebraic spaces over $S$.
Let $\mathcal{L}$ be an invertible $\mathcal{O}_X$-module.
We say $\mathcal{L}$ is {\it relatively ample}, or {\it $f$-relatively ample},
or {\it ample on $X/Y$}, or {\it $f$-ample} if $f : X \to Y$
is representable and for every morphism $Z \to Y$
where $Z$ is a scheme, the pullback $\mathcal{L}_T$ of $\mathcal{L}$
to $X_Z = Z \times_Y X$ is ample on $X_Z/Z$.
\end{definition}

\begin{lemma}[Serre vanishing for algebraic spaces, {\cite[\href{http://stacks.math.columbia.edu/tag/0D38}{Tag 0D38}]{stacks-project}}]
Let $R$ be a Noetherian ring. Let $X$ be an algebraic space over $R$
such that the structure morphism $f : X \to \Spec(R)$ is proper.
Let $\mathcal{L}$ be an invertible $\mathcal{O}_X$-module.
The following are equivalent
\begin{enumerate}
\item $\mathcal{L}$ is ample on $X/R$
(Definition \ref{definition-relatively-ample}),
\item for every coherent $\mathcal{O}_X$-module $\mathcal{F}$
there exists an $n_0 \geq 0$ such that
$H^p(X, \mathcal{F} \otimes \mathcal{L}^{\otimes n}) = 0$
for all $n \geq n_0$ and $p > 0$.
\end{enumerate}
\end{lemma}

\section{Intersection theory}
\begin{lemma}[{\cite[\href{http://stacks.math.columbia.edu/tag/0DN4}{Tag 0DN4}]{stacks-project}}]\label{tag:0DN4}
Let $k$ be a field. Let $X$ be a proper algebraic space over $k$.
Let $\mathcal{F}$ be a coherent $\mathcal{O}_X$-module. Let
$\mathcal{L}_1, \ldots, \mathcal{L}_r$ be invertible $\mathcal{O}_X$-modules.
The map
$$
(n_1, \ldots, n_r) \longmapsto
\chi(X, \mathcal{F} \otimes
\mathcal{L}_1^{\otimes n_1} \otimes \ldots \otimes
\mathcal{L}_r^{\otimes n_r})
$$
is a numerical polynomial in $n_1, \ldots, n_r$ of total degree at
most the dimension of the scheme theoretic support of $\mathcal{F}$.
\end{lemma}
We can then define:
\begin{definition}[{cf.\ \cite[\href{http://stacks.math.columbia.edu/tag/0BEP}{Tag 0BEP}]{stacks-project}}]\label{tag:0BEP}
Let $k$ be a field. Let $X$ be a proper algebraic space over $k$. Let
$i : Z \to X$ be a closed subscheme of dimension $d$. Let
$\mathcal{L}_1, \ldots, \mathcal{L}_d$ be invertible
$\mathcal{O}_X$-modules. We define the {\it intersection number}
$(\mathcal{L}_1 \cdots \mathcal{L}_d \cdot Z)$
as the coefficient of $n_1 \ldots n_d$ in the numerical polynomial
$$
\chi(X, i_*\mathcal{O}_Z \otimes \mathcal{L}_1^{\otimes n_1} \otimes
\ldots \otimes \mathcal{L}_d^{\otimes n_d}) =
\chi(Z, \mathcal{L}_1^{\otimes n_1} \otimes
\ldots \otimes \mathcal{L}_d^{\otimes n_d}|_Z)
$$
In the special
case that $\mathcal{L}_1 = \ldots = \mathcal{L}_d = \mathcal{L}$
we write $(\mathcal{L}^d \cdot Z)$.
\end{definition}

\section{Introduction}
In this section, we work out some ampleness critera for schemes over
algebraically closed fields.

\begin{theorem}[Nakai--Moishezon Criterion]
\label{theorem-nakai-moishezon}
Let $X$ be a proper scheme over an algebraically closed field $k$.
Let $\mathcal{L}$ be an invertible $\mathcal{O}_X$-module.
Then the following are equivalent:
\begin{enumerate}
  \item $\mathcal{L}$ is ample.
  \item For every closed integral subscheme $Y$ of $X$,
    $(\mathcal{L}^{\dim(Y)} \cdot Y) > 0$.
\end{enumerate}
\end{theorem}

\begin{proof}
Assume $\mathcal{L}$ is ample.
Then there is some $n > 0$ such that $\mathcal{L}^\otimes{n}$ is very ample.
Replacing $\mathcal{L}$ by $\mathcal{L}^\otimes{n}$, we may assume
$\mathcal{L}$ is very ample.
Let $f : X \hookrightarrow \mathbf{P}^n$ be a closed immersion such that
$f^*\mathcal{O}_{\mathbf{P}^n}(1) = \mathcal{L}$.
If $Y$ is any integral subscheme of $X$,
$(\mathcal{L}^{\dim(Y)} \cdot Y) = \deg(Y) > 0$.\todo{Find reference for this or something.}

Assume (2) holds.
Reduce to the integral case.\todo{Figure out how to package this reduction.}
We proceed by induction on $\mathrm{dim}(X)$.
When $\mathrm{dim}(X) = 1$, (2) says that $\deg(\mathcal{L}) > 0$ and hence
by ...\todo{Find reference} $L$ is ample.

Now assume $\dim(X) > 1$ and the theorem is true for proper schemes of lower
dimension.
Let $\mathcal{K}_X$ be the sheaf of meromorphic functions on $X$.
Then $\mathcal{L}$ is a subsheaf of $\mathcal{K}_X$.\todo{Explain. Maybe need to fix
an identification with $\mathcal{O}_X(D)$ to show that the ideals have positive
codimension.}
Set $\mathcal{I}_1 = \mathcal{O}_X \cap \mathcal{L}$
and $\mathcal{I}_2 = \mathcal{O}_X \cap \mathcal{L}^{-1}$ with intersections
taken inside $\mathcal{K}_X$.
Then $\mathcal{I}_1$ and $\mathcal{I}_2$ are ideal sheaves on $X$.
Let $Y_j$ be the closed subscheme of $X$ defined by $\mathcal{I}_j$, $j = 1,2$.
Note that the $Y_j$ have positive codimension in $X$.\todo{Justify.}
Then, for all $m \geq $, we the following commutative diagram with exact rows:
$$
\xymatrix{
  0 \ar[r]
    & \mathcal{I}_1 \otimes \mathcal{L}^{\otimes m} \ar[r] \ar@{=}[d]
    & \mathcal{L}^{\otimes m} \ar[r]
    & \mathcal{L}^{\otimes m}\rvert_{Y_1} \ar[r]
    & 0 \\
  0 \ar[r]
    & \mathcal{I}_2 \otimes \mathcal{L}^{\otimes (m-1)} \ar[r]
    & \mathcal{L}^{\otimes (m-1)} \ar[r]
    & \mathcal{L}^{\otimes (m-1)}\rvert_{Y_2} \ar[r]
    & 0.
}
$$
By induction, $\mathcal{L}\rvert_{Y_j}$ is ample on $Y_j$ for $j = 1,2$.
Hence by Serre vanishing\todo{Find reference}, there is some $m_0 > 0$
such that for all $m \geq m_0$,
$H^i(Y_j,\mathcal{L}^{\otimes m}\rvert_{Y_j}) = 0$ for all $i > 0$.
Thus, taking the long exact sequence in cohomology of the sequences above,
for $i \geq 2$,
$$
  h^i(X,\mathcal{L}^{\otimes m})
    = h^i(X,\mathcal{I}_1 \otimes \mathcal{L}^{\otimes m})
    = h^i(X,\mathcal{I}_2 \otimes \mathcal{L}^{\otimes (m - 1)})
    = h^i(X,\mathcal{L}^{\otimes (m-1)})
$$
for all $m > m_0$.
Hence, for all $m > m_0$,
$$
 \sum\nolimits_{i = 2}^{\dim(X)} h^i(X,\mathcal{L}^{\otimes m})
$$
is a constant, say, $N$.
Since
$$
  \chi(X,\mathcal{L}^{\otimes m})
    = h^0(X,\mathcal{L}^{\otimes m}) - h^1(X,\mathcal{L}^{\otimes m})
      + N \to \infty
$$
as $m \to \infty$ just by the hypothesis that the leading coefficient
$(\mathcal{L}^{\dim X} \cdot X)$ of the Euler characteristic
$\chi(X,\mathcal{L}^{\otimes m})$ is positive\todo{Reference a lemma. OK},
we have that
$h^0(X,\mathcal{L}^{\otimes m}) - h^1(X,\mathcal{L}^{\otimes m}) \to \infty$
as $m \to \infty$.
Hence $h^0(X,\mathcal{L}^{\otimes m}) \to \infty$ as $m \to \infty$.
Thus we may replace $\mathcal{L}$ by a multiple $\mathcal{L}^{\otimes m}$
so that $\mathcal{L} = \mathcal{O}_X(D)$ for some effective Cartier divisor
$D$.

The short exact sequence\todo{Put pushforward along inclusion \(i \colon D
    \hookrightarrow X\)? OK}
$$
  0 \to \mathcal{O}_X((m-1)D) \to \mathcal{O}_X(mD) \to \mathcal{O}_D(mD) \to 0
$$
together with Serre vanishing\todo{Reference.} applied to $\mathcal{O}_D(mD)$,
we have a surjection
$$
\rho_m \colon H^1(X,\mathcal{O}_X((m -1)D)) \to H^1(X,\mathcal{O}_X(mD))
$$
for all $m$ sufficiently large.
Hence the $h^1(X,\mathcal{O}_X(mD))$ form a nonincreasing sequence and thus
stabilizes for $m$ sufficiently large, in which time the map
$$
H^0(X,\mathcal{O}_X(mD)) \to H^0(D,\mathcal{O}_D(mD))
$$
is surjective.
By induction, $\mathcal{O}_D(D)$ is ample on $D$ and hence $\mathcal{O}_D(mD)$
is generated by global sections for $m$ large.
By Nakayama's Lemma, $\mathcal{O}_X(mD)$ is generated by global sections\todo{How do.}.
Hence there is a morphism $f : X \to \mathbf{P}^n_k$ such that
$f^*\mathcal{O}_{\mathbf{P}^n}(1) = \mathcal{O}_X(mD)$.
As $X$ is proper, $f$ is proper.\todo{Maybe reference the required property.}
Since $\mathcal{O}_X(mD)$ restricted to any fibre is trivial, (2) implies that
every fibre must be zero dimensional\todo{Be more careful with this; split out to lemma?}.
Hence $f$ is quasi-finite, and therefore finite by ...\todo{Reference.}.
By ...\todo{Pullback by finite of ample is ample.}, $\mathcal{O}_X(mD)$ is ample
and so $\mathcal{O}_X(D)$ is ample.
\end{proof}

\bibliographystyle{unsrt}
\bibliography{references}

\end{document}

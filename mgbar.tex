\IfFileExists{stacks-project.cls}{%
\documentclass{stacks-project}
}{%
\documentclass{amsart}
}

% The following AMS packages are automatically loaded with
% the amsart documentclass:
%\usepackage{amsmath}
%\usepackage{amssymb}
%\usepackage{amsthm}

% For dealing with references we use the comment environment
\usepackage{verbatim}
\newenvironment{reference}{\comment}{\endcomment}
%\newenvironment{reference}{}{}
\newenvironment{slogan}{\comment}{\endcomment}
\newenvironment{history}{\comment}{\endcomment}

% For commutative diagrams you can use
% \usepackage{amscd}
\usepackage[all]{xy}

% We use 2cell for 2-commutative diagrams.
\xyoption{2cell}
\UseAllTwocells

% To put source file link in headers.
% Change "template.tex" to "this_filename.tex"
% \usepackage{fancyhdr}
% \pagestyle{fancy}
% \lhead{}
% \chead{}
% \rhead{Source file: \url{template.tex}}
% \lfoot{}
% \cfoot{\thepage}
% \rfoot{}
% \renewcommand{\headrulewidth}{0pt}
% \renewcommand{\footrulewidth}{0pt}
% \renewcommand{\headheight}{12pt}

\usepackage{multicol}

% For cross-file-references
\usepackage{xr-hyper}

% Package for hypertext links:
\usepackage{hyperref}

% For any local file, say "hello.tex" you want to link to please
% use \externaldocument[hello-]{hello}
\externaldocument[introduction-]{introduction}
\externaldocument[conventions-]{conventions}
\externaldocument[sets-]{sets}
\externaldocument[categories-]{categories}
\externaldocument[topology-]{topology}
\externaldocument[sheaves-]{sheaves}
\externaldocument[sites-]{sites}
\externaldocument[stacks-]{stacks}
\externaldocument[fields-]{fields}
\externaldocument[algebra-]{algebra}
\externaldocument[brauer-]{brauer}
\externaldocument[homology-]{homology}
\externaldocument[derived-]{derived}
\externaldocument[simplicial-]{simplicial}
\externaldocument[more-algebra-]{more-algebra}
\externaldocument[smoothing-]{smoothing}
\externaldocument[modules-]{modules}
\externaldocument[sites-modules-]{sites-modules}
\externaldocument[injectives-]{injectives}
\externaldocument[cohomology-]{cohomology}
\externaldocument[sites-cohomology-]{sites-cohomology}
\externaldocument[dga-]{dga}
\externaldocument[dpa-]{dpa}
\externaldocument[hypercovering-]{hypercovering}
\externaldocument[schemes-]{schemes}
\externaldocument[constructions-]{constructions}
\externaldocument[properties-]{properties}
\externaldocument[morphisms-]{morphisms}
\externaldocument[coherent-]{coherent}
\externaldocument[divisors-]{divisors}
\externaldocument[limits-]{limits}
\externaldocument[varieties-]{varieties}
\externaldocument[topologies-]{topologies}
\externaldocument[descent-]{descent}
\externaldocument[perfect-]{perfect}
\externaldocument[more-morphisms-]{more-morphisms}
\externaldocument[flat-]{flat}
\externaldocument[groupoids-]{groupoids}
\externaldocument[more-groupoids-]{more-groupoids}
\externaldocument[etale-]{etale}
\externaldocument[chow-]{chow}
\externaldocument[intersection-]{intersection}
\externaldocument[pic-]{pic}
\externaldocument[adequate-]{adequate}
\externaldocument[dualizing-]{dualizing}
\externaldocument[duality-]{duality}
\externaldocument[discriminant-]{discriminant}
\externaldocument[local-cohomology-]{local-cohomology}
\externaldocument[curves-]{curves}
\externaldocument[resolve-]{resolve}
\externaldocument[models-]{models}
\externaldocument[pione-]{pione}
\externaldocument[etale-cohomology-]{etale-cohomology}
\externaldocument[proetale-]{proetale}
\externaldocument[crystalline-]{crystalline}
\externaldocument[spaces-]{spaces}
\externaldocument[spaces-properties-]{spaces-properties}
\externaldocument[spaces-morphisms-]{spaces-morphisms}
\externaldocument[decent-spaces-]{decent-spaces}
\externaldocument[spaces-cohomology-]{spaces-cohomology}
\externaldocument[spaces-limits-]{spaces-limits}
\externaldocument[spaces-divisors-]{spaces-divisors}
\externaldocument[spaces-over-fields-]{spaces-over-fields}
\externaldocument[spaces-topologies-]{spaces-topologies}
\externaldocument[spaces-descent-]{spaces-descent}
\externaldocument[spaces-perfect-]{spaces-perfect}
\externaldocument[spaces-more-morphisms-]{spaces-more-morphisms}
\externaldocument[spaces-flat-]{spaces-flat}
\externaldocument[spaces-groupoids-]{spaces-groupoids}
\externaldocument[spaces-more-groupoids-]{spaces-more-groupoids}
\externaldocument[bootstrap-]{bootstrap}
\externaldocument[spaces-pushouts-]{spaces-pushouts}
\externaldocument[groupoids-quotients-]{groupoids-quotients}
\externaldocument[spaces-more-cohomology-]{spaces-more-cohomology}
\externaldocument[spaces-simplicial-]{spaces-simplicial}
\externaldocument[spaces-duality-]{spaces-duality}
\externaldocument[formal-spaces-]{formal-spaces}
\externaldocument[restricted-]{restricted}
\externaldocument[spaces-resolve-]{spaces-resolve}
\externaldocument[formal-defos-]{formal-defos}
\externaldocument[defos-]{defos}
\externaldocument[cotangent-]{cotangent}
\externaldocument[examples-defos-]{examples-defos}
\externaldocument[algebraic-]{algebraic}
\externaldocument[examples-stacks-]{examples-stacks}
\externaldocument[stacks-sheaves-]{stacks-sheaves}
\externaldocument[criteria-]{criteria}
\externaldocument[artin-]{artin}
\externaldocument[quot-]{quot}
\externaldocument[stacks-properties-]{stacks-properties}
\externaldocument[stacks-morphisms-]{stacks-morphisms}
\externaldocument[stacks-limits-]{stacks-limits}
\externaldocument[stacks-cohomology-]{stacks-cohomology}
\externaldocument[stacks-perfect-]{stacks-perfect}
\externaldocument[stacks-introduction-]{stacks-introduction}
\externaldocument[stacks-more-morphisms-]{stacks-more-morphisms}
\externaldocument[stacks-geometry-]{stacks-geometry}
\externaldocument[moduli-]{moduli}
\externaldocument[moduli-curves-]{moduli-curves}
\externaldocument[examples-]{examples}
\externaldocument[exercises-]{exercises}
\externaldocument[guide-]{guide}
\externaldocument[desirables-]{desirables}
\externaldocument[coding-]{coding}
\externaldocument[obsolete-]{obsolete}
\externaldocument[fdl-]{fdl}
\externaldocument[index-]{index}

% Theorem environments.
%
\theoremstyle{plain}
\newtheorem{theorem}[subsection]{Theorem}
\newtheorem{proposition}[subsection]{Proposition}
\newtheorem{lemma}[subsection]{Lemma}

\theoremstyle{definition}
\newtheorem{definition}[subsection]{Definition}
\newtheorem{example}[subsection]{Example}
\newtheorem{exercise}[subsection]{Exercise}
\newtheorem{situation}[subsection]{Situation}

\theoremstyle{remark}
\newtheorem{remark}[subsection]{Remark}
\newtheorem{remarks}[subsection]{Remarks}

\numberwithin{equation}{subsection}

% Macros
%
\def\lim{\mathop{\rm lim}\nolimits}
\def\colim{\mathop{\rm colim}\nolimits}
\def\Spec{\mathop{\rm Spec}}
\def\Hom{\mathop{\rm Hom}\nolimits}
\def\Ext{\mathop{\rm Ext}\nolimits}
\def\SheafHom{\mathop{\mathcal{H}\!{\it om}}\nolimits}
\def\SheafExt{\mathop{\mathcal{E}\!{\it xt}}\nolimits}
\def\Sch{\textit{Sch}}
\def\Mor{\mathop{\rm Mor}\nolimits}
\def\Ob{\mathop{\rm Ob}\nolimits}
\def\Sh{\mathop{\textit{Sh}}\nolimits}
\def\NL{\mathop{N\!L}\nolimits}
\def\proetale{{pro\text{-}\acute{e}tale}}
\def\etale{{\acute{e}tale}}
\def\QCoh{\textit{QCoh}}
\def\Ker{\mathop{\rm Ker}}
\def\Im{\mathop{\rm Im}}
\def\Coker{\mathop{\rm Coker}}
\def\Coim{\mathop{\rm Coim}}

%
% Macros for moduli stacks/spaces
%
\def\QCohstack{\mathcal{QC}\!{\it oh}}
\def\Cohstack{\mathcal{C}\!{\it oh}}
\def\Spacesstack{\mathcal{S}\!{\it paces}}
\def\Quotfunctor{{\rm Quot}}
\def\Hilbfunctor{{\rm Hilb}}
\def\Curvesstack{\mathcal{C}\!{\it urves}}
\def\Polarizedstack{\mathcal{P}\!{\it olarized}}
\def\Complexesstack{\mathcal{C}\!{\it omplexes}}
% \Pic is the operator that assigns to X its picard group, usage \Pic(X)
% \Picardstack_{X/B} denotes the Picard stack of X over B
% \Picardfunctor_{X/B} denotes the Picard functor of X over B
\def\Pic{\mathop{\rm Pic}\nolimits}
\def\Picardstack{\mathcal{P}\!{\it ic}}
\def\Picardfunctor{{\rm Pic}}
\def\Deformationcategory{\mathcal{D}\!{\it ef}}

\newcommand{\citeSP}[1]{\cite[\href{http://stacks.math.columbia.edu/tag/#1}{Tag #1}]{stacks-project}}
\newcommand{\todo}[1]{\footnote{\textbf{TODO.} #1}}


\date{\today}
\begin{document}
\title{Projectivity of moduli of curves}
\maketitle
\tableofcontents

\section{Introduction}
\label{section-introduction}
...\todo{Write a legitimate introduction. What is this
  document about; what does it aim to do; how did it come about. Add some
  references.}

\section{Positivity of invertible sheaves}
\label{section-positivity}
For material on positivity, we largely follow~\cite{lazI,lazII}.

\begin{definition}
\label{definition-nef-invertible-sheaf}
Let $X$ be a proper scheme over a field $k$.
An invertible $\mathcal{O}_X$-module is said to be \emph{numerically effective}
or \emph{nef} if $(\mathcal{L} \cdot C) \geq 0$ for every irreducible closed
subscheme $C \subset X$ of dimension $1$.
\end{definition}

\begin{lemma}
\label{lemma-nef-pullback-nef}
\begin{reference}
\cite[Example 1.4.4(i)]{lazI}
\end{reference}
Let $f : Y \to X$ be a morphism of proper schemes over $k$.
If $\mathcal{L}$ is a nef invertible sheaf on $X$,
then $f^*\mathcal{L}$ is a nef invertible sheaf on $Y$.
\end{lemma}

\begin{proof}
...\todo{Prove this!}
\end{proof}

\begin{lemma}
\label{lemma-nef-proper-surjective-nef}
\begin{reference}
\cite[Example 1.4.4(ii)]{lazI}
\end{reference}
Let $f : Y \to X$ be a surjective proper morphism of schemes over $k$.
Assume $X$ is proper over $k$.
Let $\mathcal{L}$ be an invertible sheaf on $X$.
Then $\mathcal{L}$ is nef on $X$ if and only if $f^*\mathcal{L}$ is nef on $Y$.
\end{lemma}

\begin{proof}
...\todo{Prove this!}
\end{proof}

\begin{lemma}
\label{lemma-degree-of-nef-on-curve}
Let $X$ be a proper scheme of dimension $1$ over a field $k$.
Let $\mathcal{L}$ be an invertible $\mathcal{O}_X$-module.
If $\mathcal{L}$ is nef, then $\deg(\mathcal{L}) \geq 0$.
\end{lemma}

\begin{proof}
In the case $X$ is irreducible, then the conclusion follows from~\citeSP{0BEY}
and the definitions.

In general, let $C_1,\ldots,C_t$ be the irreducible components of $X$, viewed
as subschemes of $X$ with the reduced induced subscheme structure.
By~\citeSP{0AYW},
$$
\deg(\mathcal{L}) = \sum\nolimits_i m_i\deg(\mathcal{L}|_{C_i})
$$
for some positive integers $m_i$.
The irreducible case gives $\deg(\mathcal{L}|_{C_i}) \geq 0$, and thus
$\deg(\mathcal{L}) \geq 0$.
\end{proof}

\begin{definition}
Let $k$ be a field.
Let $X$ be a proper scheme over $k$.
An invertible sheaf $\mathcal{L}$ on $X$ is said to be \emph{big}
if...\todo{Fill in the definition and show a few characterizations.}.
\end{definition}

\section{Tensor sheaves}
\label{section-tensor-sheaves}
Linear algebra constructions give many constructions to combine locally free
sheaves.
These constructions can be unified by viewing each as coming from a
representation of a product of general linear groups, and then the composite
sheaf can be described by applying the representation to the transition
functions of the constituent sheaves.

\begin{definition}
\label{definition-tensor-sheaf}
Let $k$ be a field.
Let $X$ be a scheme over $k$.
Let $\mathcal{E}_1,\ldots,\mathcal{E}_s$ be locally free sheaves of finite
ranks $r_1,\ldots,r_s$ on $X$.
Let $\rho : \prod_t \mathrm{GL}_{r_t} \to \mathrm{GL}_n$ be a homomorphism of
group schemes over $k$.
Then $\rho(\mathcal{E}_1,\ldots,\mathcal{E}_s) = \mathcal{E}$ is the
locally free sheaf of finite rank $n$ on $X$ constructed as follows:
Let $X = \bigcup_i U_i$ be an open cover over which each of
$\mathcal{E}_1,\ldots,\mathcal{E}_s$ is trivial and suppose $\mathcal{E}_t$ is
given by \v{C}ech cocycles $\{(U_j,\varphi_{t,ij})\}$.
Then $\mathcal{E}$ is the locally free sheaf given by the \v{C}ech cocycle
$\{(U_i,\rho(\varphi_{1,ij},\ldots,\varphi_{s,ij}))\}$.
\end{definition}

Following the terminology of~\cite[p.76]{hartshorne_ample_vb}, we may refer to
coherent locally free sheaves of the form
$\rho(\mathcal{E}_1,\ldots,\mathcal{E}_s)$ as \emph{tensor sheaves}.
We generally refer to this procedure as applying \emph{tensor operations}
to locally free sheaves.

\begin{remark}
\label{remark-tensor-sheaf-other-groups}
The construction in Definition~\ref{definition-tensor-sheaf} can be described
more generally for any homomorphism $\rho : G \to H$ of group schemes
and any $G$-torsor $\mathcal{E}$.
\end{remark}

\begin{remark}
\label{remark-tensor-sheaf-via-H1}
The construction in Definition~\ref{definition-tensor-sheaf} and its
generalization in Remark~\ref{remark-tensor-sheaf-other-groups} can be described
more abstractly.
Let $X$ be a scheme.
Let $\rho : G \to H$ be a homomorphism of group schemes
and write $\rho^* : H^1(X,G) \to H^1(X,H)$ be the induced map on first
cohomology sets.
Let $\mathcal{E}$ be a $G$-torsor over $X$.
By~\citeSP{02FQ}, $\mathcal{E}$ corresponds to a class $[\mathcal{E}] \in H^1(X,G)$.
Invoking, once again, the correspondence between torsors and classes in $H^1$,
$\rho^*([\mathcal{E}]) \in H^1(X,H)$ gives rise to a $H$-torsor.
This is $\rho(E)$.
\end{remark}

\begin{lemma}
\label{lemma-pullback-tensor-sheaves}
Let $k$ be a field.
Let $f: X' \to X$ be a morphism of schemes over $k$..
Let $\mathcal{E}_1,\ldots,\mathcal{E}_s$ be locally free sheaves of finite
ranks $r_1,\ldots,r_s$ on $X$.
Let $\rho : \prod_t \mathrm{GL}_{r_t} \to \mathrm{GL}_n$ be a homomorphism of
group schemes over $k$.
Then $\rho(f^*\mathcal{E}_1,\ldots,f^*\mathcal{E}_s)$ and
$f^*\rho(\mathcal{E}_1,\ldots,\mathcal{E}_s)$ are canonically isomorphic as
locally free $\mathcal{O}_{X'}$-modules.
\end{lemma}

\begin{proof}
This is immediate from the description given in
Remark~\ref{remark-tensor-sheaf-via-H1}.
\end{proof}

Certain tensor operations on sheaves preserve nonnegativity properties of
locally free sheaves.
We single one class of such tensor operations in the next definition.

\begin{definition}
\label{definition-nonneg-rep}
Let $k$ be a field.
Let $V_1,\ldots,V_s,V$ be finite dimensional vector spaces over $k$.
A homomorphism $\rho : \prod_t \mathrm{GL}(V_t) \to \mathrm{GL}(V)$ of group
schemes over $k$ is called \textit{nonnegative} if it extends to a homomorphism
$\tilde{\rho} : \prod_t \mathrm{End}(V_t) \to \mathrm{End}(V)$ of monoids
over $k$.
\end{definition}

Roughly, a nonnegative representation is one where there are ``not too many
$\det$s in the denominator.''
The choice of terminology in Definition~\ref{definition-nonneg-rep} comes about
as a generalization of \emph{positive representations} as defined
in~\cite[p.76]{hartshorne_ample_vb}.
A positive representation, as defined in Hartshorne's paper, is a nonnegative
representation.

\begin{lemma}
\label{lemma-sums-nonneg-rep}
Let $k$ be a field.
Let $V_1,\ldots,V_s$ be finite dimensional vector spaces over $k$.
Then the natural homomorphism
$\rho : \prod_t \mathrm{GL}(V_t) \to \mathrm{GL}(V_1 \oplus \cdots \oplus V_s)$
is nonnegative.
\end{lemma}

\begin{proof}
Omitted.
\end{proof}

\begin{lemma}
\label{lemma-tensor-nonneg-rep}
Let $k$ be a field.
Let $V_1,\ldots,V_s$ be finite dimensional vector spaces over $k$.
Then the natural homomorphism
$\rho : \prod_t \mathrm{GL}(V_t) \to \mathrm{GL}(V_1 \otimes_k \cdots \otimes_k V_s)$
is nonnegative.
\end{lemma}

\begin{proof}
Omitted.
\end{proof}

\begin{lemma}
\label{lemma-subreps-nonneg-rep}
Let $k$ be a field.
Let $V_1,\ldots,V_s,V$ be finite dimensional vector spaces over $k$.
Let $\rho : \prod_t \mathrm{GL}(V_t) \to \mathrm{GL}(V)$ be a homomorphism of
group schemes over $k$
Assume $\rho$ is a nonnegative representation.
Suppose $V' \subset V$ is an invariant subspace, defining a subrepresentation
$\rho' : \prod_t \mathrm{GL}(V_t) \to \mathrm{GL}(V')$.
Then $\rho'$ is a nonnegative representation.
\end{lemma}

\begin{proof}
Let $\rho:G\to \mathrm{GL}(W)$ be a nef representation and let
$W'\subset W$ be a subrepresentation.
We may easily reduce to the case $G=\mathrm{GL}(V)$.
Then, $\rho$ extends to a morphism of monoids
$\overline{\rho}:\mathrm{End}(V)\to \mathrm{End}(W)$.
The representation $\rho$ also gives a morphism of schemes
$\phi:\mathrm{GL}(V)\times W\to W$, where $W$ has been given the obvious scheme
structure.
Similarly, $\overline{\rho}$ gives rise to an extension
$\overline{\phi}:\mathrm{End}(V)\times W\to W$.

The fact that $W'\subset W$ is a subrepresentation is equivalent to the fact
that the image of $\phi$ when restricted to the closed subscheme
$\mathrm{GL}(V)\times W'$ is $W'$.
It is sufficient to prove the same statement for the closed subscheme
$\mathrm{End}(V)\times W'$ of $\mathrm{End}(V)\times W$.
However, by assumption the dense open subscheme $\mathrm{GL}(V)\times W'$ of
$\mathrm{End}(V)\times W'$ maps to $W'$, so $\mathrm{End}(V)\times W'$ must as
well, as desired. One should check that the resulting map on $W'$ is linear;
details omitted.\todo{Edit this to match notation.}
\end{proof}

\begin{lemma}
\label{lemma-quotient-nonneg-rep}
Let $k$ be a field.
Let $V_1,\ldots,V_s,V$ be finite dimensional vector spaces over $k$.
Let $\rho : \prod_t \mathrm{GL}(V_t) \to \mathrm{GL}(V)$ be a homomorphism of
group schemes over $k$
Assume $\rho$ is a nonnegative representation.
Suppose $V \to V''$ is a quotient of $V$ such that $\rho$ descends to a
representation $\rho'' : \prod_t \mathrm{GL}(V_t) \to \mathrm{GL}(V'')$.
Then $\rho''$ is a nonnegative representation.
\end{lemma}

\begin{proof}
Omitted.\todo{Maybe comment that the proof is basically the same as the subrep case.}
\end{proof}

\begin{lemma}
\label{lemma-star-rho-star-nonneg-rep}
Let $k$ be a field.
Let $V_1,\ldots,V_s,V$ be finite dimensional vector spaces over $k$.
Let $\rho : \prod_t \mathrm{GL}(V_t) \to \mathrm{GL}(V)$ be a homomorphism
of group schemes over $k$.
Let $\sigma : \prod_t \mathrm{GL}(V_t^\vee) \to \prod_t \mathrm{GL}(V_t)$ be
the antihomomorphism of monoids taking an automorphism of $V_t^\vee$ to its dual.
Let $\tau : \mathrm{GL}(V) \to \mathrm{GL}(V^\vee)$ denote the antihomomorphism
of monoids taking an automorphism of $V$ to its dual.
If $\rho$ is a nonnegative representation, then
$$
  \tau \circ \rho \circ \sigma : \prod \mathrm{GL}(V_t^\vee) \to \mathrm{GL}(V)
$$
is a nonnegative representation.
\end{lemma}

\begin{proof}
...\todo{Draw the required diagram.}
\end{proof}

\begin{lemma}
\label{lemma-wedge-nonneg-rep}
Let $k$ be a field.
Let $V$ be finite dimensional vector spaces over $k$.
Then, for every $r \geq 0$, the natural homomorphism
$\rho : \mathrm{GL}(V) \to \mathrm{GL}(\bigwedge^r V)$
is nonnegative.
\end{lemma}

\begin{proof}
This follows from
Lemma~\ref{lemma-sums-nonneg-rep},
Lemma~\ref{lemma-tensor-nonneg-rep}, and
Lemma~\ref{lemma-quotient-nonneg-rep}.
\end{proof}

\begin{lemma}
\label{lemma-sym-nonneg-rep}
Let $k$ be a field.
Let $V$ be finite dimensional vector spaces over $k$.
Then, for every $d \geq 0$, the natural homomorphism
$\rho : \mathrm{GL}(V) \to \mathrm{GL}(\mathrm{Sym}^d(V))$
is nonnegative.
\end{lemma}

\begin{proof}
This follows from
Lemma~\ref{lemma-sums-nonneg-rep},
Lemma~\ref{lemma-tensor-nonneg-rep}, and
Lemma~\ref{lemma-quotient-nonneg-rep}.
\end{proof}

\begin{lemma}
\label{lemma-nonneg-tensor-morphisms}
Let $k$ be a field.
Let $X$ be a scheme over $k$.
Let
$\mathcal{E}_1,\ldots,\mathcal{E}_s,\mathcal{E}'_1,\ldots,\mathcal{E}'_s$
be locally free sheaves of finite rank on $X$.
Assume that for each $t = 1,\ldots,s$,
$r_t = \mathrm{rank}(\mathcal{E}_t) = \mathrm{rank}(\mathcal{E}'_t)$.
For each $t = 1,\ldots,s$, let $f_t : \mathcal{E}_i \to \mathcal{E}'_i$ be a
morphism of $\mathcal{O}_X$-modules.
Let $\rho : \prod \mathrm{GL}_{r_t} \to \mathrm{GL}_n$ be a representation.
If $\rho$ is nef, then the $f_1,\ldots,f_s$ induce a morphism
$$
  \rho(f_1,\ldots,f_s) :
    \rho(\mathcal{E}_1,\ldots,\mathcal{E}_s) \to
    \rho(\mathcal{E}'_1,\dots,\mathcal{E}'_s)
$$
of $\mathcal{O}_X$-modules.
\end{lemma}

\begin{proof}
Choose an open cover $X = \bigcup_i U_i$ over which all the locally free
sheaves are trivial and fix a trivialization of each sheaf.
With respect to this open cover, each morphism $f_t$ can be locally described
by a $r_t \times r_t$ matrix.
Since $\rho$ is nonnegative, there exists an extension
$\tilde\rho : \prod \mathrm{Mat}_{r_t} \to \mathrm{Mat}_n$.
The matrices $\tilde\rho(f_1,\ldots,f_t)$ locally define the morphism of
$\mathcal{O}_X$-modules in the statement.
Details omitted.
\end{proof}

\begin{lemma}
\label{lemma-tensor-of-split-bundle}
Let $k$ be a field.
Let $X$ be a $k$-scheme.
For $i = 1, 2, \ldots, n$, let $V_i = \bigoplus_j \mathcal{L}_{ij}$ be a
locally free sheaf of rank $r_i$ that splits as a direct sum of line bundles
$\mathcal{L}_{ij}$.
Let $\rho : \prod_i \mathrm{GL}_{r_i} \to \mathrm{GL}_m$ be a nef
representation.
Then, $\rho(\mathcal{V}_1,\ldots,V_n)$ is a direct sum of line bundles of the
form $\bigoplus_{i,j}\bigotimes\mathcal{L}_{ij}^{a_{ij}}$, where $a_{ij}\ge0$ for
each $i,j$.\todo{Edit this to be consistent with this section.}
\end{lemma}

\begin{proof}
For each $i$, we have a diagonal embedding
$\psi_i : \prod_{j = 1}^{r_i}\mathrm{GL}_1 \to \mathrm{GL}_{r_i}$.
Taking the product over $i$ and post-composing with $\rho$ yields a map
$$
  \psi : \prod_{i = 1}^n \prod_{j = 1}^{r_i} \mathrm{GL}_1 \to \mathrm{GL}_m,
$$
which factors through a maximal torus
$\prod_{k = 1}^m \mathrm{GL}_1 \subset \mathrm{GL}_m$,
as in the following diagram:
$$
\xymatrix{
  \prod_{i = 1}^n \prod_{j = 1}^{r_i} \mathrm{GL}_1 \ar[r]^\psi \ar[rd] &
  \prod_{i = 1}^{n} \mathrm{GL}_{r_i} \ar[r]^\rho &
  \mathrm{GL}_m \\
    &
  \prod_{k = 1}^m \mathrm{GL}_1 \ar[ru] \ar[d]^{\mathrm{pr}_t} \\
    &
  \mathrm{GL}_1
}
$$

Now consider the projection on the $t$\textsuperscript{th} factor
$\mathrm{pr}_t : \prod_{k = 1}^m \mathrm{GL}_1\to \mathrm{GL}_1$.
The composition
$\prod_{i = 1}^n \prod_{j = 1}^{r_i} \mathrm{GL}_1\to \mathrm{GL}_1$ is a
character of the form $(z_{ij})_{i,j}\mapsto \prod z_{ij}^{a_{ij,t}}$.
On a trivializing cover of $X$, the $z_{ij}$ correspond to the transition
functions of $\mathcal{L}_{ij}$, hence $\prod z_{ij}^{a_{ij,t}}$ correspond to
$\otimes\mathcal{L}_{ij}^{a_{ij,t}}$.

Pulling back to $\prod_{k = 1}^m \mathrm{GL}_m$,
we see that $\rho(V_1,\ldots,V_n)$ splits as the direct sum of the line bundles
$\bigoplus_t \bigotimes_{i,j} \mathcal{L}_{ij}^{a_{ij,t}}$.
\end{proof}

\section{Nef locally free sheaves}
\label{section-nef-bundles}

In this section, we define and study basic properties of nef
bundles.

\begin{lemma}
\label{lemma-nef-bundle-equivalences}
Let $X$ be a proper scheme over a field $k$.
Let $\mathcal{E}$ be a coherent locally free sheaf on $X$.
Then the following are equivalent:
\begin{enumerate}
\item $\mathcal{O}_{\mathbf{P}(\mathcal{E})}(1)$ is nef on
  $\mathbf{P}(\mathcal{E})$; and
\item for every $k$-morphism $f : C \to X$ from a proper $k$-scheme $C$
of dimension $1$, and for every surjection $f^*\mathcal{E} \to \mathcal{L}$
of locally free $\mathcal{O}_C$-modules with $\mathrm{rank}(\mathcal{L}) = 1$,
we have $\deg_C(\mathcal{L}) \geq 0$.
\end{enumerate}
\end{lemma}

\begin{proof}
Let $\pi : \mathbf{P}(\mathcal{E}) \to X$ be the projection map from the
projective bundle $\mathbf{P}(\mathcal{E})$ down to $X$.
Assume (1) holds and let $f : C \to X$ be a morphism of $k$-schemes, with
1 dimensional $C$ which is smooth and proper over $k$, and let $\mathcal{L}$
be an invertible quotient of $f^*\mathcal{E}$ over $C$.
Applying $\mathrm{Sym}^*$ to the surjection $f^*\mathcal{E} \to \mathcal{L}$,
\citeSP{01O9} gives a morphism $r : C \to \mathbf{P}(\mathcal{E})$ such that
$\mathcal{L} \cong r^*\mathcal{O}_{\mathbf{P}(\mathcal{E})}(1)$.
Since both $C$ and $\mathbf{P}(\mathcal{E})$ are proper over $k$,~\citeSP{01W6}
shows $r$ is proper.
Thus by Lemma~\ref{lemma-nef-proper-pullback-nef}, $\mathcal{L}$ is nef.
Lemma~\ref{lemma-degree-of-nef-on-curve} now gives $\deg_C(\mathcal{L}) \geq 0$.

Assume (2) and let $g : C \hookrightarrow \mathbf{P}(\mathcal{E})$ be
any irreducible closed subscheme of dimension $1$.
Let $\mathcal{L} = g^*\mathcal{O}_{\mathbf{P}(\mathcal{E})}(1)$ and note
that it is nef by Lemma~\ref{lemma-nef-proper-pullback-nef}.
Set $f = \pi \circ g: C \to X$ to be the closed immersion of $C$ into the
projective bundle followed by the projection morphism.
By the universal property of relative Proj, the morphism $g$ gives rise
to a surjection $f^*\mathcal{E} \to \mathcal{L}$.
Now~\citeSP{0BEY} together with our hypothesis gives
$$
(\mathcal{O}_{\mathbf{P}(\mathcal{E})}(1) \cdot C) = \deg_C(\mathcal{L}) \geq 0.
$$
Therefore $\mathcal{O}_{\mathbf{P}(\mathcal{E})}(1)$ is nef.
\end{proof}

\begin{definition}
\label{definition-nef-bundle}
Let $k$ be a field.
Let $X$ be a proper scheme over $k$.
A coherent locally free sheaf $\mathcal{E}$ on $X$ is said to be \textit{nef}
if the equivalent conditions of Lemma~\ref{lemma-nef-bundle-equivalences} are
satisfied.
\end{definition}

\begin{lemma}
\label{lemma-nef-bundle-quotient-nef}
Let $k$ be a field.
Let $X$ be a proper scheme over $k$.
Let $\varphi : \mathcal{E} \to \mathcal{F}$ be a surjection of locally free
sheaves on $X$.
If $\mathcal{E}$ is nef, then $\mathcal{F}$ is nef.
\end{lemma}

\begin{proof}
Let $f : C \to X$ be a morphism of schemes over $k$ from a proper $k$-scheme
$C$ of dimension $1$ and let $f^*\mathcal{F} \to \mathcal{L}$ be an invertible
quotient.
Since $f^*$ is right exact, $f^*\varphi$ induces a surjection
$f^*\mathcal{E} \to \mathcal{L}$.
Since $\mathcal{E}$ is nef, $\deg_C(\mathcal{L}) \geq 0$.
This shows $\mathcal{F}$ is nef.
\end{proof}

\begin{lemma}
\label{lemma-nef-bundle-extension-nef}
Let $k$ be a field.
Let $X$ be a proper scheme over $k$.
Let
$$
0 \to \mathcal{E}' \to \mathcal{E} \to \mathcal{E}'' \to 0
$$
be a short exact sequence of coherent locally free sheaves on $X$.
If $\mathcal{E}'$ and $\mathcal{E}''$ are both nef, then $\mathcal{E}$ is nef.
\end{lemma}

\begin{proof}
...\todo{Prove this!}
\end{proof}

\begin{lemma}
\label{lemma-nef-pullback-nef-vb}
Let $k$ be a field.
Let $f : Y \to X$ be a morphism of proper schemes over $k$.
If $\mathcal{E}$ is a nef coherent locally free sheaf on $X$, then
$f^*\mathcal{E}$ is a nef coherent locally free sheaf on $Y$.
\end{lemma}

\begin{proof}
Let $g : C \to Y$ be a morphism of schemes over $k$ from a proper $k$-scheme
$C$ of dimension $1$ and let $g^*(f^*\mathcal{E}) \to \mathcal{L}$ be an
invertible quotient.
Since $g^*(f^*\mathcal{E}) = (f \circ g)^*\mathcal{E}$ and $\mathcal{E}$ is
nef, $\deg_C(\mathcal{L}) \geq 0$.
Thus $f^*\mathcal{E}$ is nef.
\end{proof}

\begin{lemma}\label{semipos_locus_open}
Let $X$ and $Y$ be schemes.
Let $f:Y\to X$ be a flat morphism of schemes.
Let $\mathcal{V}$ be a locally free sheaf of finite rank on $Y$.
Then, the set of points $x\in X$ such that the locally free sheaf
$\mathcal{V}_x$ is nef on the fiber $Y_x$ is open.
\end{lemma}

\begin{proof}
Let $\pi:\mathbb{P}_Y(V)\to Y$ be the projection map.
Given a point $x\in X$, by Lemma \ref{semipos=nef} the vector bundle
$\mathcal{V}_x$ is nef if and only if
$\mathcal{O}_{\mathbb{P}_Y(V)}(1)_x$ is nef.
The compositve morphism $f\circ \pi$ is flat, because $\pi$ is flat.
The claim now follows from the fact that nefness is an open condition in flat
families.\todo{Find a reference for this.}
\end{proof}

\begin{lemma}\label{global_generation_of_twist_on_curve}
Let $k$ be a field.
Let $C$ be a smooth, proper, and connected $k$-scheme of of genus $g$.
Let $\mathcal{E}$ be a vector bundle on $C$.
Let $\mathcal{L}$ be a line bundle on $C$ of degree at least $2g$.
Then, $\mathcal{E}\otimes\mathcal{L}$ is globally generated.
\end{lemma}

\begin{proof}
By~\citeSP{0B57}, it suffices to assume that $k$ is algebraically closed.
Let $x\in C$ be a closed point with ideal sheaf $\mathcal{O}(-x)$.
We have a short exact sequence
\begin{equation}
  0 \to
  \mathcal{E} \otimes \mathcal{L}(-x) \to
  \mathcal{E} \otimes \mathcal{L} \to
  \mathcal{E} \otimes \mathcal{L}\mid_x \to
  0
\end{equation}
It suffices to show that $H^1(C,\mathcal{E}\otimes\mathcal{L}(-x)) = 0$.
By Serre Duality\todo{Find a reference in the Stacks Project for this} and
since $\mathcal{L}(-x)$ is locally free,
$$
H^1(C,\mathcal{E}\otimes\mathcal{L}(-x)) \cong
  \Hom_{\mathcal{O}_C}(\mathcal{E} \otimes \mathcal{L}(-x),\omega_C)^\vee \cong
  \Hom_{\mathcal{O}_C}(\mathcal{E},\omega_C\otimes\mathcal{L}^{-1}(x))^\vee.
$$
Consider a morphism $f : \mathcal{E} \to \omega_C \otimes \mathcal{L}^{-1}(x)$
of $\mathcal{O}_C$-modules.
Since $\mathcal{E}$ is locally free, the image of $f$ is a torsion-free
subsheaf of $\omega_C \otimes \mathcal{L}^{-1}(x)$.\todo{Is this argument
right? \emph{The target of such a nonzero map is a locally free sheaf of
  negative degree, so the image subsheaf is torsion-free and hence locally free
  sheaf of negative degree.
  However, $\mathcal{E}$ has no negative line bundle quotients, so no such map
  can exist.}
I am not sure why the image is locally free?
All I can see right now is that the image is locally free away from finitely
many points.
}
\end{proof}


\begin{lemma}\label{lemma-no-neg-quot-curves-p}
Let $k$ be a field of characteristic $p > 0$.
Let $C$ be a smooth, proper, and connected $k$-scheme of dimension 1.
Let $\mathcal{V}_1,\ldots,\mathcal{V}_n$ be nef vector bundles of
ranks $r_1,\ldots,r_n$ on $C$.
Let $\rho : \prod \mathrm{GL}_{r_i}\to \mathrm{GL}_m$ be a nef
representation.
Then $\rho(\mathcal{V}_1,\ldots,\mathcal{V}_n)$ is a nef vector bundle.
\end{lemma}
\begin{proof}
Let $g$ be the genus of $C$, and fix a line bundle $\mathcal{L}$ of degree at
least $2g$.
By Lemma \ref{global_generation_of_twist_on_curve},
$\mathcal{V}_i \otimes\mathcal{L}$ is generated by global sections for each
$i$, so there exist maps
$$
  f_i : (\mathcal{L}^{-1})^{\oplus r_i} \to \mathcal{V}_i,
$$
which are surjective at a given closed point of $x \in C$, and hence
generically surjective.

Applying
Lemma~\ref{lemma-pullback-tensor-sheaves},
Lemma~\ref{lemma-tensor-of-split-bundle}, and
Lemma \ref{lemma-nonneg-tensor-morphisms}, we get a generically surjective map
$$
  \rho(f_1,\ldots,f_n) :
    \bigoplus_i \mathcal{L}^{-\otimes b_i} \to
    \rho(\mathcal{V}_1,\ldots,\mathcal{V}_n),
$$
where the source is
$\rho(\mathcal{L}^{-\otimes r_1},\ldots,\mathcal{L}^{-\otimes r_n})$.
Thus the $b_i$ are positive integers depending on the ranks of the $V_i$ but
not the $V_i$ themselves.
In particular, there exists a positive integer $N$, independent of the
$\mathcal{V}_i$, such that $\min\deg\mathcal{L}^{-b_i} = -N$.
Moreover, $\rho(f_1,\ldots,f_n)$ is surjective at $x$ by the construction in
Lemma \ref{lemma-nonneg-tensor-morphisms}, as the image of an $r$-tuple of invertible
matrices is invertible under $\overline{\rho}$.

Suppose $\mathcal{M}$ is an invertible quotient of $\rho(V_1,\ldots,V_n)$.
By pre-composing with $\rho(f_1,\ldots,f_n)$, we obtain a nonzero map
$$
  \bigoplus_i \mathcal{L}^{-\otimes b_i} \to \mathcal{M}.
$$
Thus, we conclude that
\begin{equation}\label{degree_of_quotient_not_too_small}
  \deg(\mathcal{M}) \ge -N.
\end{equation}

Finally, suppose that $\rho(V_1,\ldots,V_n)$ has a line bundle quotient
$\mathcal{N}$ of degree $d < 0$.
Let $F : C \to C$ be the absolute Frobenius map on $C$.
Pulling back the surjection $\rho(V_1,\ldots,V_n) \to \mathcal{N}$ by $F^t$,
we obtain a quotient line bundle of
$\rho((F^t)^* V_1,\ldots,(F^t)^* V_n)$ of degree $dp^t < -N$ for sufficiently
large $t$, by Lemma \ref{lemma-pullback-tensor-sheaves}.
However, as $(F^{t})^*V_i$ is nef for each $i$, this contradicts
\eqref{degree_of_quotient_not_too_small}, so hence $\mathcal{N}$ cannot exist.
\end{proof}

\begin{lemma}\label{lemma-no-neg-quot-curves}
Let $k$ be a field.
Let $C$ be a smooth, proper, and connected $k$-scheme of dimension 1.
Let $\mathcal{V}_1,\ldots,\mathcal{V}_n$ be nef vector bundles of
ranks $r_1,\ldots,r_n$ on $C$.
Let $\rho : \prod \mathrm{GL}_{r_i} \to \mathrm{GL}_m$ be a nef
representation.
Then $\rho(\mathcal{V}_1,\ldots,\mathcal{V}_n)$ is a nef vector bundle.
\end{lemma}
\begin{proof}
If the characteristic of $k$ is positive, this is Lemma
\ref{lemma-no-neg-quot-curves-p}.
If the characteristic of $k$ is zero, after possibly enlarging $k$, we may find
an integral and finite type $\mathbb{Z}$-algebra $A$ over which $C$,
$\mathcal{V}_1,\ldots,\mathcal{V}_n$, and $\rho$ are all defined, and such that
we have a smooth morphism of schemes $\pi : \mathcal{C} \to \Spec(A)$.
By Lemma \ref{semipos_locus_open}, we may replace $\Spec(A)$ with an open
subset so that each $\mathcal{V}_i$ is nef when restricted to any
fiber of $\pi$.
By Lemma~\ref{lemma-no-neg-quot-curves-p},
$\mathcal{W} = \rho(\mathcal{V}_1,\ldots,\mathcal{V}_n)$ is nef when
restricted to any fiber of $\pi$ above a point of $\Spec(A)$ with residue field
of positive characteristic.
This locus on $\Spec(A)$ is non-empty, so by another application of Lemma
\ref{semipos_locus_open}, so
$\mathcal{W} = \rho(\mathcal{V}_1,\ldots,\mathcal{V}_n)$ is nef when
restricted to the generic fiber, which is what was needed.
Some additional details may be necessary.\todo{Does this even work??}
\end{proof}


\begin{lemma}
\label{lemma-nonneg-rep-nef}
Let $k$ be a field.
Let $X$ be a proper scheme over $k$.
Let $\mathcal{E}_1,\ldots,\mathcal{E}_s$ be nef coherent locally free sheaves
on $X$ of ranks $r_1,\ldots,r_s$.
Let $\rho : \prod_t \mathrm{GL}_{r_t} \to \mathrm{GL}_n$ be a nonnegative
representation.
Let $\mathcal{E} = \rho(\mathcal{E}_1,\ldots,\mathcal{E}_n)$.
Then $\mathcal{E}$ is nef on $X$.
\end{lemma}

\begin{proof}
Immediate from Lemma \ref{lemma-no-neg-quot-curves}.
\end{proof}

%\begin{lemma}\label{semipos_of_symd_curve_charp}
%Let $C$ be a smooth curve over a field of characteristic $p>0$. Let $V$ be a nef vector bundle on $C$. Then, for any positive integer $d$, $Sym^dV$ is a nef vector bundle.
%\end{lemma}
%\begin{proof}
%Let $g$ be the genus of $C$, and let $\mathcal{L}$ be a line bundle on $C$ of degree at least $2g$.
%By Lemma \ref{global_generation_of_twist_on_curve}, $V\otimes\mathcal{L}^{-1}$ is globally generated, so there exists a generically surjective map
%\begin{equation}
%f:(\mathcal{L}^{-1})^{\oplus r_i}\to V
%\end{equation}
%coming from a collection of $r$ global sections generating the fiber at some closed point of $C$.
%Applying $Sym^d$, we thus get a generically surjective map
%\begin{equation}
%Sym^df:(\mathcal{L}^{-\otimes d})^{\binom{d+r}{r}}\to Sym^{d}V
%\end{equation}
%
%Suppose $\mathcal{M}$ is a line bundle quotient of $Sym^dV$. By pre-composing with $\rho(f_1,\ldots,f_n)$, we obtain a nonzero map
%\begin{equation}
%\bigoplus_{i}\mathcal{L}^{-\otimes d}\to\mathcal{M}.
%\end{equation}
%Thus, we conclude that
%\begin{equation}\label{degree_of_quotient_not_too_small}
%\deg\mathcal{M}\ge d\cdot\deg(\mathcal{L}).
%\end{equation}
%
%Finally, suppose that $\rho(V_1,\ldots,V_n)$ has a line bundle quotient $\mathcal{N}$ of degree $e<0$. Let $F:C\to C$ be the absolute Frobenius map on $C$. Then, by pulling back the surjection $Sym^dV\to\mathcal{N}$ by $F^t$, where $t$ is a positive integer, we obtain a quotient line bundle of $(F^{t})^*Sym^dV\cong Sym^d(F^{*}V)$ of degree $dp^{t}<-N$ for sufficiently large $t$, which contradicts (\ref{degree_of_quotient_not_too_small}) due to Lemma \todo{pullback is still positive}.
%
%\end{proof}
%
%
%\begin{lemma}\label{semipos_of_symd_curve_all_char}
%Let $C$ be a smooth curve over a field of arbitrary characteristic. Let $V$ be a nef vector bundle on $C$. Then, for any positive integer $d$, $Sym^dV$ is a nef vector bundle.
%\end{lemma}
%\begin{proof}
%If the characteristic of the ground field is positive, this is Lemma \ref{semipos_of_symd_curve_charp}. Suppose that the characteristic of the ground field is zero.
%\end{proof}

\begin{lemma}
\label{lemma-wedge-vb-nef}
Let $k$ be a field.
Let $X$ be a proper scheme over $k$.
Let $\mathcal{E}$ be a nef coherent locally free sheaf on $X$.
Then for every $r \geq 0$, $\bigwedge^r(\mathcal{E})$ is nef on $X$.
\end{lemma}

\begin{proof}
This is Lemma~\ref{lemma-nonneg-rep-nef} applied to the nef
representation $\bigwedge^r$, see Lemma~\ref{lemma-wedge-nonneg-rep}.
\end{proof}

\begin{lemma}\label{lemma-sym-vb-nef}
Let $k$ be a field.
Let $X$ be a proper scheme over $k$.
Let $\mathcal{E}$ a nef coherent locally free sheaf on $X$.
Then for every $d \geq 0$, $\mathrm{Sym}^d(\mathcal{E})$ is nef on $X$.
\end{lemma}

\begin{proof}
This is Lemma~\ref{lemma-nonneg-rep-nef} applied to the nef
representation $\mathrm{Sym}^d$, see Lemma~\ref{lemma-sym-nonneg-rep}.
\end{proof}


\begin{lemma}
\label{lemma-star-sym-star-nef}
Let $k$ be a field.
Let $X$ be a proper scheme over $k$.
Let $\mathcal{E}$ be a nef coherent locally free sheaf on $X$.
Then for every $d \geq 0$,
$\mathrm{Sym}^d(\mathcal{E}^\vee)^\vee$ is nef on $X$.
\end{lemma}

\begin{proof}
This is Lemma~\ref{lemma-nonneg-rep-nef} applied to the nef
representation $(-)^\vee \circ \mathrm{Sym}^d \circ (-)^\vee$,
see Lemma~\ref{lemma-sym-nonneg-rep} and
Lemma~\ref{lemma-star-rho-star-nonnegrep}.
\end{proof}

\begin{lemma}
Let $k$ be a field.
Let $X$ be a $k$-scheme.
Let $\mathcal{V}_1,\ldots,\mathcal{V}_n$ be nef vector bundles on $X$
of ranks $r_1,\ldots,r_n$, respectively.
Let $G = \prod_{i=1}^n \mathrm{GL}_{r_i}$.
Let $\rho : G \to \mathrm{GL}_m$ be a nef representation.
Let $\mathcal{W} = \rho(\mathcal{V}_1,\ldots,\mathcal{V}_n)$.
Let $\mathcal{V} \subset \mathcal{W}$ be a $G$-invariant subsheaf of $\mathcal{W}$.
Then $\mathcal{V}$ is nef.
\end{lemma}

\begin{proof}
Let $x\in X$ be any point.
The $G$-invariant subspace $\mathcal{V}_x \subset \mathcal{W}_x$ defines a
subrepresentation $\sigma$ of $\rho$, and we see that
$\mathcal{V} = \sigma(\mathcal{V}_1,\ldots,\mathcal{V}_n)$.
By Lemma \ref{semipos_rep_examples}(\ref{subreps_and_quotients_semipos}),
$\sigma$ is a nef representation, so by Lemma
\ref{apply_rho_still_semipos}, $\mathcal{V}$ is a nef locally free
sheaf.
\end{proof}

\section{Nef locally free sheaves on spaces}
\label{section-nef-bundles-spaces}
In this section, we generalize the discussion of Section~\ref{section-nef-bundles}
to algebraic spaces.\todo{Port over results.}

\section{Nakai--Moishezon for schemes}
In this section, we prove an ampleness criterion for invertible sheaves on
schemes over a field using intersection theory.
The next result can be seen as a converse to~\citeSP{0BEV}.

\begin{lemma}[Nakai--Moishezon Criterion]
\label{schemes-lemma-nakai-moishezon}
Let $X$ be a proper scheme over an algebraically closed field $k$.
Let $\mathcal{L}$ be an invertible $\mathcal{O}_X$-module.
Then $\mathcal{L}$ is ample on $X$ if and only if
for every integral closed subscheme $Y$ of $X$,
$(\mathcal{L}^{\dim(Y)} \cdot Y) > 0$.
\end{lemma}

\begin{proof}
If $\mathcal{L}$ is ample on $X$, then the positivity of the intersection
number on integral closed subschemes follows from~\citeSP{0BEV}.

Conversely, assume that for every integral closed subscheme $Y$ of $X$,
$(\mathcal{L}^{\dim(Y)} \cdot Y) > 0$.
We show that $\mathcal{L}$ is ample.
Using~\citeSP{0B5V} and~\citeSP{09MS}, we reduce to the case $X$ is integral.
We proceed by induction on $\dim(X)$.
When $\dim(X) = 1$, our assumption says that $\deg(\mathcal{L}) > 0$ and hence
$\mathcal{L}$ is ample by~\citeSP{0B5X}.

Now suppose $\dim(X) > 1$ and that the theorem is true for all proper schemes
of lower dimension.
Since $X$ is integral, $\mathcal{L}$ has a regular meromorphic section
by~\citeSP{02OZ}.
Let $\mathcal{I}_1$ be the sheaf of denominators of $\mathcal{L}$ and set
$\mathcal{I}_2 = \mathcal{I}_1 \otimes \mathcal{L}$.
Let $Y_j$ be the closed subschemes defined by $\mathcal{I}_j$ with $j = 1,2$.
By~\citeSP{02P0}, $\dim(Y_j) < \dim(X)$.
Then, for all $m \geq 0$, we the following commutative diagram with exact rows:
$$
\xymatrix{
  0 \ar[r]
    & \mathcal{I}_1 \otimes \mathcal{L}^{\otimes m} \ar[r] \ar@{=}[d]
    & \mathcal{L}^{\otimes m} \ar[r]
    & \mathcal{L}^{\otimes m}\rvert_{Y_1} \ar[r]
    & 0 \\
  0 \ar[r]
    & \mathcal{I}_2 \otimes \mathcal{L}^{\otimes (m-1)} \ar[r]
    & \mathcal{L}^{\otimes (m-1)} \ar[r]
    & \mathcal{L}^{\otimes (m-1)}\rvert_{Y_2} \ar[r]
    & 0.
}
$$
By induction, $\mathcal{L}\rvert_{Y_j}$ is ample on $Y_j$ for $j = 1,2$.
Hence by~\citeSP{0B5U}, there is some $m_0 \geq 0$
such that for all $m \geq m_0$,
$H^i(Y_j,\mathcal{L}^{\otimes m}\rvert_{Y_j}) = 0$ for all $i > 0$.
Thus, taking the long exact sequence in cohomology of the sequences above,
for $i \geq 2$,
$$
  h^i(X,\mathcal{L}^{\otimes m})
    = h^i(X,\mathcal{I}_1 \otimes \mathcal{L}^{\otimes m})
    = h^i(X,\mathcal{I}_2 \otimes \mathcal{L}^{\otimes (m - 1)})
    = h^i(X,\mathcal{L}^{\otimes (m-1)})
$$
for all $m > m_0$.
Hence, for all $m > m_0$,
$$
N := \sum\nolimits_{i = 2}^{\dim(X)} (-1)^i\,h^i(X,\mathcal{L}^{\otimes m})
$$
is a constant.
Now since $\chi(X,\mathcal{L}^{\otimes m})$ has leading coefficient
$(\mathcal{L}^{\dim X} \cdot X)$, which is positive by assumption,
we see that
$$
  \chi(X,\mathcal{L}^{\otimes m})
    = h^0(X,\mathcal{L}^{\otimes m}) - h^1(X,\mathcal{L}^{\otimes m})
      + N \to \infty
$$
as $m \to \infty$. Thus,
$h^0(X,\mathcal{L}^{\otimes m}) - h^1(X,\mathcal{L}^{\otimes m}) \to \infty$
as $m \to \infty$; in particular,
$h^0(X,\mathcal{L}^{\otimes m}) \to \infty$ as $m \to \infty$.
By Lemma~\citeSP{01PS}, we may replace $\mathcal{L}$ by
$\mathcal{L}^{\otimes m}$ to assume $\mathcal{L} = \mathcal{O}_X(D)$ for some
effective Cartier divisor $D$.

We now claim that $\mathcal{L}^{\otimes m} = \mathcal{O}_X(mD)$ is generated by
its global sections for $m \gg 0$. Consider the short exact sequence
$$
0 \longrightarrow \mathcal{O}_X((m-1)D) \longrightarrow \mathcal{O}_X(mD)
\longrightarrow \mathcal{O}_D(mD) \longrightarrow 0.
$$
Since $\mathcal{O}_D(mD)$ is ample by the inductive hypothesis,
Serre vanishing~\citeSP{0B5U} implies
$H^1(D,\mathcal{O}_D(mD)) = 0$ for $m \gg 0$, hence the maps
$$
\rho_m : H^1(X,\mathcal{O}_X((m -1)D)) \longrightarrow H^1(X,\mathcal{O}_X(mD))
$$
arising from the long exact sequence on cohomology
are surjective for all $m \gg 0$. Since the vector spaces
$H^1(X,\mathcal{O}_X(mD))$ are all finite-dimensional, we have that for $m \gg
0$, the sequence
$$h^1(X,\mathcal{O}_X(mD)) \ge h^1(X,\mathcal{O}_X((m+1)D)) \ge \cdots$$
is a nonincreasing sequence of nonnegative integers, and thus stabilizes for $m
\gg 0$. The maps $\rho_m$ are therefore bijective for $m \gg 0$, hence
\begin{equation}\label{eq:restriction-surjects}
  H^0(X,\mathcal{O}_X(mD)) \longrightarrow H^0(D,\mathcal{O}_D(mD))
\end{equation}
is surjective for $m \gg 0$. We can now show $\mathcal{O}_X(mD)$ is generated by
global sections. If $x \in X \setminus D$, then a global section in
$H^0(X,\mathcal{O}_X(mD))$ defining $mD$ generates $(\mathcal{O}_X(mD))_x$,
since $\mathcal{O}_X(mD)$ restricted to $X \setminus D$ is trivial.
Otherwise, suppose $x \in D$. By inductive hypothesis, since $\mathcal{O}_D(D)$
is ample, the invertible sheaf $\mathcal{O}_D(mD)$ is generated by global
sections for all $m \gg 0$, i.e., the morphism
$$
H^0(D,\mathcal{O}_D(mD)) \otimes_k \mathcal{O}_D \longrightarrow
\mathcal{O}_D(mD)
$$
is surjective. Thus, in the commutative diagram
$$
\xymatrix{
  H^0(X,\mathcal{O}_X(mD)) \otimes_k \mathcal{O}_X/\mathfrak{m}_x
  \ar[r]\ar@{->>}[d] & \mathcal{O}_X(mD) \otimes_{\mathcal{O}_X}
  \mathcal{O}_X/\mathfrak{m}_x \ar[d]^\simeq\\
  H^0(D,\mathcal{O}_D(mD)) \otimes_k \mathcal{O}_D/\mathfrak{m}_x \ar@{->>}[r] &
  \mathcal{O}_D(mD) \otimes_{\mathcal{O}_D} \mathcal{O}_D/\mathfrak{m}_x
}
$$
the bottom arrow is surjective. By \eqref{eq:restriction-surjects}, the left
arrow is surjective. The commutativity of the diagram implies the top row is
surjective, hence
$$
H^0(X,\mathcal{O}_X(mD)) \otimes_k \mathcal{O}_{X,x} \longrightarrow
(\mathcal{O}_X(mD))_x
$$
is surjective by Nakayama's lemma, i.e., $\mathcal{O}_X(mD)$ is generated by
global sections.

We therefore see that $\mathcal{O}_X(mD)$ induces a morphism $f : X \to
\mathbf{P}^n_k$ such that $f^*\mathcal{O}_{\mathbf{P}^n}(1) =
\mathcal{O}_X(mD)$ for $m \gg 0$. As $X$ is proper, $f$ is proper.
We now claim that for all $y \in \mathbf{P}^n_k$, the fiber $X_y$ has finitely
many points. If not, we have a commutative diagram
$$
\xymatrix{
  X \ar[r]^f & \mathbf{P}^n_k\\
  X_y \ar@{^{(}->}[u] \ar[r] & \Spec(k(y)) \ar@{^{(}->}[u]\\
  C \ar@{^{(}->}[u]\ar[ur]_\pi
}
$$
where the top square is cartesian, and $C$ is a complete curve in $X_y$. By
commutativity of the diagram, we see that
$$
  \mathcal{L}\rvert_C =
    (f^*\mathcal{O}_{\mathbf{P}^n}(1))\rvert_C \simeq
    \pi^*\mathcal{O}_{\Spec(k(y))} =
    \mathcal{O}_C,
$$
which is a contradiction, since $(\mathcal{L}^{\dim X - 1} \cdot C) > 0$ by (2),
whereas $\mathcal{O}_C$ has constant Euler characteristic. Thus, $f$ is
quasi-finite; since it is proper by the fact that both $X$ and $\mathbf{P}^n_k$
are proper, we moreover have that $f$ is finite by~\citeSP{02OG}.
Since the pullback of an ample invertible sheaf via a quasi-finite morphism is
ample by~\citeSP{0892} and finite
morphisms are quasi-finite by definition,
$\mathcal{O}_X(mD)$ is ample for $m \gg 0$.
Hence $\mathcal{O}_X(D)$ is ample by~\citeSP{01PS}.
\end{proof}

\section{Numerical intersection theory for spaces}
For the proof of the Nakai--Moishezon critera for algebraic spaces, we will
need to enhance~\citeSP{0DN0} and~\citeSP{0DN3} with generalizations of results
from~\citeSP{0BEI} and~\citeSP{0BEL}.

To begin, we require a generalization of~\citeSP{0BEK}.

\begin{lemma}
\label{lemma-euler-characteristic-morphism}
Let $k$ be a field.
Let $f : Y \to X$ be a morphism of proper algebraic spaces over $k$.
Let $\mathcal{G}$ be a coherent $\mathcal{O}_Y$-module.
Then
$$
\chi(Y, \mathcal{G}) = \sum (-1)^i \chi(X, R^if_*\mathcal{G})
$$
\end{lemma}

\begin{proof}
The formula makes sense: the sheaves $R^if_*\mathcal{G}$ are coherent
and only a finite number of them are nonzero, see~\citeSP{08AR}
and~\citeSP{073G}.
By the Leray spectral sequence~\citeSP{0732} there is a spectral sequence with
$$
E_2^{p, q} = H^p(X, R^qf_*\mathcal{G})
$$
converging to $H^{p + q}(Y, \mathcal{G})$.
By finiteness of cohomology on $X$ we see that only a finite number of
$E_2^{p, q}$ are nonzero and each $E_2^{p, q}$ is a finite dimensional vector
space.
It follows that the same is true for $E_r^{p, q}$ for $r \geq 2$ and that
$$
\sum (-1)^{p + q} \dim_k E_r^{p, q}
$$
is independent of $r$.
Since for $r$ large enough we have $E_r^{p, q} = E_\infty^{p, q}$ and since
convergence means there is a filtration on $H^n(Y, \mathcal{G})$ whose graded
pieces are $E_\infty^{p, q}$ with $p + 1 = n$ (this is the meaning of
convergence of the spectral sequence), we conclude.
\end{proof}

This is a generalization of~\citeSP{0BET} to algebraic spaces.

\begin{lemma}
\label{lemma-intersection-number-and-pullback}
Let $k$ be a field.
Let $f : Y \to X$ be a morphism of proper algebraic spaces over $k$.
Let $Z \subset Y$ be an integral closed subspace of dimension $d$ and let
$\mathcal{L}_1, \ldots, \mathcal{L}_d$ be invertible $\mathcal{O}_X$-modules.
Then
$$
(f^*\mathcal{L}_1 \cdots f^*\mathcal{L}_d \cdot Z) =
\deg(f|_Z : Z \to f(Z)) (\mathcal{L}_1 \cdots \mathcal{L}_d \cdot f(Z))
$$
where $\deg(Z \to f(Z))$ is as in
{\it \cite[\href{http://stacks.math.columbia.edu/tag/0AD6}{Tag 0AD6}]{stacks-project}}
or $0$ if $\dim(f(Z)) < d$.\todo{Prove this.}
\end{lemma}
%\begin{proof}
%The left hand side is computed using the coefficient of $n_1 \ldots n_d$
%in the function
%$$
%\chi(Y, \mathcal{O}_Z \otimes f^*\mathcal{L}_1^{\otimes n_1} \otimes
%\ldots \otimes f^*\mathcal{L}_d^{\otimes n_d}) =
%\sum (-1)^i
%\chi(X, R^if_*\mathcal{O}_Z \otimes
%\mathcal{L}_1^{\otimes n_1} \otimes \ldots \otimes
%\mathcal{L}_d^{\otimes n_d})
%$$
%The equality follows from Lemma \ref{lemma-euler-characteristic-morphism}
%and the projection formula
%\cite[\href{http://stacks.math.columbia.edu/tag/0944}{Tag 0944}]{stacks-project}.
%If $f(Z)$ has dimension $< d$, then the right hand side
%is a polynomial of total degree $<d$ by
%Lemma \ref{lemma-numerical-polynomial-from-euler}
%and the result is true. Assume $\dim(f(Z)) = d$. Let
%$\xi \in \lvert f(Z) \rvert$ be the generic point. By
%dimension theory
%the generic point of $Z$ is the unique point of $Z$ mapping to $\xi$.
%Then $f : Z \to f(Z)$ is finite over a nonempty open of $f(Z)$, see
%Morphisms, Lemma \ref{morphisms-lemma-generically-finite}.
%Thus $\deg(f : Z \to f(Z))$ is defined and in fact it is equal
%to the length of the stalk of $f_*\mathcal{O}_Z$ at $\xi$
%over $\mathcal{O}_{X, \xi}$. Moreover, the stalk of
%$R^if_*\mathcal{O}_X$ at $\xi$ is zero for $i > 0$ because
%we just saw that $f|_Z$ is finite in a neighbourhood of $\xi$
%(so that Cohomology of Schemes, Lemma
%\ref{coherent-lemma-finite-pushforward-coherent} gives the vanishing).
%Thus the terms $\chi(X, R^if_*\mathcal{O}_Z \otimes
%\mathcal{L}_1^{\otimes n_1} \otimes \ldots \otimes
%\mathcal{L}_d^{\otimes n_d})$ with $i > 0$ have total
%degree $< d$ and
%$$
%\chi(X, f_*\mathcal{O}_Z \otimes
%\mathcal{L}_1^{\otimes n_1} \otimes \ldots \otimes
%\mathcal{L}_d^{\otimes n_d})
%=
%\deg(f : Z \to f(Z)) \chi(f(Z),
%\mathcal{L}_1^{\otimes n_1} \otimes \ldots \otimes
%\mathcal{L}_d^{\otimes n_d}|_{f(Z)})
%$$
%modulo a polynomial of total degree $< d$ by
%Lemma \ref{lemma-numerical-polynomial-leading-term}.
%The desired result follows.
%\end{proof}

\section{Nakai--Moishezon for spaces}

\begin{lemma}
\label{lemma-surjective-finite-morphism-ample}
Let $R$ be a Noetherian ring.
Let $X$ and $Y$ be algebraic spaces over $R$.
Let $f : Y \to X$ be a proper morphism of algebraic spaces over $R$.
Let $\mathcal{L}$ be an invertible $\mathcal{O}_X$-module.
Assume $f$ is finite and surjective.
Then $\mathcal{L}$ is ample if and only if $f^*\mathcal{L}$ is ample.
\end{lemma}

\begin{proof}
Suppose that $\mathcal{L}$ is ample.
Then $X \to \Spec(R)$ is representable and hence $X$ is a scheme.
But $f : Y \to X$ is finite and hence affine, so,
by~\citeSP{03WG}, $f$ is representable.
Therefore $Y$ is a scheme.
Then $f^*\mathcal{L}$ is ample by the schemes case,~\citeSP{0B5V}.

\medskip\noindent
Assume that $f^*\mathcal{L}$ is ample.
Let $P$ be the following property on coherent $\mathcal{O}_X$-modules
$\mathcal{F}$:
there exists an $n_0$ such that $H^p(X, \mathcal{F} \otimes
\mathcal{L}^{\otimes n}) = 0$ for all $n \geq n_0$ and $p > 0$.
We will prove that $P$ holds for any coherent $\mathcal{O}_X$-module
$\mathcal{F}$, which suffices to prove that $\mathcal{L}$ is ample.
We are going to apply~\citeSP{07UT}.
Thus we have to verify (1), (2) and (3) of that lemma for $P$.
Property (1) follows from the long exact cohomology sequence associated
to a short exact sequence of sheaves and the fact that tensoring with
an invertible sheaf is an exact functor.
Property (2) follows since $H^p(X, -)$ is an additive functor.
To see (3) let $Z \subset X$ be reduced closed subspace with $|Z|$
irreducible.
Form the fibre product diagram
$$
\xymatrix{
  W \ar[r]_{i'} \ar[d]_{f'} & Y \ar[d]^{f} \\
  Z \ar[r]^{i} & X.
}
$$
Let $\mathcal{G} = f'_*\mathcal{O}_W$.
Since $f'$ is surjective, part (3)(a) of~\citeSP{07UT} holds.
For part (3)(b), let $\mathcal{I} \subset \mathcal{O}_Z$ be a nonzero sheaf
of ideals.
Let $\mathcal{G}' = \mathcal{I}\mathcal{G}$.
We have
$$
\mathcal{I} \mathcal{G} = f'_*(\mathcal{I}')
$$
where $\mathcal{I}' = \Im((f')^*\mathcal{I} \to \mathcal{O}_W)$.
This is true because $f'$ is a (representable) affine morphism of
algebraic spaces and hence the result can be checked on an \'etale
covering of $Z$ by a scheme in which case the result is~\citeSP{01YP}.
Finally, $f'$ is affine, hence $R^pf'_*\mathcal{I}' = 0$ for all $p > 0$
by~\citeSP{073H}.
Hence
\begin{align*}
H^p(X, \mathcal{G}' \otimes_{\mathcal{O}_X} \mathcal{L}^{\otimes n})
  & = H^p(X, i_*f_*'\mathcal{I}' \otimes_{\mathcal{O}_X} \mathcal{L}^{\otimes n}) \\
  & = H^p(X, f_*(i'_*\mathcal{I}' \otimes_{\mathcal{O}_Y} f^*\mathcal{L}^{\otimes n})) \\
  & = H^p(Y, i'_*\mathcal{I}' \otimes_{\mathcal{O}_Y} f^*\mathcal{L}^{\otimes n}) = 0
\end{align*}
since $f^*\mathcal{L}$ is ample.
This verifies (3)(c) of~\citeSP{07UT} as desired.
\end{proof}

\begin{lemma}[Nakai--Moishezon Criterion for algebraic spaces]
\label{spaces-lemma-nakai-moishezon}
Let $k$ be a field.
Let $X$ be a proper algebraic space over $k$.
Let $\mathcal{L}$ be an invertible $\mathcal{O}_X$-module.
Then $\mathcal{L}$ is ample on $X$ with respect to $k$ if and only if
for every integral closed subspace $Y$ of $X$,
$(\mathcal{L}^{\dim(Y)} \cdot Y) > 0$.
\end{lemma}
\begin{proof}
  Assume that $\mathcal{L}$ is ample on $X$ with respect to $k$.
  Then $X$ is a scheme, and $\mathcal{L}$ is ample on $X/k$ in the
  scheme-theoretic sense by~\citeSP{0D32}.
  Now~\citeSP{0BEV} implies the positivity of interesction numbers.

  Suppose that for every integral closed subspace $Y$ of $X$,
  $(\mathcal{L}^{\dim(Y)} \cdot Y) > 0$.
  We show that $\mathcal{L}$ is ample on $X$ with respect to $k$.
  By~\citeSP{09YC}, there is a finite surjective map $p : X' \to X$ from a
  scheme $X'$.
  By Lemma~\ref{lemma-surjective-finite-morphism-ample}, $\mathcal{L}$ is
  ample if and only if $p^*\mathcal{L}$ is ample.
  Thus, by the Nakai--Moishezon criterion for schemes
  (Lemma~\ref{schemes-lemma-nakai-moishezon}),
  it suffices to show that for every integral closed subscheme $Y'$ in $X'$, we
  have $(p^*\mathcal{L}^{\dim(Y')} \cdot Y') > 0$.
  Let $Y$ be the image of $Y'$ in $Y$;
  by Lemma~\ref{lemma-intersection-number-and-pullback}, we then have
  $$
  ( (p^*\mathcal{L})^{\dim Y'} \cdot Y') = \deg(p\rvert_{Y'} : Y' \to
  Y)(\mathcal{L}^{\dim Y'} \cdot Y).
  $$
  Note $\dim Y' = \dim Y$ since $p\rvert_{Y'} : Y' \to Y$ is finite surjective.
  Since $(\mathcal{L}^{\dim Y} \cdot Y) > 0$ by (2), we therefore see that
  $( (p^*\mathcal{L})^{\dim Y'} \cdot Y') > 0$.
\end{proof}

\section{Ampleness Lemma}
In this section, we formulate and prove the Ampleness Lemma
of~\cite[Lemmas 3.9 and 3.13]{ko90}.

\begin{lemma}
\label{lemma-pgl-bundle-from-loc-free}
Let $k$ be a field.
Let $X$ be a scheme over $k$.
Let $\mathcal{E}$ be a locally free sheaf of rank $n$.
Let $\mathbf{P} = \mathbf{P}((\mathcal{E}^\vee)^{\oplus n})$.
Let $\pi : \mathbf{P} \to X$ be the structure map.
Then there exists a closed subscheme $D \subset \mathbf{P}$ such that the
the restriction
$$
\pi : \mathbf{P} \setminus D \to X
$$
exhibits $\mathbf{P} \setminus D$ as a $\mathrm{PGL}_n$-torsor over $X$.
\end{lemma}

\begin{proof}
First we should describe the $\mathrm{PGL}_n$ action on $\mathbf{P}$.
Let $T$ be a scheme over $k$ and $x : T \to \mathbf{P}$ be a $T$-point.
By~\citeSP{01NK}, this corresponds to a morphism $f : T \to X$ of $k$-schemes
together with a surjection
$$
\varphi = (\varphi_1,\ldots,\varphi_n) :
f^*(\mathcal{E}^\vee)^{\oplus n} \to \mathcal{L},
$$
where $\mathcal{L}$ an invertible sheaf on $T$.
Let $g \in \mathrm{PGL}_n(T)$ and pick a representative matrix
$(g_{ij}) \in \mathrm{GL}_n(T)$ for $g$.
Consider the morphism
$$
g \cdot \varphi =
\big(\sum\nolimits_j g_{1j}\varphi_j,\ldots,\sum\nolimits_j g_{nj}\varphi_j\big)
: f^*(\mathcal{E}^\vee)^{\oplus n} \to \mathcal{L}
$$
of $\mathcal{O}_T$-modules.
Since $\varphi$ is surjective and $(g_{ij})$ is invertible, $g \cdot \varphi$
is surjective.
Therefore $g \cdot \varphi$ defines a $T$-point of $\mathbf{P}$.
This is $g \cdot x$.
Note that $g \cdot x$ is independent of the representative $(g_{ij})$ of $g$
as a different representative of $g$ differs by an element of $\mathbf{G}_m(T)$
and the resulting morphism differs by an automorphism of $\mathcal{L}$, so the
resulting $T$-point of $\mathbf{P}$ will be the same.

Write $\mathbf{P} = \mathbf{P}((\mathcal{E}^\vee)^{\oplus n})$ and let
$\pi : \mathbf{P} \to X$ be the structure morphism.
Using the projection formula~\citeSP{01E6},
\begin{align*}
H^0(\mathbf{P}, \pi^*\mathcal{E}(1))
  & \cong H^0(X, \pi_*(\pi^*\mathcal{E} \otimes_{\mathcal{O}_{\mathbf{P}}}
      \mathcal{O}_{\mathbf{P}}(1))) \\
  & \cong H^0(X,\mathcal{E} \otimes_{\mathcal{O}_X}
      \pi_*\mathcal{O}_{\mathbf{P}}(1)) \\
  & \cong H^0(X,\mathcal{E} \otimes (\mathcal{E}^\vee)^{\oplus n}) \\
  & \cong H^0(X,\mathrm{End}(\mathcal{E})^{\oplus n})
\end{align*}
has $n$ sections, corresponding to the identity on each summand.
We therefore have an evaluation morphism
$$
B : \mathcal{O}_{\mathbf{P}}^{\oplus n} \longrightarrow \pi^*\mathcal{E}(1).
$$
Taking determinants, this gives a map
$$
\det(B) : \mathcal{O}_{\mathbf{P}} \longrightarrow \det(\pi^*\mathcal{E})(n).
$$
Let $\mathcal{I} \otimes \det(\pi^*\mathcal{E}) (n)$ be the image of $\det B$,
and let $D$ be the subscheme of $\mathbf{P}$ defined by $\mathcal{I}$.
Then, $D$ defines the locus for which $\det B$ vanishes, which is the locus of
$\mathbf{P}$ where the corresponding matrices are not invertible.
We therefore see that $\mathbf{P} \setminus D$ corresponds to those matrices
that are invertible, i.e., are elements of $\mathrm{PGL}_n$.
This turns $\mathbf{P} \setminus D$ into a principal $\mathrm{PGL}_n$-bundle
over $X$.


\end{proof}

\begin{lemma}
\label{lemma-grassman-quotient-stack-map}
Let $X$ be a scheme over a field $k$.
Let $\mathcal{E}$ be a locally free sheaf on $X$ of rank $n$.
Let $\mathcal{F} = \mathrm{Sym}^d(\mathcal{E})$ for some $d \geq 1$ and
let $N = \binom{n + d}{d}$ be the rank.
Let $\mathcal{F} \to \mathcal{Q}$ be a locally free quotient of rank $q$.
Then the following hold:
\begin{enumerate}
\item There exists a closed subscheme
  $D \subset \mathbf{P}((\mathcal{E}^\vee)^{\oplus n})$ such that
$$
\mathbf{P}((\mathcal{E}^\vee)^{\oplus n}) \setminus D \longrightarrow X
$$
is a principal $\mathrm{PGL}_n$-bundle.
\item There is a $\mathrm{PGL}_n$-action on $\mathbf{G}(q,N)$, and
a $\mathrm{PGL}_n$-equivariant map
$$
\mathbf{P}((\mathcal{E}^\vee)^{\oplus n}) \setminus D \longrightarrow
\mathbf{G}(q,N).
$$
\item There exists a map of stacks
$$
X \longrightarrow [\mathbf{G}(q,N)/\mathrm{PGL}_n].
$$
\end{enumerate}
\end{lemma}
\begin{proof}
We first prove (1).
For (2), note that $\mathrm{PGL}_n$ acts on $\mathbf{G}(q,N)$ via
$$
  \mathrm{GL}_n \longrightarrow
  (\mathrm{GL}_n)^{\times d} \longrightarrow
  \mathrm{Aut}(\mathrm{Sym}^d (k^{\oplus n})),
$$
which descends to their quotients by $k^*$; note that
$\mathrm{Aut}(\mathrm{Sym}^d (k^{\oplus n}))/k^*$ is a subgroup of the
automorphism group of the Grassmannian $\mathbf{G}(q,N)$.
To show the morphism claimed exists, we fist consider what happens when we
apply $\mathrm{Sym}^d$ to $B$, giving a morphism
$$
  \mathrm{Sym}^d(B) : \mathcal{O}_{\mathbf{P}}^{\oplus N} \to
    \pi^*\mathrm{Sym}^d(\mathcal{E})(d) = \pi^*\mathcal{F}(d).
$$
Composing this will the given surjection $\mathcal{F} \to \mathcal{Q}$, we
obtain a morphism
$$
  U : \mathcal{O}_{\mathbf{P}}^{\oplus N} \to
        \pi^*\mathcal{F}(d) \to
        \pi^*\mathcal{Q}(d)
$$
which is surjective away from $D$.
By the universal property of the Grassmannian, we obtain a morphism
$\mathbf{P} \setminus D \to \mathbf{G}(q,N)$.
It is $G$-equivariant by tracing the definitions.

For (3), this follows from general properties of quotient stacks, plus the fact
that $(\mathbf{P} \setminus D)/\mathrm{PGL}_n \simeq X$.
\end{proof}

\begin{lemma}
\label{lemma-positive-self-intersection}
Let $X$ be a normal projective variety over a field $k$.
Let $\mathcal{E}$ be a nef locally free sheaf on $X$ of rank $n$.
Set $\mathcal{F} = \mathrm{Sym}^d(\mathcal{E})$ for some $d \geq 1$ and set
$N = \binom{n + d}{d}$ be the rank.
Let $\mathcal{Q}$ be a locally free quotient of $\mathcal{F}$ of rank $q$.
Assume that the map
$$
  u : X \to [\mathbf{G}(q,N)/G]
$$
constructed in Lemma \ref{lemma-grassman-quotient-stack-map}
is generically quasi-finite.
Then $(\det(\mathcal{Q})^{\dim(Y)}) > 0$.
\end{lemma}

\begin{proof}
Let $\mathbf{P} = \mathbf{P}((\mathcal{E}^\vee)^{\oplus n})$.
Let $D \subset \mathbf{P}$ be the closed subscheme given by part (1) of
Lemma \ref{lemma-grassman-quotient-stack-map}.
Then there is a fibre product diagram\todo{Why, exactly, is this a fibre product
diagram? Also, the $u$ is off.}
$$
\xymatrix{
  \mathbf{P} \setminus D \ar[r] \ar[d] & \mathbf{G}(q,N) \ar[d] \\
  X \ar[r]^u & [\mathbf{G}(q,N)/\mathrm{PGL}_m]
}
$$
Since $u$ is generically quasi-finite, the map
$\mathbf{P} \setminus D \to \mathbf{G}(q,N)$ is generically quasi-finite.

As in Lemma \ref{lemma-grassman-quotient-stack-map},\todo{Maybe factor
  this statement out?}
the morphism $\mathbf{P} \setminus D \to \mathbf{G}(q,N)$ is defined by
the morphism of locally free sheaves
$$
  U : \mathcal{O}_{\mathbf{P}}^{\oplus n} \to \pi^*\mathcal{F}(d)
                                          \to \pi^*\mathcal{Q}(d).
$$
Here, the map $\mathcal{O}_{\mathbf{P}}^{\oplus N} \to \pi^*\mathcal{F}$ is
an isomorphism away from $D$ and thus $U$ is surjective away from $D$.
Taking $q$\textsuperscript{th} exterior powers, we obtain a morphism
$$
  \bigwedge^q U : \mathcal{O}_{\mathbf{P}}^{\oplus \binom{N}{q}} \to
    \pi^*\det(\mathcal{Q})(qd)
$$
which is surjective away from $D$.
Thus the image of $\bigwedge^q U$ is of the form
$\pi^*\det(\mathcal{Q})(qd) \otimes \mathcal{I}$ for some ideal sheaf
$\mathcal{I} \subset \mathcal{O}_{\mathbf{P}}$.

Let $g : \mathbf{P}' \to \mathbf{P}$ be the blow up of $\mathbf{P}$ along the
sheaf $\mathcal{I}$.
Then the rational map from $\mathbf{P}$ to $\mathbf{G}(q,N)$ defined above
resolves to a morphism
$$
  u' : \mathbf{P}' \to \mathbf{G}(q,N)
$$
such that
$$
  g^*\pi^*\det(\mathcal{Q})(qd) \cong
    u'^*\mathcal{O}_{\mathbf{G}(q,N)}(1)
      \otimes_{\mathcal{O}_{\mathbf{P}'}}
    \mathcal{O}_{\mathbf{P}'}(E)
$$
where $E$ is the exceptional divisor of $g : \mathbf{P}' \to \mathbf{P}$.

Since $u$ is generically quasi-finite and $g$ is birational,
$u'$ is generically quasi-finite.
Hence, by ...\todo{Prove this or reference this somewhere.}
$u'^*\mathcal{O}_{\mathbf{G}(q,N)}(1)$ is big\todo{Define big somewhere}
on $\mathbf{P}'$.
Fix an ample invertible sheaf $\mathcal{L}$ on $Y$.
Then by ...\todo{Using properties of big. I guess this depends on what the definition
  is and then what the characterizations are.}, there is some $m \geq 0$
such that
$\mathcal{O}_{\mathbf{G}(q,N)}(m)
  \otimes_{\mathcal{O}_{\mathbf{P}'}} g^*\pi^*\mathcal{L}^{-1}$
has a nonzero section.
But
\begin{align*}
  \mathcal{O}_{\mathbf{G}(q,N)}(m)
  = \big(g^*\pi^*\det(\mathcal{Q})(qd) \otimes_{\mathcal{O}_{\mathbf{P}'}}
    \mathcal{O}_{\mathbf{P}'}(-E)\big)^{\otimes m}.
\end{align*}
Thus we obtain a nonzero morphism of sheaves
$$
\mathcal{O}_{\mathbf{P}'} \to
    \big(g^*\pi^*\det(\mathcal{Q})(qd) \otimes_{\mathcal{O}_{\mathbf{P}'}}
      \otimes_{\mathcal{O}_{\mathbf{P}'}}
    \mathcal{O}_{\mathbf{P}'}(-E)\big)^{\otimes m}
      \otimes_{\mathcal{O}_{\mathbf{P}'}}
    g^*\pi^*\mathcal{L}^{-1}
$$
for all $m \geq 0$ large.
Since $E$ is effective, we can compose this morphism with the natural
inclusion $\mathcal{O}_{\mathbf{P}'}(-E) \to \mathcal{O}_{\mathbf{P}'}$
to obtain a nonzero morphism
$$
\mathcal{O}_{\mathbf{P}'} \to
    g^*\big(\pi^*\det(\mathcal{Q})^{\otimes m}(mqd)
      \otimes_{\mathcal{O}_{\mathbf{P}}}
    \pi^*\mathcal{L}^{-1}\big).
$$
Pushing this forward to $X$ and applying the projection formula, we obtain
a nonzero morphism
$$
\mathcal{O}_X \to
  \det(\mathcal{Q})^{\otimes m}
    \otimes_{\mathcal{O}_X}
  \pi_*\mathcal{O}_{\mathbf{P}}(mqd)
    \otimes_{\mathcal{O}_X}
  \mathcal{L}^{-1}.
$$
Since each term on the right is locally free, this is equivalent to a nonzero
morphism
$$
\tau :
  (p_*\mathcal{O}_{\mathbf{P}}(mqd))^\vee \otimes_{\mathcal{O}_X} \mathcal{L}
    \to \det(\mathcal{Q})^{\otimes m}.
$$
As before, there exists an ideal sheaf $\mathcal{J} \subseteq \mathcal{O}_X$
such that the image of $\tau$ is the subsheaf
$\mathcal{J} \otimes_{\mathcal{O}_X} \det(\mathcal{Q})^{\otimes m}$ of
$\det(\mathcal{Q})^{\otimes m}$.
Let $s : X' \to X$ be the blow up of $X$ along $\mathcal{J}$.
Set
$P = \mathbf{P}((\pi_*\mathcal{O}_{\mathbf{P}}(mqd))^\vee
        \otimes_{\mathcal{O}_X} \mathcal{L})$.
Then $\tau$ gives rise to a morphism
$v : X' \to P$
such that
$$
v^*\mathcal{O}_P(1) = s^*\det(\mathcal{Q})^{\otimes m}
                        \otimes_{\mathcal{O}_{X'}} \mathcal{O}_{X'}(-F)
$$
for the exceptional divisor $F$ of the blow up $s : X' \to X$.
Set $P' = \mathbf{P}((\pi_*\mathcal{O}_{\mathbf{P}}(mqd))^\vee)$.
Then there exists a canonical isomorphism\todo{Find a reference for this.}
$P \to P'$.
Let $v' : X' \to P'$ be the composition of $v$ with this isomorphism.
Then
$
\mathcal{L} \otimes_{\mathcal{O}_{X'}} v'^* \mathcal{O}_{P'}(1)
  = v^* \mathcal{O}_P
$, so
$$
  s^*\det(\mathcal{Q})^{\otimes m}
    = s^*\mathcal{L}
        \otimes_{\mathcal{O}_{X'}} v'^* \mathcal{O}_{P'}(1)
        \otimes_{\mathcal{O}_{X'}} \mathcal{O}_{X'}(F).
$$

We now compute the top self-intersection of $\det(\mathcal{Q})^{\otimes m}$.
Since $s : X' \to X$ is birational, by~\citeSP{0BET},
$$
  ((\det(\mathcal{Q})^{\otimes m})^{\dim(X)})
    = ((s^*\det(\mathcal{Q})^{\otimes m})^{\dim(X')}).
$$
Write
$\mathcal{L}'
  = v'^*\mathcal{O}_{P'}(1) \otimes_{\mathcal{O}_{X'}} \mathcal{O}_{X'}(F)$.
By~\citeSP{0BER} we further have
$$
  ((\det(\mathcal{Q})^{\otimes m})^{\dim(X)})
    = (s^*\mathcal{L}^{\dim(X)})
      + \sum\nolimits_{i = 1}^{\dim(X)}
          (s^*\mathcal{L}^{\dim(X) - i} (s^*\det(\mathcal{Q})^{\otimes m})^{i - 1} \mathcal{L}').
$$
Since $(\pi_*\mathcal{O}_{\mathbf{P}}(mqd))^\vee$\todo{Explain why this is of
  the required form.} is nef by
Lemma~\ref{lemma-star-sym-star-nef},
$\mathcal{O}_{P'}(1)$ is nef and hence its $v'$ pullback to $X'$ is nef.
As $F$ is effective, $\mathcal{L}'$ is nef.
Since $\mathcal{L}$ is ample, $s^*\mathcal{L}$ is nef, so
$s^*\det(\mathcal{Q})^{\otimes m}$ is nef.
All this shows that the second term in the sum above is nonnegative.
Applying~\citeSP{0BET} again,
$$
  (s^*\mathcal{L}^{\dim(X)}) = (\mathcal{L}^{\dim(X)}) > 0
$$
since $\mathcal{L}$ is ample on $X$.
Thus $((\det(\mathcal{Q})^{\otimes m})^{\dim(X)}) > 0$, so by additivity of
numerical intersections again, $(\det(\mathcal{Q})^{\dim(X)}) > 0$.
\end{proof}

\begin{lemma}
\label{lemma-ampleness-of-semipos-quotients}
Let $k$ be a field.
Let $X$ be a proper algebraic space over $k$.
Let $\mathcal{E}$ be a nef locally free sheaf of rank $n$ on $X$.
Set $\mathcal{F} = \mathrm{Sym}^d(\mathcal{E})$ for some $d \geq 1$ and
let $N = \binom{n + d}{d}$ be the rank.
Let $\mathcal{Q}$ be a locally free quotient of $\mathcal{F}$ of rank $q$.
Assume that the map
$$
u : X \to [\mathbf{G}(q,N)/\mathrm{PGL}_n]
$$
constructed in Lemma \ref{lemma-grassman-quotient-stack-map}
is generically quasi-finite.
Then $\det(\mathcal{Q})$ is ample.
\end{lemma}

\begin{proof}
By the Nakai--Moishezon Criterion for spaces,
Lemma~\ref{spaces-lemma-nakai-moishezon},
we must show that $\det(Q)$ has positive top intersection with every
irreducible subspace $Z \subset X$.
So let $Z \subset X$ be any irreducible subspace and write $i : Z \to X$ for
the inclusion.
By~\citeSP{088U} there is a projective scheme $Z'$ and a proper birational
morphism $g : Z' \to Z$.
Now
$$
  (\det(Q)^{\dim(Z)} \cdot Z)
    = (\det(i^*Q)^{\dim(Z)})
    = (\det(g^*i^*Q)^{\dim(Z')})
$$
so we need to show that the top self-intersection of $\det(g^*i^*\mathcal{Q})$
on $Z'$ is positive.
But now $g^*i^*\mathcal{E}$ is nef by Lemma~\ref{lemma-nef-pullback-nef-spaces},
$g^*i^*\mathcal{F} \cong \mathrm{Sym}^d(g^*i^*\mathcal{E})$,
and $g^*i^*\mathcal{Q}$ is a quotient bundle of $g^*i^*\mathcal{F}$ of rank $q$
on $Z'$.
Also, the map $u \circ i \circ g : Z' \to [\mathbf{G}(q,N)/\mathrm{PGL}_n]$
is quasi-finite on the dense open on which $g$ is an isomorphism.
Thus Lemma~\ref{lemma-positive-self-intersection},
$(\det(g^*i^*\mathcal{Q})^{\dim(Z')}) > 0$ and we are done.
\end{proof}


\section{Nefness results}
In this section, we prove a nefness result,
Lemma \ref{lemma-stable-curves-nef}, for the pushforward of the
relative dualizing sheaf of families of curves.

\begin{situation}\label{kollar_prop_4.5}
Let $k$ be a field.
Let $C$ be a connected, proper smooth $k$-scheme of dimension 1.
Let $S$ be a connected, proper smooth $k$-scheme of dimension 2.
Let $f:S\to C$ be a surjective morphism of $k$-schemes.
Assume that the general fiber of $f$ is smooth (over its residue field) of genus at least 2.
Let $m\ge2$ be an integer.
\end{situation}

\begin{lemma}\label{pushforward_invertible}
Suppose we are in Situation \ref{kollar_prop_4.5}.
Then, $f_*\omega_{S/C}^{\otimes m}$ is an invertible sheaf on $C$.
\end{lemma}

\begin{proof}
Immediate from Grauert's Theorem and Riemann-Roch.
\end{proof}

%throughout, when we prove these things are nef maybe we need to keep referring to this lemma...

\begin{lemma}\label{ekedahl_corollary}
Let $S$ be a connected, proper smooth $k$-scheme of dimension 2.
Assume that $S$ is a surface of general type.
Let $m\ge2$ be an integer.
Let $\mathcal{L}$ be an invertible sheaf on $S$ that is effective.
Assume that $\mathcal{L}^2=0$.
Then, we have $H^1(\omega_X^{\otimes m}\otimes\mathcal{L})=0$ if $\mathrm{char}(k)\neq2$ and  $H^1(\omega_X^{\otimes m}\otimes\mathcal{L})=0\leq 1$ if $\mathrm{char}(k)=2$.
\end{lemma}
\begin{proof}
Let $D\subset S$ be an effective divisor such that $\mathcal{L}\cong\mathcal{O}_S(D)$.
Then we get an exact sequence
$$
0\to \omega_S^{\otimes m}\to \omega^{\otimes m}_S\otimes \mathcal{L}\to (\omega_S^{\otimes m}\otimes \mathcal{L})|_D\to 0.
$$
From the long exact sequence for cohomology we get
$$
H^1(S,\omega_S^{\otimes m})\to H^1(S,\omega_S^{\otimes m}\otimes \mathcal{L})\to H^1(S,(\omega_S^{\otimes m}\otimes \mathcal{L})|_D)
$$

By assumption, $D^2=0$, so
$$
(mK_S+D)|_D=(mK_S+mD)|_D=mK_D,
$$
where the last equality follows from the adjunction formula.
Then for $m\geq 2$ we get
$$
H^1(S,(\omega_S^{\otimes m}\otimes \mathcal{L})|_D)\cong H^1(D,\omega_D^{\otimes m})\cong H^0(D,\omega_D^{\otimes(1-m)})^\vee=0
$$
by Serre Duality.

Therefore, the result follows from the analogous result of Ekedahl \todo{add ref}for $H^{1}(S,\omega_S^{\otimes m})$.
\end{proof}




\begin{lemma}\label{4.5_minimal_base_genus_2_char_p}
Suppose we are in Situation \ref{kollar_prop_4.5}.
Assume the characteristic of $k$ is equal to $p>0$.
Assume that $S$ is minimal, i.e. it has no $(-1)$-curves.
Assume that the genus of $C$ is at least 2.
Then, $f_*\omega_{S/C}^{\otimes m}$ is nef for $m \geq 2$.
\end{lemma}

\begin{proof}
First, observe that $S$ is of general type. \todo{why?}

Suppose there exists a surjection of invertible sheaves
$$f_*(\omega^{\otimes m}_{S/C})\to \mathcal{M}^{-1},$$
where $\mathcal{M}$ an invertible sheaf and $\deg \mathcal{M}=d<0$.
Let $F_S:S\to S$ and $F_C:C\to C$ denote the absolute Frobenius morphisms
By functoriality for the relative dualizing sheaf,
$$
F_C^*f_*(\omega^{\otimes m}_{S/C})\cong f_*F_S^*(\omega^{\otimes m}_{S/C})=f_*(\omega^{\otimes m}_{S/C})
$$
Thus, $f_*(\omega^{\otimes m}_{S/C})$ has the quotient invertible sheaf $F^*\mathcal{M}^{-1}$ with $\deg F^*\mathcal{M}^{-1}=dp$, and we can replace $\mathcal{M}$ with $F^{*}\mathcal{M}^{-1}$. Hence, we can assume that $\deg \mathcal{M}=d>>0$, and assume that $\mathcal{M}\cong \omega_{C}^{\otimes m}\otimes \mathcal{L}$ where $\mathcal{L}$ is very ample.

We therefore have a surjection of sheaves
$$\omega_C^{\otimes m}\otimes \mathcal{L} \otimes f_*(\omega^m_{S/C})\to\mathcal{O}_C\to 0$$
which yields the surjection
$$H^1(C, \omega^{\otimes m}_C\otimes \mathcal{L}\otimes f_*(\omega^{\otimes m}_{S/C}))\to H^1(\mathcal{O}_C)\to 0$$
after applying the long exact sequence for cohomology, as $C$ has dimension 1.

By Serre Duality, we have
$$\dim_k H^1(\mathcal{O}_C)\cong \dim_k H^0(\omega_C)=g.$$
Therefore, $$\dim_k H^1(C, \omega^{\otimes m}_C\otimes \mathcal{L}\otimes f_*(\omega^{\otimes m}_{S/C}))\geq g$$

Let $\mathcal{F}=f^*(\omega^{\otimes m}_C\otimes \mathcal{L})\otimes \omega^{\otimes m}_{S/C}\cong \omega_{S}^{\otimes m}\otimes f^{*}\mathcal{L}$ by \todo{add ref}. We have the Leray Spectral Sequence
$$E_2^{p,q} \quad H^p(C, R^qf_*\mathcal{F})\implies H^{p-q}(S,\mathcal{F}),$$
whose only non-zero entries on the $E_2$-page are $(p,q)=(0,0),(0,1),(1,0),(1,1)$, and thus degenerates on the $E_2$ page.

%The only nonzero entries on the $E_2$ page are:
%\begin{center}
%$H^1(C, f_*\mathcal{F})$\quad  \quad $H^1(C, R^1f_*\mathcal{F})$\\
%$H^0(C, f_*\mathcal{F})$ \quad \quad $H^0(C,R^1f_\mathcal{F})$
%\end{center}

Hence, we have the short exact sequence
$$0\to H^1(C, f_*\mathcal{F})\to H^1(S, \mathcal{F})\to H^0(C, R^1f_*\mathcal{F})\to 0$$

By the projection formula,
$$
f_*\mathcal{F}\cong\mathcal{L}\otimes f_*(\omega^{\otimes m}_{S/C})\otimes \omega^m_C
$$
Thus, $\dim_k H^1(C,f_{*}\mathcal{F})\ge g$, from which we conclude
$$
\dim_k H^1(S,\mathcal{F})=\dim_kH^{1}(S,\omega_{S}^{\otimes m}\otimes f^{*}\mathcal{L})\ge g\ge2.
$$

However, the invertible sheaf $f^*\mathcal{L}$ is effective on $S$ because $\mathcal{L}$ is very ample on $C$, and has self-intersection zero because it is a union of fibers, so this contradicts Lemma \ref{ekedahl_corollary}.
\end{proof}

\begin{lemma}\label{4.5_base_genus_2_char_p}
Suppose we are in Situation \ref{kollar_prop_4.5}.
Assume the characteristic of $k$ is equal to $p>0$.
Assume that the genus of $C$ is at least 2.
Then, $f_*\omega_{S/C}^{\otimes m}$ is nef for $m \geq 2$.
\end{lemma}

\begin{proof}
We will reduce to the case in which $S$ is minimal, then apply Lemma
\ref{4.5_minimal_base_genus_2_char_p}.

Suppose that $S$ contains a $(-1)$-curve $C'$.
Then, if $C'$ is not contained in a fiber of $f$, then $f|_{C'}:C'\to C$ is a
dominant morphism from a curve of genus 0 to a curve of genus 2, which is a
contradiction.
Thus, all $(-1)$-curves of $S$ are contained in fibers of $f$.
Thus, we have a $k$-smooth minimal model $S_{min}$ along with $k$-morphisms
$b:S\to S_{min}$ and $f':S_{min}\to C$ such that $f=f'\circ b$.

The morphism $f'$ satisfies the hypotheses of Lemma
\ref{4.5_minimal_base_genus_2_char_p}, so
$f'_*(\omega_{S_{min}/C}^{\otimes m})$ is nef.
Then
\begin{align*}
  f_*(\omega^{\otimes m}_{S/C})
    & \cong f'_*b_*(\omega^{\otimes m}_{S/C})
      \cong f'_*b_*(\omega^{\otimes m}_{S/S_{min}}\otimes b^*\omega^{\otimes m}_{S_{min}/C})\\
    & \cong f'_*(b_*\omega^{\otimes m}_{S/S_{min}}\otimes \omega^{\otimes m}_{S_{min}/C})
      \cong f'_*\omega^{\otimes m}_{S_{min}/C}
\end{align*}
because $b_*\omega^{\otimes m}_{S/S_{min}}\cong\mathcal{O}_{S_{min}}$.\todo{this is probably in the stacks project somewhere, find ref}
The conclusion follows.
\end{proof}



\begin{lemma}\label{4.5_char_p}
Suppose we are in Situation \ref{kollar_prop_4.5}.
Assume the characteristic of $k$ is equal to $p>0$.
Then, $f_*\omega_{S/C}^{\otimes m}$ is nef for $m \geq 2$.
\end{lemma}

\begin{proof}
\todo{get rid of the beginning of this and cite \ref{base_change_okay}
  instead -- that needs to be moved up}
We will reduce to the case in which the genus of $C$ is at least 2, then apply
Lemma \ref{4.5_minimal_base_genus_2_char_p}.

Because $f$ is generically smooth, there exists a connected, smooth proper $k$-scheme $C'$ of dimension 1 and a generically \'{e}tale $k$- morphism $f:C'\to C$ branched only at points of $C$ above which the fiber of $f$ is smooth\todo{add ref}. We then have a fiber diagram
$$
\xymatrix{
S \ar[d]^f  &S' \ar[d]^{f'} \ar[l]^{\pi'} \\
C   &C' \ar[l]^{\pi}
}
$$
and $f':S'\to C'$ satisfies the hypotheses of Lemma \ref{4.5_minimal_base_genus_2_char_p}, so $f'_*(\omega_{S'/C'})$ is nef.

By Cohomology and Base Change and compatibility for relative dualizing sheaves, we have
$$
\pi^*(f_*(\omega_{S/C}))\cong f'_*\pi'^*(\omega_{S/C})\cong f'_*(\omega_{S'/C'}).
$$
Now, the locally free sheaf $f_{*}(\omega_{S/C})$ becomes nef after pullback, and is therefore nef.
\end{proof}



Compare the following with~\cite[Theorem 4.3]{ko90}.

\begin{lemma}
\label{lemma-stable-curves-nef}
Suppose we are in situation Lemma \ref{kollar_prop_4.5}.
Then $f_*(\omega_{S/C}^{\otimes m})$ is nef for $m \geq 2$.
\end{lemma}

\begin{proof}
By Lemma~\ref{4.5_char_p}, we may assume that $k$ has characteristic 0.
Spread out. \todo{add details}
\end{proof}

Compar the following with~\cite[Theorem 4.6]{ko90}.

\begin{lemma}
\label{lemma-stable-curves-nef}
Let $k$ be a field.
Let $C$ be a proper connected smooth $k$-scheme of dimension $1$.
Let $f : S \to C$ be a family of stable curves,
as defined in Lemma \ref{lemma-stable-curves}.
Let $C_t$ be a section of $f$.
Then $\omega_{S/C}\cdot C_t\geq 0$.
\end{lemma}

\begin{proof}
Let $i:C_t\hookrightarrow S$ be the inclusion map, and $g=f\circ i:C_t\to C$ be the corresponding isomorphism.
Consider the short exact sequence:
$$0\to \mathcal{O}_S(-C_t)\to\mathcal{O}_S\to i_*\mathcal{O}_{C_t}\to 0.$$
Tensoring with $\omega_{S/C}^{\otimes m}$, we obtain:
$$0\to \omega_{S/C}^{\otimes m}(-C_t)\to \omega_{S/C}^{\otimes m}\to i_*\mathcal{O}_{C_t}\otimes\omega_{S/C}^{\otimes m}\to 0.$$
Applying the functor $f_*$ to the sequence, we get a map
$$f_*\omega_{S/C}^{\otimes m}\to f_*(i_*\mathcal{O}_{C_t}\otimes \omega_{S/C}^{\otimes m})\to R^1f_*(\omega_{S/C}^{\otimes m}(-C_t)).$$
For $m\gg 0$, using cohomology and base change we obtain $R^1f_*(\omega_{S/C}^{\otimes m}(-C_t))=0$, so we have a surjection:
$$f_*\omega_{S/C}^{\otimes m}\to f_*(i^*\mathcal{O}_{C_t}\otimes \omega_{S/C}^{\otimes m})\to 0.$$
Applying the functor $g^*$ to the above sequence, we obtain:
$$g^*f_*\omega_{S/C}^{\otimes m}\to g^*f_*(i^*\mathcal{O}_{C_t}\otimes \omega_{S/C}^{\otimes m})\to 0.$$
We now claim that the term on the right is $i^*\omega_{S/C}^{\otimes m}$, since
\begin{align}
i^*\omega_{S/C}^{\otimes m}& =g^*g_*(i^*\omega_{S/C}^{\otimes m})\nonumber\\
& =g^*f_*i_*(i^*\omega_{S/C}^{\otimes m})\nonumber\\
&=g^*f_*(i^*\mathcal{O}_{C_t}\otimes \omega_{S/C}^{\otimes m}).\nonumber
\end{align}
This way we obtain a contradiction, since the quotient $g^*f_*\omega_{S/C}^{\otimes m}\to i^*\omega_{S/C}^{\otimes m}\to 0$ contradicts nefness.
\end{proof}

We make the following definition, following~\cite[Definition 4.1(i)]{ko90}.

\begin{definition}
\label{definition-semismooth}
Let $k$ be a field.
Let $X$ be an algebraic variety over $k$.
We say $X$ is \textit{semismooth} if all of its closed points are analytically
isomorphic to one of the following:
\begin{enumerate}
\item a smooth point;
\item a double crossing point $\{x_1x_2 = 0\} \subset \mathbf{A}^n$; or
\item a pinch point $\{x_1^2 - x_2^2x_3 = 0\} \subset \mathbf{A}^n$.
\end{enumerate}
In this case the singular locus is smooth, and we call it the {\it double
divisor} of $X$.
\end{definition}

\begin{situation}
\label{situation-kollar-theorem-4.3}
Let $k$ be a field.
Let $S$ be a complete Gorenstein reduced $k$-scheme of dimension $2$ that is
semismooth.
Let $C$ be a complete integral smooth $k$-scheme of dimension $1$.
Let $f : S \to C$ be a surjective map onto $C$, such that the general fiber of
$f$ has only nodes as singularities.
\end{situation}

\begin{lemma}
\label{lemma-base-change-okay}
Consider Situation \ref{situation-kollar-theorem-4.3}.
Let $C'$ be a complete integral smooth $k$-scheme of dimension $1$, and
consider the cartesian square
$$
\xymatrix{
S' \ar[d]_{f'}\ar[r]^{g'} & S\ar[d]^f\\
C' \ar[r]^g & C
}
$$
where $g: C' \to C$ is surjective.
If for some $m$, the sheaf $f'_*(\omega_{S'/C'}^{\otimes m})$ is nef,
then $f_*(\omega_{S/C}^{\otimes m})$ is nef.
\end{lemma}
\begin{proof}
Suppose $f_*(\omega_{S/C}^{\otimes m})$ is not nef, i.e., there
exists a quotient
$$f_*(\omega_{S/C}^{\otimes m}) \longrightarrow \mathcal{L}$$
where $\mathcal{L}$ is an invertible sheaf of negative degree.
Pulling back to $C'$, we then obtain a quotient
$$g^*f_*(\omega_{S/C}^{\otimes m}) \longrightarrow g^*\mathcal{L}$$
of $g^*f_*(\omega_{S/C}^{\otimes m})$ that has negative degree on $C'$.
Since $f$ is flat \cite[\href{http://stacks.math.columbia.edu/tag/00R4}{Tag
00R4}]{stacks-project}, we can apply flat base change
\cite[\href{http://stacks.math.columbia.edu/tag/02KH}{Tag 02KH}]{stacks-project}
to obtain
$$
g^*f_*(\omega_{S/C}^{\otimes m})
\simeq f'_*g'^*(\omega_{S/C}^{\otimes m})
\simeq f'_*(\omega_{S'/C'}^{\otimes m})
$$
where the second isomorphism is
by the compatibility of the relative dualizing sheaf with pullbacks
\cite[\href{http://stacks.math.columbia.edu/tag/0E4P}{Tag
0E4P}]{stacks-project}.
We therefore obtain a negative quotient $g^*\mathcal{L}$ of
$f'_*(\omega_{S'/C'}^{\otimes m})$, which contradicts the assumption that
$f'_*(\omega_{S'/C'}^{\otimes m})$ was nef.
\end{proof}

Compare the following with~\cite[Theorem 4.3]{ko90}.

\begin{theorem}
\label{theorem-rel-dualizing-nef}
In Situation \ref{situation-kollar-theorem-4.3}, suppose moreover that the
generic fiber of $f$ is a stable curve.
Then the sheaf $f_*(\omega_{S/C}^{\otimes m})$ is nef for $m \geq 2$.
\end{theorem}
\begin{proof}
We first claim that we may assume that every double curve in $S$ dominates $C$.
Suppose that a double curve $D$ maps to a point in $C$.
Blowing up $D$ gives a map $b :
\widetilde{S} \to S$.
Suppose we know the sheaf $(b \circ f)_*(\omega_{\widetilde{S}/C}^{\otimes m})$
is nef.
The Grothendieck trace morphism
$$
\mathrm{Tr}_b : b_*(\omega_{\widetilde{S}/C}^{\otimes m})
\longrightarrow \omega_{S/C}^{\otimes m}
$$
for the blowup is injective, since $b$ is birational, and is an isomorphism away
from $D$ \cite[Prop.\ 5.77]{km}.
We therefore have a short exact sequence
$$
0 \longrightarrow b_*(\omega_{\widetilde{S}/C}^{\otimes m})
\longrightarrow \omega_{S/C}^{\otimes m} \longrightarrow \mathcal{Q}
\longrightarrow 0,
$$
where $\mathcal{Q}$ is supported in $D$.
Pushing this forward $\mathrm{Tr}_b$ to the curve $C$, we then have the
left-exact sequence
$$
0 \longrightarrow (f \circ b)_*(\omega_{\widetilde{S}/C}^{\otimes m})
\longrightarrow f_*(\omega_{S/C}^{\otimes m}) \longrightarrow f_*\mathcal{Q}.
$$
Since $D$ is contained in a fiber over $C$, the sheaf $f_*\mathcal{Q}$ is a
skyscraper sheaf supported at a point on $C$.
We can therefore replace $f_*\mathcal{Q}$ with the subsheaf $\mathcal{Q}'$
making the sequence
$$
0 \longrightarrow (f \circ b)_*(\omega_{\widetilde{S}/C}^{\otimes m})
\longrightarrow f_*(\omega_{S/C}^{\otimes m}) \longrightarrow \mathcal{Q}'
\longrightarrow 0
$$
exact.
Since $\mathcal{Q}'$ is a skyscraper sheaf, it has positive degree on $C$.
Thus, assuming $(f \circ b)_*(\omega_{\widetilde{S}/C}^{\otimes m})$ is
nef, we have that $f_*(\omega_{S/C}^{\otimes m})$ is also.
After repeating finitely many blowups of double curves not dominating $C$,
all remaining double curves will dominate $C$.

We now claim that we may assume $S$ has no pinch points. If there are pinch
points, then blowing them up gives a new surface birational to $S$ that only has
double crossing points; the argument of the previous paragraph shows we may
reduce to this case, since pinch points are isolated (as in, there cannot be a
curve that consists wholely of pinch points).

We now claim that we may assume that every double curve $D$ in $S$ is a
section of $f : S \to C$.
If it is not, then consider the base change by $g : D \to C$:
$$
\xymatrix{
S \times_C D \ar[r]^{g'}\ar[d]_{f'} & S \ar[d]^f\\
D \ar[r]^g & C
}
$$
It suffices to show the statement of the Theorem for the morphism $f'$ by Lemma
\ref{lemma-base-change-okay}, and the inverse image of $D$ in $S \times_C D$ is
now a section of $f'$ by using the cartesianness of the square:
$$
\xymatrix{
D\ar@{=}@/_/[ddr]\ar@{^{(}->}@/^/[drr]\ar@{-->}[dr]\\
& S \times_C D \ar[r]^{g'}\ar[d]_{f'} & S \ar[d]^f\\
& D \ar[r]^g & C
}
$$
Repeating finitely many base changes, we can ensure that every double curve is a
section of $f : S \to C$.

We now claim that if $\nu : S' \to S$ is the normalization of $S$, then the
preimage $D'$ of the double curve $D$ still consists of sections.
Since the double curve only consists of double crossing points, the
normalization (analyticall) locally splits up the double curve into two separate
curves on two different irreducible components.
These two curves will still be sections.

We have therefore reduced the statement to the situation of Lemma
\ref{kollar-theorem-4.9}.
%We are now at the top of p. 255
\end{proof}

\section{Nefness after twisting by sections}

\begin{situation}\label{kollar_prop_4.7}
Let $k$ be a field.
Let $C$ be a connected, proper smooth $k$-scheme of dimension 1.
Let $S$ be a proper smooth $k$-scheme of dimension 2.
Let $f:S\to C$ be a family of stable nodal curves whose fibers have arithmetic genus $g$.
Let $C_1,\ldots,C_n$ be a set of pairwise distinct sections of $f$.
Let $m\ge2$ be an integer.
Let $a_1,\ldots,a_n$ be non-negative integers such that $a_i\le m$ for each $i$.
\end{situation}

\begin{lemma}\label{base_case_genus_0}
Suppose we are in Situation \ref{kollar_prop_4.7} where $g=0$ and the sections $C_i$ are pairwise disjoint.
Assume that $\sum a_i\le 2k-1$.
Then $f_{*}(\omega_{S/C}^{\otimes m}(\sum a_iC_i))=0$.
\end{lemma}
\begin{proof}
Immediate from the fact that the sheaf $\omega_{S/C}^{\otimes m}(\sum a_iC_i)$ has negative degree on each fiber of $f$.
\end{proof}

\begin{lemma}\label{base_case_genus_1}
Suppose we are in Situation \ref{kollar_prop_4.7} where $g=1$ and the sections $C_i$ are pairwise disjoint.
Then $f_{*}(\omega_{S/C}^{\otimes m})$ is a nef locally free sheaf.
\end{lemma}
\begin{proof}
By Bombieri-Mumford \todo{add reference}, $\omega_{S/C}$ is nef.
By Grauert's Theorem, $f_{*}\omega_{S/C}$ is locally free.
Moreover, we can check fiberwise that $f^{*}f_{*}\omega_{S/C}\to\omega_{S/C}$ is a surjection, and hence and isomorphism of invertible sheaves.
In particular, $f_{*}\omega_{S/C}$ must be an invertible sheaf of non-negative degree on $C$, hence nef.
Details omitted.
\end{proof}

\begin{lemma}\label{inductive_step_genus_0}
Suppose we are in Situation \ref{kollar_prop_4.7} where $g=0$ and the sections $C_i$ are pairwise disjoint.
Then $f_{*}(\omega_{S/C}^{\otimes m}(\sum a_iC_i))$ is a nef locally free sheaf.
\end{lemma}
\begin{proof}
This proof is nearly identical to that in genus 1 and genus at least 2.

We proceed by induction on $\sum a_i$.
By the Hodge Index Theorem, we may assume that among $C_1,\ldots,C_n$, $C_1$ is the only section with positive self-intersection.
The base cases where $a_i\le 2m-1$ is Lemma \ref{base_case_genus_0}.

Assume the claim is proven for $D_{j-1}=\sum a_iC_i$ where $\sum a_i\ge 2m-1$; we will prove it for $D_{j}=D_{j-1}+C_t$.
Because $a_{1}\le m$, we may assume $C_1\neq C_t$.
By the adjunction formula, $\omega_{S/C}(C_t)|_{C_t}\cong\mathcal{O}_{C_t}$, so $\omega_{S/C}\cdot C_t=-C_t^2\ge0$, as $C_t\neq C_1$.

Consider the exact sequence
\begin{equation*}\label{exact_seq_of_section_twisted_genus_0}
0\to\omega_{S/C}^{\otimes m}(D_{j-1})\to\omega_{S/C}^{\otimes m}(D_j)\to \omega_{S/C}^{\otimes m}(D_j)|_{C_t}\to0
\end{equation*}
obtained by tensoring the closed subscheme exact sequence for $C_t$ with $\omega_{S/C}^{\otimes m}(D_j)$.

Then, $\omega_{S/C}^{\otimes m}(D_j)|_{C_t}\cong \omega_{S/C}^{\otimes (k-a_{t}-1)}|_{C_t}$, because the $C_i$ are pairwise disjoint.
Because $a_{t}+1\le k$ by assumption, this invertible sheaf has non-negative degree.

We now claim that $R^{1}f_{*}\omega_{S/C}^{\otimes m}(D_{j-1}))=0$.
Indeed, by Cohomology and Base Change and the fact that $f$ has relative dimension 1, it suffices to note that $H^{1}(S_x,\omega_{S_x/\kappa(x)}^{\otimes m}(D_{j-1}\cdot S_x))=0$ for each fiber $S_x$ of $f$, which is clear by Serre Duality and degree considerations.

Thus, applying $f_{*}$ to (\ref{exact_seq_of_section_twisted_genus_0})
expresses $f_{*}(\omega_{S/C}^{\otimes m}(\sum a_iC_i))$ as an extension of the
positive degree invertible sheaf
$f_{*}\omega_{S/C}^{\otimes (k-a_{t}-1)}|_{C_t}$ by the nef locally free sheaf
$f_{*}\omega_{S/C}^{\otimes m}(D_{j-1})$, which is nef by Lemma~\ref{lemma-nef-ext-nef}.
Here we have used the fact that $C_t$ is a section, hence $f_{*}\omega_{S/C}^{\otimes (k-a_{t}-1)}|_{C_t}$ is an invertible sheaf on $C$ of the same degree as that of $\omega_{S/C}^{\otimes (k-a_{t}-1)}|_{C_t}$ on $C_t$.
\end{proof}

\begin{lemma}\label{inductive_step_genus_1}
Suppose we are in Situation \ref{kollar_prop_4.7} where $g=1$ and the $C_i$ are pairwise disjoint.
Then $f_{*}(\omega_{S/C}^{\otimes m}(\sum a_iC_i))$ is a nef locally free sheaf.
\end{lemma}
\begin{proof}
This proof is nearly identical to that in genus 0 and genus at least 2.

We proceed by induction on $\sum a_i$.
The base case where all of the $a_i$ are equal to zero is Lemma \ref{base_case_genus_1}.

Assume the claim is proven for $D_{j-1}=\sum a_iC_i$; we will prove it for
$D_{j}=D_{j-1}+C_t$.
By Bombieri-Mumford\todo{add reference} $\omega_{S/C}$ is nef so in particular,
$\omega_{S/C}\cdot C_t\ge0$.

Consider the exact sequence
\begin{equation}\label{exact_seq_of_section_twisted_genus_1}
0\to\omega_{S/C}^{\otimes m}(D_{j-1})\to\omega_{S/C}^{\otimes m}(D_j)\to \omega_{S/C}^{\otimes m}(D_j)|_{C_t}\to0
\end{equation}
obtained by tensoring the closed subscheme exact sequence for $C_t$ with $\omega_{S/C}^{\otimes m}(D_j)$.

Then, $\omega_{S/C}^{\otimes m}(D_j)|_{C_t}\cong \omega_{S/C}^{\otimes (k-a_{t}-1)}|_{C_t}$, because the $C_i$ are pairwise disjoint.
Because $a_{t}+1\le k$ by assumption, this invertible sheaf has non-negative degree.

We now claim that $R^{1}f_{*}\omega_{S/C}^{\otimes m}(D_{j-1}))=0$, unless $D_{j-1}=0$.
Indeed, by Cohomology and Base Change and the fact that $f$ has relative dimension 1, it suffices to note that $H^{1}(S_x,\omega_{S_x/\kappa(x)}^{\otimes m}(D_{j-1}\cdot S_x))=0$ for each fiber $S_x$ of $f$, which is clear by Serre Duality and degree considerations.
On the other hand, when $D_{j-1}=0$, Cohomology and Base Change shows that $R^{1}f_{*}\omega_{S/C}^{\otimes m}(D_{j-1}))$ is an invertible sheaf.


Thus, if $D_{j-1}\neq0$, applying $f_{*}$ to
(\ref{exact_seq_of_section_twisted_genus_1}) expresses
$f_{*}(\omega_{S/C}^{\otimes m}(\sum a_iC_i))$ as an extension of the positive
degree invertible sheaf $f_{*}\omega_{S/C}^{\otimes (k-a_{t}-1)}|_{C_t}$ by the
nef locally free sheaf $f_*\omega_{S/C}^{\otimes m}(D_{j-1})$, which is nef
by Lemma~\ref{lemma-nef-ext-nec}
Here we have used the fact that $C_t$ is a section, hence
$f_{*}\omega_{S/C}^{\otimes (k-a_{t}-1)}|_{C_t}$ is an invertible sheaf on $C$
of the same degree as that of $\omega_{S/C}^{\otimes (k-a_{t}-1)}|_{C_t}$ on
$C_t$.

If $D_{j-1}=0$, the situation is similar; we obtain the exact sequence
$$
0 \to f_{*}(\omega_{S/C}^{\otimes m})\to f_{*}((\omega_{S/C}^{\otimes m})(C_t))\to f_{*}((\omega_{S/C}^{\otimes m})(C_t))|_{C_t}\to R^{1}f_{*}(\omega_{S/C}^{\otimes m})\to0
$$
because $R^{1}f_{*}\omega_{S/C}^{\otimes m}(C_t)=0$.
The rightmost map is a surjection of invertible sheaves, hence an isomorphism.
Therefore the leftmost map is an isomorphism as well, and we obtain the desired conclusion once again.
\end{proof}



\begin{lemma}\label{inductive_step_genus_2}
Suppose we are in Situation \ref{kollar_prop_4.7} where $g\ge2$ and the sections $C_i$ are pairwise disjoint.
Then $f_{*}(\omega_{S/C}^{\otimes m}(\sum a_iC_i))$ is a nef locally free sheaf.
\end{lemma}
\begin{proof}
This proof is nearly identical to that in genus 0 and genus 1.

We proceed by induction on $\sum a_i$.
The base case where all of the $a_i$ are equal to zero is Lemma \todo{insert ref to kollar 4.5}.

Assume the claim is proven for $D_{j-1}=\sum a_iC_i$; we will prove it for $D_{j}=D_{j-1}+C_t$.
By \todo{insert ref to our version of kollar 4.6}, $\omega_{S/C}\cdot C_t\ge0$.

Consider the exact sequence
\begin{equation*}\label{exact_seq_of_section_twisted_genus_2}
0\to\omega_{S/C}^{\otimes m}(D_{j-1})\to\omega_{S/C}^{\otimes m}(D_j)\to \omega_{S/C}^{\otimes m}(D_j)|_{C_t}\to0
\end{equation*}
obtained by tensoring the closed subscheme exact sequence for $C_t$ with $\omega_{S/C}^{\otimes m}(D_j)$.

Then, $\omega_{S/C}^{\otimes m}(D_j)|_{C_t}\cong \omega_{S/C}^{\otimes (k-a_{t}-1)}|_{C_t}$, because the $C_i$ are pairwise disjoint.
Because $a_{t}+1\le k$ by assumption, this invertible sheaf has non-negative degree.

We now claim that $R^{1}f_{*}\omega_{S/C}^{\otimes m}(D_{j-1}))=0$.
Indeed, by Cohomology and Base Change and the fact that $f$ has relative dimension 1, it suffices to note that $H^{1}(S_x,\omega_{S_x/\kappa(x)}^{\otimes m}(D_{j-1}\cdot S_x))=0$ for each fiber $S_x$ of $f$, which is clear by Serre Duality and degree considerations.

Thus, applying $f_{*}$ to (\ref{exact_seq_of_section_twisted_genus_2})
expresses $f_{*}(\omega_{S/C}^{\otimes m}(\sum a_iC_i))$ as an extension of the
positive degree invertible sheaf
$f_*\omega_{S/C}^{\otimes (k-a_{t}-1)}|_{C_t}$ by the nef locally free sheaf
$f_*(\omega_{S/C}^{\otimes m}(D_{j-1})$, which is nef by
Lemma~\ref{lemma-nef-ext-nef}.
Here we have used the fact that $C_t$ is a section, hence $f_{*}\omega_{S/C}^{\otimes (k-a_{t}-1)}|_{C_t}$ is an invertible sheaf on $C$ of the same degree as that of $\omega_{S/C}^{\otimes (k-a_{t}-1)}|_{C_t}$ on $C_t$.
\end{proof}

\begin{lemma}
Suppose we are in Situation \ref{kollar_prop_4.7}.
Then $f_{*}\left(\omega_{S/C}^{\otimes m}\left(\sum a_iC_i\right)\right)$ is nef.
\end{lemma}
\begin{proof}
Let $b:\widetilde{S}/\to S$ be a blowup morphism such that the proper transforms $C'_i$ of the $C_i$ are pairwise disjoint, and are sections of $f\circ b$.
By Lemma \ref{inductive_step_genus_0}, Lemma \ref{inductive_step_genus_1}, and Lemma \ref{inductive_step_genus_2}, $(f\circ b)_{*}\omega_{\widetilde{S}/C}^{\otimes m}$ is nef.

The Grothendieck trace morphism
$$
\mathrm{Tr}_b : b_*(\omega_{\widetilde{S}/C}^{\otimes m})
\longrightarrow \omega_{S/C}^{\otimes m}
$$
for the blowup is injective, since $b$ is birational, and is an isomorphism away
from $D$ \cite[Prop.\ 5.77]{km}.
Tensoring with $\mathcal{O}_{S}\left(\sum a_iC_i\right)$ and applying the
projection formula yields an injection
$$
\mathrm{Tr}'_b : b_*\left(\omega_{\widetilde{S}/C}^{\otimes m}\otimes\mathcal{O}_{\widetilde{S}}\left(\sum a_ib^{*}C_i\right)\right)
\longrightarrow \omega_{S/C}^{\otimes m}\left(\sum a_iC_i\right).
$$
Therefore, we have an injection
$$
\mathrm{Tr}'':(f\circ b)_{*}\left(\omega_{\widetilde{S}/C}^{\otimes m}\otimes\mathcal{O}_{\widetilde{S}}\left(\sum a_iC'_i\right)\right)
\longrightarrow f_{*}\left(\omega_{S/C}^{\otimes m}\left(\sum a_iC_i\right)\right)
$$
after precomposing with the inclusions of $C'_i$ into $b^*C_i$ and applying $f_*$.

Because $b$ is a blowup of points on $S$, $\mathrm{Tr}''$ is generically an isomorphism.
We thus conclude that an invertible quotient of $\mathcal{L}$ of $f_{*}\left(\omega_{S/C}^{\otimes m}\left(\sum a_iC_i\right)\right)$ has a subsheaf that is a quotient of $(f\circ b)_{*}\left(\omega_{\widetilde{S}/C}^{\otimes m}\otimes\mathcal{O}_{\widetilde{S}}\left(\sum a_iC'_i\right)\right)$, and thus must have non-negative degree by the nefness of $(f\circ b)_{*}\left(\omega_{\widetilde{S}/C}^{\otimes m}\otimes\mathcal{O}_{\widetilde{S}}\left(\sum a_iC'_i\right)\right)$.
This completes the proof.
\end{proof}

Compare with~\cite[Theorem 4.9]{ko90}.

\begin{lemma}
\label{kollar-theorem-4.9}
Let $k$ be a field.
Let $C$ be a proper connected smooth $k$-scheme of dimension $1$.
Let $S$ be a semismooth surface.
Let $f:S\to C$ be a surjective map whose general fiber is a stable curve.
Let $D$ be a double curve on $S$.
Assume $D$ consists of section of $f:S\to C$.
Let $g:S'\to S$ be the normalization of $S$.
Let $D'=g^{-1}(D)$.
Assume each component of $D'$ is a section.
Let $m\geq 2$ any integer.
Then $f_*(\omega_{S/C}^{\otimes m})$ is nef.
\end{lemma}
\begin{proof}
By \ref{kollar_prop_4.7}\todo{This appears to be the wrong reference?}
we know that $(f\circ g)_*(\omega_{S'/C}^{\otimes m})$ is nef.
Observe that $g^*\omega_{S/C}\cong \omega_{S'/C}(D')$.
Now we consider the short exact sequence;
$$
0 \to \mathcal{O}_{S}(-D) \to \mathcal{O}_S \to \mathcal{O}_D \to 0.
$$
Tensoring the sequence with $\omega_{S/C}$ gives
$$
0 \to g_*(\omega_{S'/C}) \to \omega_{S/C} \to \omega_{S/C}\rvert_D \cong \mathcal{O}_D \to 0.
$$
Now we tensor the sequence with $\omega_{S/C}^{\otimes (m-1)}$ and apply the
projection formula, obtaining:
$$
0 \to g_*\omega_{S'/C}^{\otimes m}((m-1)D') \to \omega_{S/C}^{\otimes m}\to \mathcal{O}_D\to 0.
$$
Now we apply the functor $f_*$ to the sequence and use that
$R^1(f\circ g)_*\omega_{S'/C}^{\otimes m}((m-1)D')=0$ for $m\geq 2$,
obtaining the following sequence:
$$
  0 \to
  (f\circ g)_*\omega_{S'/C}^{\otimes m}((m-1)D') \to
  f_*\omega_{S/C}^{\otimes m} \to
  f_*\mathcal{O}_D \to
  0.
$$
The sheaf $f_*\mathcal{O}_D$ is isomorphic to the sum of copies of
$\mathcal{O}_C$, therefore $f_*\omega_{S/C}^{\otimes m}$ is nef for
all $k\geq 2$.
\end{proof}

\section{Projectivity of $\overline{\mathcal{M}_g}$}
In this section, we show that the moduli stack $\overline{\mathcal{M}}_g$ of
stable curves of genus $g \geq 2$ is projective over $\Spec(\mathbf{Z})$ in the
sense that it admits a coarse moduli space which is projective over
$\Spec(\mathbf{Z})$.
This is done by, first, specializing the ampleness criterion
of~\cite[Theorem 2.6]{ko90} to the case at hand in
Lemma~\ref{lemma-mgbar-proj-field} to show that $\overline{M}_g$ is projective
over any field $k$.
This is then used to show $\overline{M}_g$ is projective over $\mathbf{Z}$
in Theorem~\ref{theorem-mgbar-proj}.

\begin{lemma}
\label{lemma-coarse-mgbar}
The stacks $\overline{\mathcal{M}}$ and $\overline{\mathcal{M}}_g$
defined in~\citeSP{0E77} admit uniform categorical moduli spaces
$f : \overline{\mathcal{M}} \to \overline{M}$ and
$f_g : \overline{\mathcal{M}}_g \to \overline{M}_g$ such that $f$ and $f_g$
are separated, quasi-compact, and universal homeomorphisms.
\end{lemma}

\begin{proof}
The stacks of stable curves have finite inertia by~\citeSP{0E7A}
and~\citeSP{0DSW}.
The existence of $f$ and $f_g$ now follows from by~\citeSP{0DUT}.
\end{proof}

\begin{definition}
\label{definition-deligne-mumford-coarse}
The algebraic space $\overline{M}$ from Lemma~\ref{lemma-coarse-mgbar} is
called the \emph{coarse moduli space of curves}.
Similarly, for $g \geq 2$, the algebraic space $\overline{M}_g$ is called the
\emph{coarse moduli space of curves of genus $g$}.
\end{definition}

\begin{lemma}
\label{lemma-mgbar-hodge-bundle}
Let $\overline{\mathcal{M}}_g$ be the stack of stable curves of genus $g \geq
2$ over $\Spec(\mathbf{Z})$.
Then for each $m \geq 2$, there exists an invertible sheaf $\lambda_m$ on
$\overline{\mathcal{M}}_g$ such that for every family of curves $f : X \to S$
with classifying map $g : S \to \overline{\mathcal{M}_g}$,
$g^*\lambda_m = \det(f_*\omega_{X/S})$.
\end{lemma}

\begin{proof}
...\todo{Prove this at some point. Perhaps~\cite[\S13.2]{acmg} will be useful.
  Note that I really need for this to be integral.}
\end{proof}

\begin{lemma}
\label{lemma-family-with-all-curves}
Let $k$ be a field.
Let $g \geq 2$ and let $\overline{M}_g$ be the coarse moduli space of stable
curves of genus $g$.
Then there exists a family of stable curves $f : X \to S$ such that
$S$ a finite type scheme over $k$ equipped with a finite surjective map
$p : S \to \overline{M}_g$ such that for every
$s \in S(\overline k)$, the point $p(s) \in \overline{M}_g(\overline k)$ is
the moduli point corresponding to the stable curve
$f_s : X \times_S \Spec(\overline k) \to \Spec(\kappa(s))$.\todo{State this
better.}
\end{lemma}

\begin{proof}
  ...\todo{Prove this. Compare with~\cite[Proposition 2.7]{ko90}.}
\end{proof}

\begin{lemma}
\label{lemma-ample-determinant-of-rel-canonical}
Let $k$ be a field.
Let $g \geq 2$ and let $f : X \to S$ be a family of stable curves of genus $g$
such that the classifying map $S \to \overline{\mathcal{M}}_g$ is finite.
Then $\det(f_*\omega_{X/S})$ is ample on $S$.
\end{lemma}

\begin{proof}
We will apply Lemma~\ref{lemma-ampleness-of-semipos-quotients} to the
multiplication map
$$
\mu : \mathrm{Sym}^d(f_*(\omega_{X/S}^{\otimes m})) \to f_*(\omega_{X/S}^{\otimes md}),
$$
for some choice of $m$ and $d$.
Specifically, choose positive integers $m$ and $d$ such that the following
hold:
\begin{enumerate}
\item $\omega_{X/S}^{\otimes m}$ is $f$-very ample,
\item $R^if_*(\omega_{X/S}^{\otimes m}) = 0$ for all $i > 0$ and the
  map $f_*(\omega_{X/S}^{\otimes m})_s \to H^0(X_s,\omega_{X/\kappa(s)})$ is
  an isomorphism for every $s \in S$,
\item the multiplication map
  $\mathrm{Sym}^d(f_*\omega_{X/S}^{\otimes m}) \to f_*\omega_{X/S}^{\otimes md}$
  is surjective, and
\item for every $s \in S$, the fibre $X_s = X \times_S \Spec(\kappa(s))$
  embedded via $\omega_{X_s/\kappa(s)}^{\otimes m}$ is defined by equations
  of degree at most $d$\todo{Make this hypothesis cleaner.}.
\end{enumerate}
Item (1) can be arranged for by the fact~\cite[Corollary of (1.2)]{dm} that
$\omega_{X/S}$ is $f$-ample together with~\citeSP{01VU};
(2) can be arranged, by~\citeSP{0D2M} and quasi-compactness of $S$, after
possibly taking a larger $m$;
(3) can be arranged by the arguments in the proof of~\citeSP{0C6T}; and
(4) can be arranged since pluricanonically embedded curves are defined
by quadratic equations\todo{Check this and find a reference}.

Now we set up our application of Lemma~\ref{lemma-ampleness-of-semipos-quotients}
to the map $\mu$.
Note that, by (3), $\mu$ is a quotient map.
Moreover, by Lemma~\ref{theorem-rel-dualizing-nef},
$f_*\omega_{X/S}^{\otimes md}$ is nef.
It remains to check that the classifying map
$$
u : S \to [\mathbf{G}(q,N)/\mathrm{PGL}_n]
$$
is quasi-finite\todo{I want the fibres of this map to be finite; is this really
  the same as demanding that this is quasi-finite as a morphism of stacks?},
where $n = 2m(g-1)$, $N = \binom{2m(g - 1) + d}{d}$ and $q = 2md(g - 1)$.
It suffices to check that the fibre $S_x$ of $u$ is a finite set for every
geometric point
$x : \Spec(\overline k) \to [\mathbf{G}(q,N)/\mathrm{PGL}_n]$.
By~\citeSP{04UV}, $x$ corresponds to a diagram
$$
\xymatrix{
  P \ar[d] \ar[r]_\varphi & \mathbf{G}(q,N) \ar[d] \\
  \Spec(\overline k) \ar[r] & \Spec(k)
}
$$
where $P \to \Spec(\overline k)$ is a $\mathrm{PGL}_n$-torsor,
$\varphi : P \to \mathbf{G}(q,N)$ is a $\mathrm{PGL}_n$-equivariant morphism of
$\Spec(k)$-schemes.
The morphism $\varphi$ to the Grassmannian\todo{Grassmannians in the Stacks project
parameterize subs, so need to change $q$ to $N-q$ is most places...},
by construction~\citeSP{089U}, corresponds to an isomorphism class of
surjections
$$
\mu_x : \mathcal{O}_P^{\oplus N} \to \mathcal{Q}
$$
where $\mathcal{Q}$ is a locally free $\mathcal{O}_P$-module of rank $q$.
Thus, by construction of the classifying map from
Lemma~\ref{lemma-grassman-quotient-stack-map},
the points $s : \Spec(\overline k) \to S$ in the fibre $S_x$ are those
where the multiplication map
$$
\mu_s :
  \mathrm{Sym}^d(f_*(\omega_{X/S}^{\otimes m}))_s \to
  f_*(\omega_{X/S}^{\otimes md})_s
$$
is isomorphic to $\mu_x$.
By assumption (2), $\mu_s$ can be identified as the multiplication map
$$
\mu_s :
  \mathrm{Sym}^d(H^0(X_s,\omega_{X_s/\overline{k}}^{\otimes m})) \to
  H^0(X_s,\omega_{X_s/\overline{k}}^{\otimes md}).
$$
Now, up to the action of $\mathrm{PGL}_n$, we may identify
$$
\mathrm{Sym}^d(H^0(X_s,\omega_{X_s/\overline k}^{\otimes m}))
  = \mathrm{Sym}^d(H^0(\mathbf{P}_{\overline k}^{n - 1},
      \mathcal{O}_{\mathrm{P}^{n-1}_{\overline{k}}}(1)))
$$
via the embedding of $X_s$ by $\omega_{X_s/\overline k}^{\otimes m}$.
Thus the isomorphism class of $\mu_s$ is determined by the subspace of
degree $d$ equations defining $X_s$ in $\mathbf{P}_{\overline k}$.
By assumption (4), each fibre $X_s$ is determined up to isomorphism
by such equations.
Thus all members of the family $X \times_S S_x \to S_x$ are isomorphic.
It remains to remark that $S \to \overline{\mathcal{M}}_g$ is finite and
hence, for every stable curve $C$, there are at most finitely many $s \in S$
such that $X_s$ is isomorphic to $C$.
Thus $S_x$ is finite, completing the proof that $u$ is quasi-finite.
\end{proof}

\begin{lemma}
\label{lemma-mgbar-proj-field}
Let $k$ be a field.
Let $\overline{\mathcal{M}}_g$ be the stack of stable curves of genus $g \geq 2$
over $k$ and let $\overline{M}_g$ be its coarse moduli space.
Then $\overline{M}_g$ is projective over $k$.
\end{lemma}

\begin{proof}
We show that the invertible sheaf $\lambda_m$ on $\overline{\mathcal{M}}_g$
obtained by base change of the invertible sheaf from
Lemma~\ref{lemma-mgbar-hodge-bundle} descends to an ample
invertible sheaf $\lambda_m^{\otimes N}$ on $\overline{M}_g$ for $N$
sufficiently divisible\todo{Make sense of this at some point and find a
  reference for what this means.}.
By Lemma~\ref{lemma-family-with-all-curves}, there exists a family of stable
curves $f : X \to S$ such that $S$ is a finite type $k$-scheme together with
a finite surjective map $p : S \to \overline{M}_g$ such that for every
$s \in S(\overline k)$, $p(s) \in \overline{M}_g(\overline k)$ is
the moduli point corresponding to the stable curve
$f_s : X \times_S \Spec(\overline k) \to \Spec(\kappa(s))$.
Now by Lemma~\ref{lemma-surjective-finite-morphism-ample},
$\lambda_m^{\otimes N}$ is ample on $\overline{M}_g$ if and only if
$p^*(\lambda_m^{\otimes N})$ is ample on $S$.
Let $\lambda_m(f)$ denote the pullback under the classifying map
$S \to \overline{\mathcal{M}}_g$ of the invertible sheaf $\lambda_m$ on the
stack $\overline{\mathcal{M}}_g$.
Then $p^*(\lambda_m^{\otimes N}) = \lambda_m(f)^{\otimes N}$.
Hence by~\citeSP{01PT}, it suffices to show $\lambda_m(f)$ is ample on $S$.
But recall from Lemma~\ref{lemma-mgbar-hodge-bundle} that
$\lambda_m(f) = \det(f_*\omega_{X/S})$.
Ampleness of $\lambda_m(f)$ on $S$ now follows from
Lemma~\ref{lemma-ample-determinant-of-rel-canonical}.
\end{proof}


\begin{theorem}
\label{theorem-mgbar-proj}
Let $\overline{\mathcal{M}_g}$ be the moduli space of stable curves of genus
$g \geq 2$ over $\mathbb{Z}$ and let $\overline{M}_g$ be its coarse moduli
space.
Then $\overline{M}_g$ is projective over $\mathbb{Z}$.
\end{theorem}

\begin{proof}
We show that the invertible sheaf $\lambda_m$, with $m \geq 3$ on
$\overline{M}_g$ constructed in Lemma~\ref{lemma-mgbar-hodge-bundle} is
relatively ample on $\overline{M}_g/\Spec(\mathbf{Z})$.
We already know from Lemma~\ref{lemma-mgbar-proj-field} that the base
change of $\lambda_m$ to $\overline{M}_g/\Spec(k)$ is relatively ample therein.
In particular, this means that $\lambda_m$ restricted to each fibre of
$\overline{M}_g \to \Spec(\mathbf{Z})$ is relatively ample.
Since $\overline{\mathcal{M}}_g \to \Spec(\mathbf{Z})$ is proper
by~\citeSP{0E9C},~\citeSP{0DUZ} shows $\overline{M}_g \to \Spec(\mathbf{Z})$ is
proper.
Hence~\citeSP{0D2N} applies to show $\lambda_m$ is ampld on all of
$\overline{M}_g/\Spec(\mathbf{Z})$, as desired.
\end{proof}

\bibliographystyle{amsplain}
\bibliography{references}
\end{document}

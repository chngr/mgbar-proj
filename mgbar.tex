\IfFileExists{stacks-project.cls}{%
\documentclass{stacks-project}
}{%
\documentclass{amsart}
}

% The following AMS packages are automatically loaded with
% the amsart documentclass:
%\usepackage{amsmath}
%\usepackage{amssymb}
%\usepackage{amsthm}

% For dealing with references we use the comment environment
\usepackage{verbatim}
\newenvironment{reference}{\comment}{\endcomment}
%\newenvironment{reference}{}{}
\newenvironment{slogan}{\comment}{\endcomment}
\newenvironment{history}{\comment}{\endcomment}

% For commutative diagrams you can use
% \usepackage{amscd}
\usepackage[all]{xy}

% We use 2cell for 2-commutative diagrams.
\xyoption{2cell}
\UseAllTwocells

% To put source file link in headers.
% Change "template.tex" to "this_filename.tex"
% \usepackage{fancyhdr}
% \pagestyle{fancy}
% \lhead{}
% \chead{}
% \rhead{Source file: \url{template.tex}}
% \lfoot{}
% \cfoot{\thepage}
% \rfoot{}
% \renewcommand{\headrulewidth}{0pt}
% \renewcommand{\footrulewidth}{0pt}
% \renewcommand{\headheight}{12pt}

\usepackage{multicol}

% For cross-file-references
\usepackage{xr-hyper}

% Package for hypertext links:
\usepackage{hyperref}

% For any local file, say "hello.tex" you want to link to please
% use \externaldocument[hello-]{hello}
\externaldocument[introduction-]{introduction}
\externaldocument[conventions-]{conventions}
\externaldocument[sets-]{sets}
\externaldocument[categories-]{categories}
\externaldocument[topology-]{topology}
\externaldocument[sheaves-]{sheaves}
\externaldocument[sites-]{sites}
\externaldocument[stacks-]{stacks}
\externaldocument[fields-]{fields}
\externaldocument[algebra-]{algebra}
\externaldocument[brauer-]{brauer}
\externaldocument[homology-]{homology}
\externaldocument[derived-]{derived}
\externaldocument[simplicial-]{simplicial}
\externaldocument[more-algebra-]{more-algebra}
\externaldocument[smoothing-]{smoothing}
\externaldocument[modules-]{modules}
\externaldocument[sites-modules-]{sites-modules}
\externaldocument[injectives-]{injectives}
\externaldocument[cohomology-]{cohomology}
\externaldocument[sites-cohomology-]{sites-cohomology}
\externaldocument[dga-]{dga}
\externaldocument[dpa-]{dpa}
\externaldocument[hypercovering-]{hypercovering}
\externaldocument[schemes-]{schemes}
\externaldocument[constructions-]{constructions}
\externaldocument[properties-]{properties}
\externaldocument[morphisms-]{morphisms}
\externaldocument[coherent-]{coherent}
\externaldocument[divisors-]{divisors}
\externaldocument[limits-]{limits}
\externaldocument[varieties-]{varieties}
\externaldocument[topologies-]{topologies}
\externaldocument[descent-]{descent}
\externaldocument[perfect-]{perfect}
\externaldocument[more-morphisms-]{more-morphisms}
\externaldocument[flat-]{flat}
\externaldocument[groupoids-]{groupoids}
\externaldocument[more-groupoids-]{more-groupoids}
\externaldocument[etale-]{etale}
\externaldocument[chow-]{chow}
\externaldocument[intersection-]{intersection}
\externaldocument[pic-]{pic}
\externaldocument[adequate-]{adequate}
\externaldocument[dualizing-]{dualizing}
\externaldocument[duality-]{duality}
\externaldocument[discriminant-]{discriminant}
\externaldocument[local-cohomology-]{local-cohomology}
\externaldocument[curves-]{curves}
\externaldocument[resolve-]{resolve}
\externaldocument[models-]{models}
\externaldocument[pione-]{pione}
\externaldocument[etale-cohomology-]{etale-cohomology}
\externaldocument[proetale-]{proetale}
\externaldocument[crystalline-]{crystalline}
\externaldocument[spaces-]{spaces}
\externaldocument[spaces-properties-]{spaces-properties}
\externaldocument[spaces-morphisms-]{spaces-morphisms}
\externaldocument[decent-spaces-]{decent-spaces}
\externaldocument[spaces-cohomology-]{spaces-cohomology}
\externaldocument[spaces-limits-]{spaces-limits}
\externaldocument[spaces-divisors-]{spaces-divisors}
\externaldocument[spaces-over-fields-]{spaces-over-fields}
\externaldocument[spaces-topologies-]{spaces-topologies}
\externaldocument[spaces-descent-]{spaces-descent}
\externaldocument[spaces-perfect-]{spaces-perfect}
\externaldocument[spaces-more-morphisms-]{spaces-more-morphisms}
\externaldocument[spaces-flat-]{spaces-flat}
\externaldocument[spaces-groupoids-]{spaces-groupoids}
\externaldocument[spaces-more-groupoids-]{spaces-more-groupoids}
\externaldocument[bootstrap-]{bootstrap}
\externaldocument[spaces-pushouts-]{spaces-pushouts}
\externaldocument[groupoids-quotients-]{groupoids-quotients}
\externaldocument[spaces-more-cohomology-]{spaces-more-cohomology}
\externaldocument[spaces-simplicial-]{spaces-simplicial}
\externaldocument[spaces-duality-]{spaces-duality}
\externaldocument[formal-spaces-]{formal-spaces}
\externaldocument[restricted-]{restricted}
\externaldocument[spaces-resolve-]{spaces-resolve}
\externaldocument[formal-defos-]{formal-defos}
\externaldocument[defos-]{defos}
\externaldocument[cotangent-]{cotangent}
\externaldocument[examples-defos-]{examples-defos}
\externaldocument[algebraic-]{algebraic}
\externaldocument[examples-stacks-]{examples-stacks}
\externaldocument[stacks-sheaves-]{stacks-sheaves}
\externaldocument[criteria-]{criteria}
\externaldocument[artin-]{artin}
\externaldocument[quot-]{quot}
\externaldocument[stacks-properties-]{stacks-properties}
\externaldocument[stacks-morphisms-]{stacks-morphisms}
\externaldocument[stacks-limits-]{stacks-limits}
\externaldocument[stacks-cohomology-]{stacks-cohomology}
\externaldocument[stacks-perfect-]{stacks-perfect}
\externaldocument[stacks-introduction-]{stacks-introduction}
\externaldocument[stacks-more-morphisms-]{stacks-more-morphisms}
\externaldocument[stacks-geometry-]{stacks-geometry}
\externaldocument[moduli-]{moduli}
\externaldocument[moduli-curves-]{moduli-curves}
\externaldocument[examples-]{examples}
\externaldocument[exercises-]{exercises}
\externaldocument[guide-]{guide}
\externaldocument[desirables-]{desirables}
\externaldocument[coding-]{coding}
\externaldocument[obsolete-]{obsolete}
\externaldocument[fdl-]{fdl}
\externaldocument[index-]{index}

% Theorem environments.
%
\theoremstyle{plain}
\newtheorem{theorem}[subsection]{Theorem}
\newtheorem{proposition}[subsection]{Proposition}
\newtheorem{lemma}[subsection]{Lemma}

\theoremstyle{definition}
\newtheorem{definition}[subsection]{Definition}
\newtheorem{example}[subsection]{Example}
\newtheorem{exercise}[subsection]{Exercise}
\newtheorem{situation}[subsection]{Situation}

\theoremstyle{remark}
\newtheorem{remark}[subsection]{Remark}
\newtheorem{remarks}[subsection]{Remarks}

\numberwithin{equation}{subsection}

% Macros
%
\def\lim{\mathop{\rm lim}\nolimits}
\def\colim{\mathop{\rm colim}\nolimits}
\def\Spec{\mathop{\rm Spec}}
\def\Hom{\mathop{\rm Hom}\nolimits}
\def\Ext{\mathop{\rm Ext}\nolimits}
\def\SheafHom{\mathop{\mathcal{H}\!{\it om}}\nolimits}
\def\SheafExt{\mathop{\mathcal{E}\!{\it xt}}\nolimits}
\def\Sch{\textit{Sch}}
\def\Mor{\mathop{\rm Mor}\nolimits}
\def\Ob{\mathop{\rm Ob}\nolimits}
\def\Sh{\mathop{\textit{Sh}}\nolimits}
\def\NL{\mathop{N\!L}\nolimits}
\def\proetale{{pro\text{-}\acute{e}tale}}
\def\etale{{\acute{e}tale}}
\def\QCoh{\textit{QCoh}}
\def\Ker{\mathop{\rm Ker}}
\def\Im{\mathop{\rm Im}}
\def\Coker{\mathop{\rm Coker}}
\def\Coim{\mathop{\rm Coim}}

%
% Macros for moduli stacks/spaces
%
\def\QCohstack{\mathcal{QC}\!{\it oh}}
\def\Cohstack{\mathcal{C}\!{\it oh}}
\def\Spacesstack{\mathcal{S}\!{\it paces}}
\def\Quotfunctor{{\rm Quot}}
\def\Hilbfunctor{{\rm Hilb}}
\def\Curvesstack{\mathcal{C}\!{\it urves}}
\def\Polarizedstack{\mathcal{P}\!{\it olarized}}
\def\Complexesstack{\mathcal{C}\!{\it omplexes}}
% \Pic is the operator that assigns to X its picard group, usage \Pic(X)
% \Picardstack_{X/B} denotes the Picard stack of X over B
% \Picardfunctor_{X/B} denotes the Picard functor of X over B
\def\Pic{\mathop{\rm Pic}\nolimits}
\def\Picardstack{\mathcal{P}\!{\it ic}}
\def\Picardfunctor{{\rm Pic}}
\def\Deformationcategory{\mathcal{D}\!{\it ef}}


\newcommand{\todo}[1]{\footnote{\textbf{TODO.} #1}}
\begin{document}
\title{Projectivity of moduli of curves}
\maketitle

\section{Stable curves}
\begin{lemma}[{\cite[\href{http://stacks.math.columbia.edu/tag/0E76}{Tag 0E76}]{stacks-project}.}]
\label{lemma-stable-curves}
There exist an open substack $\Curvesstack^{stable} \subset \Curvesstack$
such that
\begin{enumerate}
\item given a family of curves $f : X \to S$ the following are equivalent
\begin{enumerate}
\item the classifying morphism $S \to \Curvesstack$ factors
through $\Curvesstack^{stable}$,
\item $X \to S$ is a stable family of curves,
\end{enumerate}
\item given $X$ a scheme proper over a field $k$ with
$\dim(X) \leq 1$ the following are equivalent
\begin{enumerate}
\item the classifying morphism $\Spec(k) \to \Curvesstack$
factors through $\Curvesstack^{stable}$,
\item the singularities of $X$ are at-worst-nodal, $\dim(X) = 1$,
$k = H^0(X, \mathcal{O}_X)$, the genus of $X$ is $\geq 2$, and
$X$ has no rational tails or bridges,
\item the singularities of $X$ are at-worst-nodal, $\dim(X) = 1$,
$k = H^0(X, \mathcal{O}_X)$, and $\omega_{X_s}$ is ample.
\end{enumerate}
\end{enumerate}
\end{lemma}

\section{Semipositivity Results}
In this section, we prove a semipositivity result,
Lemma \ref{lemma-stable-curves-semipositive}, for the pushforward of the
relative dualizing sheaf of families of curves.

\begin{lemma}
Let $k$ be a field.
Let $f : S \to C$ be a flat family of curve...\todo{Formulate the hypotheses of
this better; this is Kollar's Proposition 4.5.}
If $k \geq 2$, then $f_*\omega_{S/C}^{\otimes k}$ is semipositive for $k \geq 2$.
\end{lemma}

\begin{lemma}
\label{lemma-stable-curves-semipositive-resolve}
Let $k$ be a field.
Let $C$ be a proper smooth $k$-scheme of dimension $1$.
Let $\pi : S \to C$ be a flat morphism whose fibres are stable nodal and arithmetic
genus $g \geq 2$.
Let $f : S' \to S$ is a resolution of the isolated singularities of $S$
and let $\pi' : S' \to C$ be the composition $\pi' = \pi \circ f$.
If $\pi'_*(\omega_{S'/C}^{\otimes m})$ is semipositive for $m \geq 2$, then
$\pi_*(\omega_{S/C}^{\otimes m})$ is semipositive.
\end{lemma}

\begin{lemma}[cf.\ {\cite[Theorem 4.3]{Ko90}}]
\label{lemma-stable-curves-semipositive}
Let $k$ be a field.
Let $C$ be a proper connected smooth $k$-scheme of dimension $1$.
Let $\pi : S \to C$ be a family of stable curves,
as defined in Lemma \ref{lemma-stable-curves}.
Then $f_*(\omega_{S/C}^{\otimes m})$ is semipositive for $m \geq 2$.
\end{lemma}

\begin{definition}[cf.\ {\cite[Definition 4.1(i)]{Ko90}}]
\label{definition-semismooth}
Let $k$ be a field.
Let $X$ be an algebraic variety over $k$.
We say $X$ is {\it semismooth} if all of its closed points are analytically
isomorphic to one of the following:
\begin{enumerate}
\item a smooth point;
\item a double crossing point $\{x_1x_2 = 0\} \subset \mathbf{A}^n$; or
\item a pinch point $\{x_1^2 - x_2^2x_3 = 0\} \subset \mathbf{A}^n$.
\end{enumerate}
In this case the singular locus is smooth, and we call it the {\it double
divisor} of $X$.
\end{definition}

\begin{situation}
\label{situation-kollar-theorem-4.3}
Let $k$ be a field.
Let $S$ be a complete Gorenstein integral $k$-scheme of dimension $2$ that is
semismooth.
Let $C$ be a complete connected regular $k$-scheme of dimension $1$.
Let $f : S \to C$ be a surjective map onto $C$, such that the general fiber of
$f$ has only nodes as singularities.
\end{situation}

\begin{lemma}
\label{lemma-base-change-okay}
Consider Situation \ref{situation-kollar-theorem-4.3}.
Let $C'$ be a complete connected smooth $k$-scheme of dimension $1$, and
consider the cartesian square
$$
\xymatrix{
S' \ar[d]_{f'}\ar[r]^{g'} & S\ar[d]^f\\
C' \ar[r]^g & C
}
$$
where $g: C' \to C$ is surjective.
If for some $m$, the sheaf $f'_*(\omega_{S'/C'}^{\otimes m})$ is semipositive,
then $f_*(\omega_{S/C}^{\otimes m})$ is semipositive.
\end{lemma}
\begin{proof}
Suppose $f_*(\omega_{S/C}^{\otimes m})$ is not semipositive, i.e., there
exists a quotient
$$f_*(\omega_{S/C}^{\otimes m}) \longrightarrow L$$
where $L$ is an invertible sheaf of negative degree.
Pulling back to $C'$, we then obtain a quotient
$$g^*f_*(\omega_{S/C}^{\otimes m}) \longrightarrow g^*L$$
of $g^*f_*(\omega_{S/C}^{\otimes m})$ that has negative degree on $C'$.
Since $f$ is flat \cite[\href{http://stacks.math.columbia.edu/tag/00R4}{Tag
00R4}]{stacks-project}, we can apply flat base change
\cite[\href{http://stacks.math.columbia.edu/tag/02KH}{Tag 02KH}]{stacks-project}
to obtain
$$
g^*f_*(\omega_{S/C}^{\otimes m})
\simeq f'_*g'^*(\omega_{S/C}^{\otimes m})
\simeq f'_*(\omega_{S'/C'}^{\otimes m})
$$
where the second isomorphism is
by the compatibility of the relative dualizing sheaf with pullbacks
\cite[\href{http://stacks.math.columbia.edu/tag/0E4P}{Tag
0E4P}]{stacks-project}.
We therefore obtain a negative quotient $g^*L$ of $f'_*(\omega_{S'/C'}^{\otimes
m})$, which contradicts the assumption that $f'_*(\omega_{S'/C'}^{\otimes m})$
was semipositive.
\end{proof}

\begin{theorem}[cf.\ {\cite[Theorem 4.3]{Ko90}}]
\label{theorem-kollar-theorem-4.3}
In Situation \ref{situation-kollar-theorem-4.3}, 
the sheaf $f_*(\omega_{S/C}^{\otimes m})$ is semipositive for $m \ge 2$. 
\end{theorem}








\section{Semipositivity after twisting by sections}

\begin{situation}\label{kollar_prop_4.7}
Let $k$ be a field.
Let $C$ be a proper smooth $k$-scheme of dimension 1.
Let $S$ be a proper smooth $k$-scheme of dimension 2.
Let $\pi:S\to C$ be a family of stable nodal curves whose fibers have arithmetic genus $g$.
Let $C_1,\ldots,C_n$ be a set of pairwise distinct sections of $f$.
Let $m\ge2$ be an integer.
Let $a_1,\ldots,a_n$ be non-negative integers such that $a_i\le m$ for each $i$.
\end{situation}

\begin{lemma}\label{base_case_genus_0}
Suppose we are in Situation \ref{kollar_prop_4.7} where $g=0$ and the sections $C_i$ are pairwise disjoint.
Assume that $\sum a_i\le 2k-1$.
Then $f_{*}(\omega_{S/C}^{k}(\sum a_iC_i))=0$. 
\end{lemma}
\begin{proof}
Immediate from the fact that the sheaf $\omega_{S/C}^{k}(\sum a_iC_i)$ has negative degree on each fiber of $f$.
\end{proof}

\begin{lemma}\label{base_case_genus_1}
Suppose we are in Situation \ref{kollar_prop_4.7} where $g=1$ and the sections $C_i$ are pairwise disjoint.
Then $f_{*}(\omega_{S/C}^{k}=0$. 
\end{lemma}
\begin{proof}

\end{proof}

\begin{lemma}\label{inductive_step_genus_0}
Suppose we are in Situation \ref{kollar_prop_4.7} where $g=0$ and the sections $C_i$ are pairwise disjoint.
Then $f_{*}(\omega_{S/C}^{k}(\sum a_iC_i))$ is semipositive.
\end{lemma}
\begin{proof}
This proof is nearly identical to that in genus 1 and genus at least 2.

We proceed by induction on $\sum a_i$. 
The base cases where $a_i\le 2m-1$ is Lemma \ref{base_case_genus_0}.

Assume the claim is proven for $D_{j-1}=\sum a_iC_i$ where $\sum a_i\ge 2m-1$; we will prove it for $D_{j}=D_{j-1}+C_t$.
By \todo{insert ref to our version of kollar 4.6}, $\omega_{S/C}\cdot C_t\ge0$.

Consider the exact sequence
\begin{equation*}\label{exact_seq_of_section_twisted_genus_0}
0\to\omega_{S/C}^{\otimes m}(D_{j-1})\to\omega_{S/C}^{\otimes m}(D_j)\to \omega_{S/C}^{\otimes m}(D_j)|_{C_t}\to0
\end{equation*}
obtained by tensoring the closed subscheme exact sequence for $C_t$ with $\omega_{S/C}^{\otimes m}(D_j)$.

Then, $\omega_{S/C}^{\otimes m}(D_j)|_{C_t}\cong \omega_{S/C}^{\otimes (k-a_{t}-1)}|_{C_t}$, because the $C_i$ are pairwise disjoint.
Because $a_{t}+1\le k$ by assumption, this invertible sheaf has non-negative degree.

We now claim that $R^{1}f_{*}\omega_{S/C}^{\otimes m}(D_{j-1}))=0$.
Indeed, by Cohomology and Base Change and the fact that $f$ has relative dimension 1, it suffices to note that $H^{1}(S_x,\omega_{S_x/\kappa(x)}^{\otimes m}(D_{j-1}\cdot S_x))=0$ for each fiber $S_x$ of $f$, which is clear by Serre Duality and degree considerations.

Thus, applying $f_{*}$ to (\ref{exact_seq_of_section_twisted_genus_0}) expresses $f_{*}(\omega_{S/C}^{k}(\sum a_iC_i))$ as an extension of the positive degree invertible sheaf $f_{*}\omega_{S/C}^{\otimes (k-a_{t}-1)}|_{C_t}$ by the semipositive locally free sheaf $f_{(}\omega_{S/C}^{\otimes m}(D_{j-1})$, which is semipositive by \todo{ref to semipositivity section}.
Here we have used the fact that $C_t$ is a section, hence $f_{*}\omega_{S/C}^{\otimes (k-a_{t}-1)}|_{C_t}$ is an invertible sheaf on $C$ of the same degree as that of $\omega_{S/C}^{\otimes (k-a_{t}-1)}|_{C_t}$ on $C_t$.
\end{proof}

\begin{lemma}\label{inductive_step_genus_1}
Suppose we are in Situation \ref{kollar_prop_4.7} are pairwise disjoint.
Then $f_{*}(\omega_{S/C}^{k}(\sum a_iC_i))$ is semipositive.
\end{lemma}
\begin{proof}
This proof is nearly identical to that in genus 0 and genus at least 2.

We proceed by induction on $\sum a_i$. 
The base case where all of the $a_i$ are equal to zero is Lemma \ref{base_case_genus_1}.

Assume the claim is proven for $D_{j-1}=\sum a_iC_i$; we will prove it for $D_{j}=D_{j-1}+C_t$.
By \todo{insert ref to our version of kollar 4.6}, $\omega_{S/C}\cdot C_t\ge0$.

Consider the exact sequence
\begin{equation}\label{exact_seq_of_section_twisted_genus_1}
0\to\omega_{S/C}^{\otimes m}(D_{j-1})\to\omega_{S/C}^{\otimes m}(D_j)\to \omega_{S/C}^{\otimes m}(D_j)|_{C_t}\to0
\end{equation}
obtained by tensoring the closed subscheme exact sequence for $C_t$ with $\omega_{S/C}^{\otimes m}(D_j)$.

Then, $\omega_{S/C}^{\otimes m}(D_j)|_{C_t}\cong \omega_{S/C}^{\otimes (k-a_{t}-1)}|_{C_t}$, because the $C_i$ are pairwise disjoint.
Because $a_{t}+1\le k$ by assumption, this invertible sheaf has non-negative degree.

We now claim that $R^{1}f_{*}\omega_{S/C}^{\otimes m}(D_{j-1}))=0$, unless $D_{j-1}=0$.
Indeed, by Cohomology and Base Change and the fact that $f$ has relative dimension 1, it suffices to note that $H^{1}(S_x,\omega_{S_x/\kappa(x)}^{\otimes m}(D_{j-1}\cdot S_x))=0$ for each fiber $S_x$ of $f$, which is clear by Serre Duality and degree considerations.
On the other hand, when $D_{j-1}=0$, Cohomology and Base Change shows that $R^{1}f_{*}\omega_{S/C}^{\otimes m}(D_{j-1}))$ is an invertible sheaf.


Thus, if $D_{j-1}\neq0$, applying $f_{*}$ to (\ref{exact_seq_of_section_twisted_genus_1}) expresses $f_{*}(\omega_{S/C}^{k}(\sum a_iC_i))$ as an extension of the positive degree invertible sheaf $f_{*}\omega_{S/C}^{\otimes (k-a_{t}-1)}|_{C_t}$ by the semipositive locally free sheaf $f_{*}\omega_{S/C}^{\otimes m}(D_{j-1})$, which is semipositive by \todo{ref to semipositivity section}.
Here we have used the fact that $C_t$ is a section, hence $f_{*}\omega_{S/C}^{\otimes (k-a_{t}-1)}|_{C_t}$ is an invertible sheaf on $C$ of the same degree as that of $\omega_{S/C}^{\otimes (k-a_{t}-1)}|_{C_t}$ on $C_t$.

If $D_{j-1}=0$, the situation is similar; we obtain the exact sequence
$$
0\to f_{*}(\omega_{S/C}^{\otimes m})\to f_{*}((\omega_{S/C}^{\otimes m})(C_t))\to f_{*}((\omega_{S/C}^{\otimes m})(C_t))|_{C_t}\to R^{1}f_{*}(\omega_{S/C}^{\otimes m})\to0
$$
because $R^{1}f_{*}\omega_{S/C}^{\otimes m}(C_t)=0$. 
The rightmost map is a surjection of invertible sheaves, hence an isomorphism.
Therefore the leftmost map is an isomorphism as well, and we obtain the desired conclusion once again.
\end{proof}



\begin{lemma}\label{inductive_step_genus_2}
Suppose we are in Situation \ref{kollar_prop_4.7} are pairwise disjoint.
Then $f_{*}(\omega_{S/C}^{k}(\sum a_iC_i))$ is semipositive.
\end{lemma}
\begin{proof}
This proof is nearly identical to that in genus 0 and genus 1.

We proceed by induction on $\sum a_i$. 
The base case where all of the $a_i$ are equal to zero is Lemma \todo{insert ref to kollar 4.5}.

Assume the claim is proven for $D_{j-1}=\sum a_iC_i$; we will prove it for $D_{j}=D_{j-1}+C_t$.
By \todo{insert ref to our version of kollar 4.6}, $\omega_{S/C}\cdot C_t\ge0$.

Consider the exact sequence
\begin{equation*}\label{exact_seq_of_section_twisted_genus_2}
0\to\omega_{S/C}^{\otimes m}(D_{j-1})\to\omega_{S/C}^{\otimes m}(D_j)\to \omega_{S/C}^{\otimes m}(D_j)|_{C_t}\to0
\end{equation*}
obtained by tensoring the closed subscheme exact sequence for $C_t$ with $\omega_{S/C}^{\otimes m}(D_j)$.

Then, $\omega_{S/C}^{\otimes m}(D_j)|_{C_t}\cong \omega_{S/C}^{\otimes (k-a_{t}-1)}|_{C_t}$, because the $C_i$ are pairwise disjoint.
Because $a_{t}+1\le k$ by assumption, this invertible sheaf has non-negative degree.

We now claim that $R^{1}f_{*}\omega_{S/C}^{\otimes m}(D_{j-1}))=0$.
Indeed, by Cohomology and Base Change and the fact that $f$ has relative dimension 1, it suffices to note that $H^{1}(S_x,\omega_{S_x/\kappa(x)}^{\otimes m}(D_{j-1}\cdot S_x))=0$ for each fiber $S_x$ of $f$, which is clear by Serre Duality and degree considerations.


Thus, applying $f_{*}$ to (\ref{exact_seq_of_section_twisted_genus_2}) expresses $f_{*}(\omega_{S/C}^{k}(\sum a_iC_i))$ as an extension of the positive degree invertible sheaf $f_{*}\omega_{S/C}^{\otimes (k-a_{t}-1)}|_{C_t}$ by the semipositive locally free sheaf $f_{(}\omega_{S/C}^{\otimes m}(D_{j-1})$, which is semipositive by \todo{ref to semipositivity section}.
Here we have used the fact that $C_t$ is a section, hence $f_{*}\omega_{S/C}^{\otimes (k-a_{t}-1)}|_{C_t}$ is an invertible sheaf on $C$ of the same degree as that of $\omega_{S/C}^{\otimes (k-a_{t}-1)}|_{C_t}$ on $C_t$.
\end{proof}



\begin{lemma}
Suppose we are in Situation \ref{kollar_prop_4.7}.
Then $f_{*}(\omega_{S/C}^{k}(\sum a_iC_i))$ is semipositive.
\end{lemma}
\begin{proof}
Reduce to the case where the sections are disjoint by blowing up.
\end{proof}

\bibliographystyle{unsrt}
\bibliography{references}
\end{document}

\input{preamble}


\newcommand{\todo}[1]{\footnote{\textbf{TODO.} #1}}

\begin{document}
\title{Semipositivity}
\maketitle

\section{Introduction}
This section develops the basic properties of semipositivity.


\begin{definition}
Let $G=\prod_i GL(E_i)$ be a product of general linear groups. A representation $\rho:G\rightarrow GL(F)$ is called semipositive (or polynomial) if it extends to a morphism $\overline{\rho}:\prod_i End(E_i)\rightarrow End(F)$.
\end{definition}

\begin{lemma}
The following representations are semipositive:
\item[(i)] Symmetric products;
\item[(ii)] Let $V_1, \dots, V_n$ be vector bundles with typical fibers $E_i$ and let $\rho$ be any representation of $G$. One can define a vector bundle $\rho(V_1,\dots,V_n)$ as follows: TO DO.
\item[(iii)] Assume in example (ii) that every $V_i$ is a direct sum of line bundles $\sum_j L_{ij}$. Then the structure group of  $V_i$ can be reduced to a torus. Thus the structure group of $\rho(V_1,\dots, V_n)$ is also a torus and is a direct sum of line bundles of the form $\otimes L_{ij}^{a_{ij}}$. Since $\rho$ is semipositive, each of the $a_{ij}$ is nonnegative.
\end{lemma}


\begin{proof}

\end{proof}


\begin{definition}
semipositivity of bundle
\end{definition}

\begin{lemma}
equivalent definition ( 3.3(i))
\end{lemma}

\begin{lemma}
3.4
\end{lemma}

\begin{lemma}
Let $X$ be a scheme, and let $V_i$, $i=1,2,\ldots,n$ be semipositive vector bundles of rank $r_i$ over a scheme $X$. Let $\rho:\prod_i GL(r_i)\to GL(m)$ be a semipositive representation, and let $W=\rho(V_1,\ldots,V_n)$. 
\end{lemma}

\begin{proof}
should probably have auxiliary lemmas
\end{proof}

\begin{lemma}
Let $X$ be a scheme, and let $V_i$, $i=1,2,\ldots,n$ be semipositive vector bundles of rank $r_i$ over a scheme $X$. Let $\rho:\prod_i GL(r_i)\to GL(m)$ be a semipositive representation, and let $W=\rho(V_1,\ldots,V_n)$. Let $V$ be the subbundle of $W$ associated to a subrepresentation of $\rho$. Then, if $W$ is semipositive, then $V$ is semipositive.
\end{lemma}

\begin{proof}

\end{proof}



\bibliographystyle{unsrt}
\bibliography{references}

\end{document}

\input{preamble}


\newcommand{\todo}[1]{\footnote{\textbf{TODO.} #1}}
\begin{document}
\title{Semipositivity}
\maketitle


\section{Introduction}
This section develops the basic properties of semipositivity.


\begin{definition}
Let $G$...
\end{definition}

\begin{lemma}
properties of semipositivity (maybe multiple lemmas)
\end{lemma}


\begin{proof}

\end{proof}


\begin{definition}
A locally free sheaf $V$ on a scheme $X$ is semipositive if for every map from a proper curve $f:C\to X$ every quotient line bundle of $f^*V$ has nonnegative degree.
\end{definition}

\begin{lemma}
A locally free sheaf $V$ on a scheme $X$ is semipositive if and only if $O_{Proj V}(1)$ is nef on $Proj_XV$.
\end{lemma}
\begin{proof}

\end{proof}

\begin{lemma}
3.4 Semipositivity is an open condition in flat families.
\end{lemma}

\begin{lemma}\label{global_generation_of_specific_twist_on_curve}
Let $C$ be a smooth curve of genus $g$ over a field and let $\mathcal{E}$ be a vector bundle on $C$. Let $\mathcal{L}$ be a line bundle on $C$ of degree at least $2g$. Then $\mathcal{E}\otimes\mathcal{L}$ is globally generated.
\end{lemma}

\begin{proof}

\end{proof}


\begin{lemma}
Let $C$ be a smooth curve over a field of characteristic $p>0$. Let $V_1,\ldots,V_n$ be vector bundles on $C$ with no line bundle quotients of negative degree. Then, $\rho(V_1,\ldots,V_n)$ has no line bundle quotients of negative degree.
\end{lemma}
\begin{proof}
Let $g$ be the genus of $C$, and let $\mathcal{L}$ be a line bundle of degree at least $2g$, as in Lemma \ref{apply_rho_still_semipos}. 
For each $i$, there exists a generically surjective map \todo{say why}
\begin{equation}
f_i:(\mathcal{L}^{-1})^{\oplus r_i}\to V_i
\end{equation}
Applying \todo{insert ref}, we thus get a generaically surjective \todo{why?} map
\begin{equation}
\rho(f_1,\ldots,f_n):\bigoplus_{i}\mathcal{L}^{-\otimes b_i}\to\rho(V_1,\ldots,V_n).
\end{equation}

Let $N(g,\rho)$ be the positive integer such that $\min\deg\mathcal{L}^{-b_i}=-N(g,\rho)$; $N$ depends on $g$ and $\rho$ but not on the $V_i$.\todo{why?}

Suppose $\mathcal{M}$ is a line bundle quotient of $\rho(V_1,\ldots,V_n)$. By pre-composing with $\rho(f_1,\ldots,f_n)$, we obtain a nonzero map
\begin{equation}
\bigoplus_{i}\mathcal{L}^{-\otimes b_i}\to\mathcal{M}.
\end{equation}
Thus, we conclude that
\begin{equation*}\label{degree_of_quotient_not_too_small}
\deg\mathcal{M}\ge -N.
\end{equation*}


Finally, suppose that $\rho(V_1,\ldots,V_n)$ has a line bundle quotient $\mathcal{N}$ of degree $d<0$. Let $F:C\to C$ be the absolute Frobenius map on $C$. Then, by pulling back the surjection $\rho(V_1,\ldots,V_n)\to\mathcal{N}$ by $F^t$, we obtain a quotient line bundle of $\rho((F^{t})^{*}V_1,\ldots,(F^{t})^{*}V_n)$ of degree $dp^{t}<-N$ for sufficiently large $t$, contradicting \ref{degree_of_quotient_not_too_small}.

\end{proof}

\begin{lemma}\label{no_negative_quotient_on_curve}
Let $C$ be a smooth curve any field. Let $V_1,\ldots,V_n$ be vector bundles on $C$ with no line bundle quotients of negative degree. Then, $\rho(V_1,\ldots,V_n)$ has no line bundle quotients of negative degree.
\end{lemma}
\begin{proof}
spread out, use positive char case.
\end{proof}


\begin{lemma}\label{apply_rho_still_semipos}
Let $X$ be a scheme, and let $V_i$, $i=1,2,\ldots,n$ be semipositive vector bundles of rank $r_i$ on $X$. 
Let $\rho:\prod_i GL(r_i)\to GL(m)$ be a semipositive representation, and let $W=\rho(V_1,\ldots,V_n)$. 
Then, $W$ is a semipositive vector bundle.
\end{lemma}

\begin{proof}
Immediate from Lemma \ref{no_negative_quotient_on_curve} and the definition of semipositivity.
\end{proof}

\begin{lemma}
Let $X$ be a scheme and $V$ a semipositive vector bundle on $X$. 
Then, $Sym^dV$ is a semipositive vector bundle.
\end{lemma}

\begin{proof}
Immediate from and Lemma \todo{sym is semipos} and Lemma \ref{apply_rho_still_semipos}.
\end{proof}

\begin{lemma}
Let $X$ be a scheme, and let $V_i$, $i=1,2,\ldots,n$ be semipositive vector bundles of rank $r_i$. 
Let $G=\prod_{i=1}^{n}GL(r_i)$, let $\rho:G\to GL(m)$ be a semipositive representation, and let $W=\rho(V_1,\ldots,V_n)$. Let $x\in X$ be a point, and let $V_x$ be a $G$-invariant subspace of $W_x$, which defines a subbundle $V\subset W$. 
Then, $V$ is a semipositive vector bundle.
\end{lemma}

\begin{proof}
The $G$-invariant subspace $V_x\subset W_x$ defines a subrepresentation $\sigma$ of $\rho$, and we see that $V=\sigma(V_1,\ldots,V_n)$.
 By \todo{ref}, $\sigma$ is semipositive, so by Lemma \ref{apply_rho_still_semipos}, $V$ is semipositive.
\end{proof}



\bibliographystyle{unsrt}
\bibliography{references}

\end{document}

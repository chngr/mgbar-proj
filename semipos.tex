\input{preamble}


\newcommand{\todo}[1]{\footnote{\textbf{TODO.} #1}}

\begin{document}
\title{Semipositivity}
\maketitle

\section{Introduction}
This section develops the basic properties of semipositivity.

\begin{remark}
We will work over a field throughout. One can extend the discussion to a more general situation, where we have a principal $\mathcal{G}$-bundle over a scheme $X$, where $\mathcal{G}$ is a product of constant group schemes associated to general linear groups, and a representation $\rho:\mathcal{G}\to\mathcal{GL}(m)$, where $\mathcal{Gl}(m)$ is the constant group scheme associated to $\mathrm{GL}(m)$. However, we are unaware of applications that need this level of generality, so we stick to $X$ over a field and $\rho$ a constant representation.
\end{remark}

\begin{remark}
This discussion should eventually be extended to algebraic spaces over a field to make the application to $\overline{M}_g$ work, but all of the statements should go through easily.
\end{remark}

\begin{definition}
Let $k$ be a field.
Let $V_1,\ldots,V_n,W$ be finite-dimensional vector spaces over $k$.
Let $G=\prod_{i=1}^{n}=\mathrm{GL}(V_i)$.
A representation $\rho:G\rightarrow \mathrm{GL}(W)$ is called \textit{semipositive} if it extends to a morphism of monoids $\overline{\rho}:\prod_i \mathrm{End}(V_i)\rightarrow\mathrm{End}(W)$.
\end{definition}



\begin{lemma}
\item[(i)] Let $k$ be a field. Let $V$ be a finite-dimensional vector space over $k$ and let $d$ be a positive integer. Then, the natural representations $\mathrm{GL}(V)\to \mathrm{GL}(\mathrm{Sym}^dV)$ and $\mathrm{GL}(V)\to \mathrm{GL}(\Lambda^dV)$ are semipositive.
\item[(ii)] Subrepresentations and quotients of semipositive representations are semipositive.
\end{lemma}

\begin{proof}
\item[(i)] Obvious.
\item[(ii)] Let $\rho:G\to \mathrm{GL}(W)$ be a semipositive representation and let $W'\subset W$ be a subrepresentation. 
We may easily reduce to the case $G=\mathrm{GL}(V)$. 
Then, $\rho$ extends to a morphism of monoids $\overline{\rho}:\mathrm{End}(V)\to \mathrm{End}(W)$. 
The representation $\rho$ also gives a morphism of schemes $\phi:\mathrm{GL}(V)\times W\to W$, where $W$ has been given the obvious scheme structure. 
Similarly, $\overline{\rho}$ gives rise to an extension $\overline{\phi}:\mathrm{End}(V)\times W\to W$. 

The fact that $W'\subset W$ is a subrepresentation is equivalent to the fact that the image of $\phi$ when restricted to the closed subscheme $\mathrm{GL}(V)\times W'$ is $W'$. 
It is sufficient to prove the same statement for the closed subscheme $\mathrm{End}(V)\times W'$ of $\mathrm{End}(V)\times W$. 
However, by assumption the dense open subscheme $\mathrm{GL}(V)\times W'$ of $\mathrm{End}(V)\times W'$ maps to $W'$, so $\mathrm{End}(V)\times W'$ must as well, as desired.

The case of quotients of semipositive representations is proved similarly.
\end{proof}

\xymatrix{
\prod_{i=1}^{n}\prod_{j=1}^{r_i} GL(1)\ar[r]^{\prod_i\psi_i} \ar[rd] &\prod_{i=1}^{n}GL(r_i)\ar[r]^{\rho} &GL(m)\\
&\prod_{k=1}^{m}GL(1)\ar[ru]\ar[d]^{proj_t}&\\
&GL(1)&
}
\end{equation}


Now consider the projection on the $t$-th factor $proj_t:\prod_{k=1}^mGL(1)\to GL(1)$. The composition $\prod_{i=1}^n\prod_{j=1}^{k} GL(1)\to GL(1)$ is a character, hence it is of the form $(z_{ij})_{i,j}\mapsto \prod z_{ij}^{a_{ij,t}}$. For a cover of $X$ that trivializes the bundles, $z_{ij}$ correspond to the transition functions of $\mathcal{L}_{ij}$, hence $\prod z_{ij}^{a_{ij,t}}$ correspond to $\otimes\mathcal{L}_{ij}^{a_{ij,t}}$.

\begin{definition}
Let $k$ be a field.
Let $X$ be a $k$-scheme.
Let $f:G\to H$ be a morphism of $k$-groups, and let $\pi:P\to X$ be a principal $G$-bundle corresponding to a class $\pi\in H^1(X,G)$. 
Let $f^{*}:H^1(X,G)\to H^1(X,H)$ be the induced map on cohomology. 
Then, we denote the principal $H$-bundle corresponding to the class $f^{*}(\pi)$ by $\pi_f:P\times_{G}H\to X$.
\end{definition}

\begin{definition}
Let $k$ be a field.
Let $X$ be a $k$-scheme. 
Let $V_1,\ldots,V_n$ locally free sheaves of finite ranks $r_1,\dots, r_n$, respectively.
Let $G=\prod_i \mathrm{GL}(r_i)$.
Let $\rho:G\rightarrow \mathrm{GL}(m)$ be a representation. 
Let $\pi':V'_1\times\cdots\times V'_n\to X$ be the principal $G$-bundle obtained by taking the product of the principal $\mathrm{GL}(r_i)$-bundles associated to the $V_i$.
Then, we denote the locally free sheaf associated to the principal $\mathrm{GL}(m)$-bundle $\pi'_{\rho}$ by $\rho(V_1,\dots, V_n)$.
\end{definition}





\begin{lemma}
Let $k$ be a field.
\item[(i)]
Let $V_1,\ldots,V_n,W$ be finite-dimensional vector spaces over $k$.
Let $G=\prod_{i=1}^{n}=\mathrm{GL}(V_i)$.
Let $\rho:G\rightarrow \mathrm{GL}(W)$ be a semipositive representation. Then, $*\rho*$ is a semipositive representation.
\item[(ii)] Let $X$ be a $k$-scheme.
Let $\mathcal{V}_1,\dots, \mathcal{V}_n$ and $\mathcal{V}'_1,\dots,\mathcal{V}'_n$ be locally free sheaves of finite rank on $X$.
Assume that for each $i$, $\mathrm{rank}(\mathcal{V}_i)=\mathrm{rank}(\mathcal{V}'_i)=r_i$.
For each $i$, let $f_i:\mathcal{V}_i\rightarrow \mathcal{V}'_i$ be a sheaf homomorphism.
Let $\rho:\prod\mathrm{GL}(r_i)\to\mathrm{GL}(m)$ be a representation.
If $\rho$ is semipositive, then there exists a natural map
$$
\rho(f_1,\dots, f_n):\rho(\mathcal{V}_1,\ldots,\mathcal{V}_n)\rightarrow \rho(\mathcal{V}'_1,\dots,\mathcal{V}'_n)
$$
\end{lemma}



\begin{lemma}\label{pullback_and_rho_commute}
Let $X$ be a scheme.
Let $\mathcal{V}_1,\ldots,\mathcal{V}_n$ be locally free sheaves on $X$. Let $f:X'\rightarrow X$ a morphism of schemes.
Then $\rho(f^*\mathcal{V}_1,\dots,f^*\mathcal{V}_n)$ and $f^*(\rho(\mathcal{V}'_1,\dots,\mathcal{V}'_n))$ are canonically isomorphic as locally free sheaves.
\end{lemma}


\begin{lemma}
Let $k$ be a field.
Let $X$ be a $k$-scheme.
For $i=1,2,\ldots,n$, let $V_i=\oplus_{j}\mathcal{L}_{ij}$ be a locally free sheaf of rank $r_i$ that splits as a direct sum of line bundles $\mathcal{L}_{ij}$.
Let $k$ be a field.
Let $\rho:\prod_{i}\mathrm{GL}(r_i)\to \mathrm{GL}(m)$ be a semipositive representation. 
Then, $\rho(\mathcal{V}_1,\ldots,V_n)$ is a a direct sum of line bundles of the form $\oplus_{i,j}\otimes\mathcal{L}_{ij}^{a_{ij}}$, where $a_{ij}\ge0$ for each $i,j$.
\end{lemma}

\begin{proof}
For each $i$, we have a diagonal embedding $\psi_i:\prod_{j=1}^{r_i}\mathrm{GL}(1)\to \mathrm{GL}(r_i)$.
Taking the product over $i$ and post-composing with $\rho$ yields a map
$$
\psi:\prod_{i=1}^{n}\prod_{j=1}^{r_i}\mathrm{GL}(1)\to \mathrm{GL}(m),
$$
which factors through a maximal torus $\prod_{k=1}^{m}\mathrm{GL}(1)\subset \mathrm{GL}(m)$, as in the following diagram:
$$
\xymatrix{
\prod_{i=1}^{n}\prod_{j=1}^{r_i} \mathrm{GL}(1)\ar[r]^{\psi} \ar[rd] &\prod_{i=1}^{n}\mathrm{GL}(r_i)\ar[r]^{\rho} &\mathrm{GL}(m)\\
&\prod_{k=1}^{m}\mathrm{GL}(1)\ar[ru]\ar[d]^{\mathrm{pr}_t}&\\
&\mathrm{GL}(1)&
}
$$


Now consider the projection on the $t$-th factor $\mathrm{pr}_t:\prod_{k=1}^m\mathrm{GL}(1)\to \mathrm{GL}(1)$. The composition $\prod_{i=1}^n\prod_{j=1}^{r_i} \mathrm{GL}(1)\to \mathrm{GL}(1)$ is a character, is of the form $(z_{ij})_{i,j}\mapsto \prod z_{ij}^{a_{ij,t}}$. On a trivializing cover of $X$, the $z_{ij}$ correspond to the transition functions of $\mathcal{L}_{ij}$, hence $\prod z_{ij}^{a_{ij,t}}$ correspond to $\otimes\mathcal{L}_{ij}^{a_{ij,t}}$.

Pulling back to $\prod_{k=1}^{m} \mathrm{GL}(m)$, we see that $\rho(V_1,\ldots,V_n)$ splits as the direct sum of the line bundles $\oplus_t \otimes_{i,j}L_{ij}^{a_{ij,t}}$.

\end{proof}



\begin{definition}\label{def_semipos_bundle}
Let $k$ be a field.
Let $X$ be a $k$-scheme.
Let $V$ be a locally free sheaf of finite rank on $X$.
We say that $V$ is \textit{semipositive} if for every map from a proper and smooth $k$-scheme of dimension 1 $C$, invertible sheaf $\mathcal{L}$ on $C$, $k$-morphism $f:C\to X$, and surjection $f^{*}V\to\mathcal{L}$, we have $\deg_C\mathcal{L}\ge0$.\end{definition}

\begin{lemma}\label{pullback of semipos}
Let $V$ be a semipositive vector bundle on a scheme $X$. For any morphism $g:Y\to X$, the pullback $f^*V$ is semipositive vector bundle on $Y$.
\end{lemma}
\begin{proof}
Consider any morphism $f:C\to Y$, as in Definition \ref{def_semipos_bundle}. Then as $f^*(g^*V)=(g\circ f)^*V$ and $V$ is semipositive, any quotient invertible sheaf of $g^*(f^*V)$ has non-negative degree. Hence, $f^*V$ is semipositive.
\end{proof}

\begin{lemma}\label{semipos=nef}
A locally free sheaf $V$ on a scheme $X$ is semipositive if and only if $\mathcal{O}_{\mathrm{Proj} V}(1)$ is nef on $\mathrm{Proj}_XV$.
\end{lemma}
\begin{proof}
Let $\pi: \mathrm{Proj}_X V\to V$ be the projection map. Assume $\mathcal{O}_{\mathbb{P}V}(1)$ be nef. Let $L$ be  any quotient line bundle of $f^*V$ over a smooth proper curve $C$. Then $\deg \mathcal{O}_{\mathbb{P}L}(1)=\deg L$. Since $\mathbb{P}_CL$ is a curve in $\mathbb{P}_CV$, we have $\deg L=\mathcal{O}_{\mathbb{P}V}(1)|\mathbb{P}_CL$. Hence, using that $\mathcal{O}_{\mathbb{P}V}(1)$ be nef is nef, we get $\deg L>0$ and so $V$ is semipositive.

For the other direction, consider any smooth proper $C\subset \mathbb{P}_XV$ curve. By the universal property of relative Proj $\mathcal{O}_{\mathbb{P}V}(1)|_C$ is a quotient of $\pi^*V|_C$. By lemma \ref{pullback of semipos} $\pi^*V$ is semipositive and hence, $\deg  \mathcal{O}_{\mathbb{P}V}(1)|_C\geq 0$. Thus, semipositivity of $V$ implies nefness of $\mathcal{O}_{\mathrm{Proj} V}(1)$.
\end{proof}

\begin{lemma}
3.4 Semipositivity is an open condition in flat families.
Let $f:Y\to X$ be a flat map and $V$ be a vector bundle on $Y$. Then the set of points $x\in X$ such that the vector bundle $V_x$ is semipositive on the fiber $Y_x$ is open.
\end{lemma}

\begin{proof}
It follows from lemma \ref{semipos=nef} and nefness is an open condition.
Let $\pi:\mathbb{P}_YV\to Y$ be the projection map.  Note $\pi$ is flat. Given a point $x\in X$, by lemma \ref{semipos=nef} the vector bunlde $V_x$ is semipositive if and only if $\mathcal{O}_{\mathrm{Proj} V}(1)_x$ is nef. The compositve morphism $f\circ \pi$ is flat. Since nefness is an open condition in flat families, the set of points $x\in X$ such that $\mathcal{O}_{\mathbb{P}V}(1)$ is nef is open in $X$.
\end{proof}

\begin{lemma}\label{global_generation_of_twist_on_curve}
Let $C$ be a smooth curve of genus $g$ over a field and let $\mathcal{E}$ be a vector bundle on $C$. 
Let $\mathcal{L}$ be a line bundle on $C$ of degree at least $2g$. 
Then $\mathcal{E}\otimes\mathcal{L}$ is globally generated.
\end{lemma}

\begin{proof}
By Tag OB57, it suffices to assume that the ground field is algebraically closed. 
Let $x\in C$ be a closed point with ideal sheaf $\mathcal{O}(-x)$. 
We have a short exact sequence
\begin{equation}
0\to\mathcal{E}\otimes\mathcal{L}(-x)\to\mathcal{E}\otimes\mathcal{L}\to\mathcal{E}\otimes\mathcal{L}\mid_x\to0
\end{equation}
It suffices to show that the latter map $\mathcal{E}\otimes\mathcal{L}\to\mathcal{E}\otimes\mathcal{L}\mid_x$ remains surjective after applying $H^0$, and thus to show that $H^1(C,\mathcal{E}\otimes\mathcal{L}(-x))=0$.

By Serre Duality, it suffices to show that $Hom_{\mathcal{O}_C}(\mathcal{E}\otimes\mathcal{L}(-x),\omega_C)=0$, which is equivalent to $Hom(\mathcal{E},\omega_C\otimes\mathcal{L}^{-1}(x))=0$.

The target of such a nonzero map is a locally free sheaf of negative degree, so the image subsheaf is torsion-free and hence locally free sheaf of negative degree. However, $\mathcal{E}$ has no negative line bundle quotients, so no such map can exist.
\end{proof}



\begin{lemma}
Let $C$ be a smooth curve over a field of characteristic $p>0$. Let $V_1,\ldots,V_n$ be semipositive vector bundles of ranks $r_1,\ldots,r_n$ on $C$. Let $\rho$ be a semipositive representation $\prod \mathrm{GL}(r_i)\to \mathrm{GL}(m)$. Then, $\rho(V_1,\ldots,V_n)$ is a semipositive vector bundle.
\end{lemma}
\begin{proof}
Let $g$ be the genus of $C$, and fix a line bundle $\mathcal{L}$ of degree at least $2g$.
By Lemma \ref{global_generation_of_twist_on_curve}, $V_i\otimes\mathcal{L}$ is generated by global sections for each $i$, so there exist generically surjective maps
\begin{equation}
f_i:(\mathcal{L}^{-1})^{\oplus r_i}\to V_i.
\end{equation}
Applying \todo{``functoriality'' of rho} and \todo{rho of direct sum of line bundles}, get a generaically surjective \todo{why?} map
\begin{equation}
\rho(f_1,\ldots,f_n):\bigoplus_{i}\mathcal{L}^{-\otimes b_i}\to\rho(V_1,\ldots,V_n),
\end{equation}
where the source is $\rho(\mathcal{L}^{-\otimes r_1},\ldots,\mathcal{L}^{-\otimes r_n})$.
Thus, the $b_i$ are positive integers depending on the ranks of the $V_i$ but not the $V_i$ themselves.
In particular, there exists a positive integer $N$ such that $\min\deg\mathcal{L}^{-b_i}=-N$, where $N$ does not depend on the $V_i$.

Suppose $\mathcal{M}$ is a line bundle quotient of $\rho(V_1,\ldots,V_n)$. By pre-composing with $\rho(f_1,\ldots,f_n)$, we obtain a nonzero map
\begin{equation}
\bigoplus_{i}\mathcal{L}^{-\otimes b_i}\to\mathcal{M}.
\end{equation}
Thus, we conclude that
\begin{equation}\label{degree_of_quotient_not_too_small}
\deg\mathcal{M}\ge -N.
\end{equation}

Finally, suppose that $\rho(V_1,\ldots,V_n)$ has a line bundle quotient $\mathcal{N}$ of degree $d<0$. 
Let $F:C\to C$ be the absolute Frobenius map on $C$. 
Then, by pulling back the surjection $\rho(V_1,\ldots,V_n)\to\mathcal{N}$ by $F^t$, we obtain a quotient line bundle of $\rho((F^{t})^{*}V_1,\ldots,(F^{t})^{*}V_n)$ of degree $dp^{t}<-N$ for sufficiently large $t$, by \todo{compatibility of $\rho$ with pullback}
However, as $(F^{t})^*V_i$ is semipositive for each $i$, this contradicts (\ref{degree_of_quotient_not_too_small}), so hence $\mathcal{N}$ cannot exist.

\end{proof}

\begin{lemma}\label{no_negative_quotient_on_curve}
Let $C$ be a smooth curve over a field. Let $V_1,\ldots,V_n$ be semipositive vector bundles of ranks $r_1,\ldots,r_n$ on $C$. Let $\rho$ be a semipositive representation $\prod\mathrm{GL}(r_i)\to \mathrm{GL}(m)$. Then, $\rho(V_1,\ldots,V_n)$ is a semipositive vector bundle.
\end{lemma}
\begin{proof}
If the characteristic of the ground field is positive, this is Lemma \ref{semipos_of_symd_curve_charp}. Suppose that the characteristic of the ground field is zero.
\end{proof}


\begin{lemma}\label{apply_rho_still_semipos}
Let $X$ be a scheme, and let $V_i$, $i=1,2,\ldots,n$ be semipositive vector bundles of rank $r_i$ on $X$. 
Let $\rho:\prod_i \mathrm{GL}(r_i)\to \mathrm{GL}(m)$ be a semipositive representation, and let $W=\rho(V_1,\ldots,V_n)$. 
Then, $W$ is a semipositive vector bundle.
\end{lemma}

\begin{proof}
Immediate from Lemma \ref{no_negative_quotient_on_curve} and the definition of semipositivity.
\end{proof}

	
%\begin{lemma}\label{semipos_of_symd_curve_charp}
%Let $C$ be a smooth curve over a field of characteristic $p>0$. Let $V$ be a semipositive vector bundle on $C$. Then, for any positive integer $d$, $Sym^dV$ is a semipositive vector bundle.
%\end{lemma}
%\begin{proof}
%Let $g$ be the genus of $C$, and let $\mathcal{L}$ be a line bundle on $C$ of degree at least $2g$. 
%By Lemma \ref{global_generation_of_twist_on_curve}, $V\otimes\mathcal{L}^{-1}$ is globally generated, so there exists a generically surjective map
%\begin{equation}
%f:(\mathcal{L}^{-1})^{\oplus r_i}\to V
%\end{equation}
%coming from a collection of $r$ global sections generating the fiber at some closed point of $C$.
%Applying $Sym^d$, we thus get a generically surjective map
%\begin{equation}
%Sym^df:(\mathcal{L}^{-\otimes d})^{\binom{d+r}{r}}\to Sym^{d}V
%\end{equation}
%
%Suppose $\mathcal{M}$ is a line bundle quotient of $Sym^dV$. By pre-composing with $\rho(f_1,\ldots,f_n)$, we obtain a nonzero map
%\begin{equation}
%\bigoplus_{i}\mathcal{L}^{-\otimes d}\to\mathcal{M}.
%\end{equation}
%Thus, we conclude that
%\begin{equation}\label{degree_of_quotient_not_too_small}
%\deg\mathcal{M}\ge d\cdot\deg(\mathcal{L}).
%\end{equation}
%
%Finally, suppose that $\rho(V_1,\ldots,V_n)$ has a line bundle quotient $\mathcal{N}$ of degree $e<0$. Let $F:C\to C$ be the absolute Frobenius map on $C$. Then, by pulling back the surjection $Sym^dV\to\mathcal{N}$ by $F^t$, where $t$ is a positive integer, we obtain a quotient line bundle of $(F^{t})^*Sym^dV\cong Sym^d(F^{*}V)$ of degree $dp^{t}<-N$ for sufficiently large $t$, which contradicts (\ref{degree_of_quotient_not_too_small}) due to Lemma \todo{pullback is still positive}.
%
%\end{proof}
%
%
%\begin{lemma}\label{semipos_of_symd_curve_all_char}
%Let $C$ be a smooth curve over a field of arbitrary characteristic. Let $V$ be a semipositive vector bundle on $C$. Then, for any positive integer $d$, $Sym^dV$ is a semipositive vector bundle.
%\end{lemma}
%\begin{proof}
%If the characteristic of the ground field is positive, this is Lemma \ref{semipos_of_symd_curve_charp}. Suppose that the characteristic of the ground field is zero.
%\end{proof}

\begin{lemma}\label{symd_semipos}
Let $X$ be a scheme and $V$ a semipositive vector bundle on $X$. 
Then, $\mathrm{Sym}^dV$ is a semipositive vector bundle.
\end{lemma}
\begin{proof}
This is Lemma \ref{apply_rho_still_semipos} applied to the semipositive representation $\mathrm{Sym}^d$, see \todo{ref}.
\end{proof}


\begin{lemma}
Let $X$ be a scheme, and let $V_1,\ldots,V_n$ be semipositive vector bundles on $X$ of ranks $r_1,\ldots,r_n$, respectively. 
Let $G=\prod_{i=1}^{n}\mathrm{GL}(r_i)$, let $\rho:G\to \mathrm{GL}(m)$ be a semipositive representation, and let $W=\rho(V_1,\ldots,V_n)$.
Let $V\subset W$ be a $G$-subbundle of $W$. \todo{subbundle or subsheaf that is a bundle?}
Then, $V$ is a semipositive vector bundle.
\end{lemma}

\begin{proof}
Let $x\in X$ be any point.
The $G$-invariant subspace $V_x\subset W_x$ defines a subrepresentation $\sigma$ of $\rho$, and we see that $V=\sigma(V_1,\ldots,V_n)$.
By \todo{add ref}, $\sigma$ is semipositive, so by Lemma \ref{apply_rho_still_semipos}, $V$ is semipositive.
\end{proof}



\bibliographystyle{unsrt}
\bibliography{references}

\end{document}

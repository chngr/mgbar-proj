\IfFileExists{stacks-project.cls}{%
\documentclass{stacks-project}
}{%
\documentclass{amsart}
}

% The following AMS packages are automatically loaded with
% the amsart documentclass:
%\usepackage{amsmath}
%\usepackage{amssymb}
%\usepackage{amsthm}

% For dealing with references we use the comment environment
\usepackage{verbatim}
\newenvironment{reference}{\comment}{\endcomment}
%\newenvironment{reference}{}{}
\newenvironment{slogan}{\comment}{\endcomment}
\newenvironment{history}{\comment}{\endcomment}

% For commutative diagrams you can use
% \usepackage{amscd}
\usepackage[all]{xy}

% We use 2cell for 2-commutative diagrams.
\xyoption{2cell}
\UseAllTwocells

% To put source file link in headers.
% Change "template.tex" to "this_filename.tex"
% \usepackage{fancyhdr}
% \pagestyle{fancy}
% \lhead{}
% \chead{}
% \rhead{Source file: \url{template.tex}}
% \lfoot{}
% \cfoot{\thepage}
% \rfoot{}
% \renewcommand{\headrulewidth}{0pt}
% \renewcommand{\footrulewidth}{0pt}
% \renewcommand{\headheight}{12pt}

\usepackage{multicol}

% For cross-file-references
\usepackage{xr-hyper}

% Package for hypertext links:
\usepackage{hyperref}

% For any local file, say "hello.tex" you want to link to please
% use \externaldocument[hello-]{hello}
\externaldocument[introduction-]{introduction}
\externaldocument[conventions-]{conventions}
\externaldocument[sets-]{sets}
\externaldocument[categories-]{categories}
\externaldocument[topology-]{topology}
\externaldocument[sheaves-]{sheaves}
\externaldocument[sites-]{sites}
\externaldocument[stacks-]{stacks}
\externaldocument[fields-]{fields}
\externaldocument[algebra-]{algebra}
\externaldocument[brauer-]{brauer}
\externaldocument[homology-]{homology}
\externaldocument[derived-]{derived}
\externaldocument[simplicial-]{simplicial}
\externaldocument[more-algebra-]{more-algebra}
\externaldocument[smoothing-]{smoothing}
\externaldocument[modules-]{modules}
\externaldocument[sites-modules-]{sites-modules}
\externaldocument[injectives-]{injectives}
\externaldocument[cohomology-]{cohomology}
\externaldocument[sites-cohomology-]{sites-cohomology}
\externaldocument[dga-]{dga}
\externaldocument[dpa-]{dpa}
\externaldocument[hypercovering-]{hypercovering}
\externaldocument[schemes-]{schemes}
\externaldocument[constructions-]{constructions}
\externaldocument[properties-]{properties}
\externaldocument[morphisms-]{morphisms}
\externaldocument[coherent-]{coherent}
\externaldocument[divisors-]{divisors}
\externaldocument[limits-]{limits}
\externaldocument[varieties-]{varieties}
\externaldocument[topologies-]{topologies}
\externaldocument[descent-]{descent}
\externaldocument[perfect-]{perfect}
\externaldocument[more-morphisms-]{more-morphisms}
\externaldocument[flat-]{flat}
\externaldocument[groupoids-]{groupoids}
\externaldocument[more-groupoids-]{more-groupoids}
\externaldocument[etale-]{etale}
\externaldocument[chow-]{chow}
\externaldocument[intersection-]{intersection}
\externaldocument[pic-]{pic}
\externaldocument[adequate-]{adequate}
\externaldocument[dualizing-]{dualizing}
\externaldocument[duality-]{duality}
\externaldocument[discriminant-]{discriminant}
\externaldocument[local-cohomology-]{local-cohomology}
\externaldocument[curves-]{curves}
\externaldocument[resolve-]{resolve}
\externaldocument[models-]{models}
\externaldocument[pione-]{pione}
\externaldocument[etale-cohomology-]{etale-cohomology}
\externaldocument[proetale-]{proetale}
\externaldocument[crystalline-]{crystalline}
\externaldocument[spaces-]{spaces}
\externaldocument[spaces-properties-]{spaces-properties}
\externaldocument[spaces-morphisms-]{spaces-morphisms}
\externaldocument[decent-spaces-]{decent-spaces}
\externaldocument[spaces-cohomology-]{spaces-cohomology}
\externaldocument[spaces-limits-]{spaces-limits}
\externaldocument[spaces-divisors-]{spaces-divisors}
\externaldocument[spaces-over-fields-]{spaces-over-fields}
\externaldocument[spaces-topologies-]{spaces-topologies}
\externaldocument[spaces-descent-]{spaces-descent}
\externaldocument[spaces-perfect-]{spaces-perfect}
\externaldocument[spaces-more-morphisms-]{spaces-more-morphisms}
\externaldocument[spaces-flat-]{spaces-flat}
\externaldocument[spaces-groupoids-]{spaces-groupoids}
\externaldocument[spaces-more-groupoids-]{spaces-more-groupoids}
\externaldocument[bootstrap-]{bootstrap}
\externaldocument[spaces-pushouts-]{spaces-pushouts}
\externaldocument[groupoids-quotients-]{groupoids-quotients}
\externaldocument[spaces-more-cohomology-]{spaces-more-cohomology}
\externaldocument[spaces-simplicial-]{spaces-simplicial}
\externaldocument[spaces-duality-]{spaces-duality}
\externaldocument[formal-spaces-]{formal-spaces}
\externaldocument[restricted-]{restricted}
\externaldocument[spaces-resolve-]{spaces-resolve}
\externaldocument[formal-defos-]{formal-defos}
\externaldocument[defos-]{defos}
\externaldocument[cotangent-]{cotangent}
\externaldocument[examples-defos-]{examples-defos}
\externaldocument[algebraic-]{algebraic}
\externaldocument[examples-stacks-]{examples-stacks}
\externaldocument[stacks-sheaves-]{stacks-sheaves}
\externaldocument[criteria-]{criteria}
\externaldocument[artin-]{artin}
\externaldocument[quot-]{quot}
\externaldocument[stacks-properties-]{stacks-properties}
\externaldocument[stacks-morphisms-]{stacks-morphisms}
\externaldocument[stacks-limits-]{stacks-limits}
\externaldocument[stacks-cohomology-]{stacks-cohomology}
\externaldocument[stacks-perfect-]{stacks-perfect}
\externaldocument[stacks-introduction-]{stacks-introduction}
\externaldocument[stacks-more-morphisms-]{stacks-more-morphisms}
\externaldocument[stacks-geometry-]{stacks-geometry}
\externaldocument[moduli-]{moduli}
\externaldocument[moduli-curves-]{moduli-curves}
\externaldocument[examples-]{examples}
\externaldocument[exercises-]{exercises}
\externaldocument[guide-]{guide}
\externaldocument[desirables-]{desirables}
\externaldocument[coding-]{coding}
\externaldocument[obsolete-]{obsolete}
\externaldocument[fdl-]{fdl}
\externaldocument[index-]{index}

% Theorem environments.
%
\theoremstyle{plain}
\newtheorem{theorem}[subsection]{Theorem}
\newtheorem{proposition}[subsection]{Proposition}
\newtheorem{lemma}[subsection]{Lemma}

\theoremstyle{definition}
\newtheorem{definition}[subsection]{Definition}
\newtheorem{example}[subsection]{Example}
\newtheorem{exercise}[subsection]{Exercise}
\newtheorem{situation}[subsection]{Situation}

\theoremstyle{remark}
\newtheorem{remark}[subsection]{Remark}
\newtheorem{remarks}[subsection]{Remarks}

\numberwithin{equation}{subsection}

% Macros
%
\def\lim{\mathop{\rm lim}\nolimits}
\def\colim{\mathop{\rm colim}\nolimits}
\def\Spec{\mathop{\rm Spec}}
\def\Hom{\mathop{\rm Hom}\nolimits}
\def\Ext{\mathop{\rm Ext}\nolimits}
\def\SheafHom{\mathop{\mathcal{H}\!{\it om}}\nolimits}
\def\SheafExt{\mathop{\mathcal{E}\!{\it xt}}\nolimits}
\def\Sch{\textit{Sch}}
\def\Mor{\mathop{\rm Mor}\nolimits}
\def\Ob{\mathop{\rm Ob}\nolimits}
\def\Sh{\mathop{\textit{Sh}}\nolimits}
\def\NL{\mathop{N\!L}\nolimits}
\def\proetale{{pro\text{-}\acute{e}tale}}
\def\etale{{\acute{e}tale}}
\def\QCoh{\textit{QCoh}}
\def\Ker{\mathop{\rm Ker}}
\def\Im{\mathop{\rm Im}}
\def\Coker{\mathop{\rm Coker}}
\def\Coim{\mathop{\rm Coim}}

%
% Macros for moduli stacks/spaces
%
\def\QCohstack{\mathcal{QC}\!{\it oh}}
\def\Cohstack{\mathcal{C}\!{\it oh}}
\def\Spacesstack{\mathcal{S}\!{\it paces}}
\def\Quotfunctor{{\rm Quot}}
\def\Hilbfunctor{{\rm Hilb}}
\def\Curvesstack{\mathcal{C}\!{\it urves}}
\def\Polarizedstack{\mathcal{P}\!{\it olarized}}
\def\Complexesstack{\mathcal{C}\!{\it omplexes}}
% \Pic is the operator that assigns to X its picard group, usage \Pic(X)
% \Picardstack_{X/B} denotes the Picard stack of X over B
% \Picardfunctor_{X/B} denotes the Picard functor of X over B
\def\Pic{\mathop{\rm Pic}\nolimits}
\def\Picardstack{\mathcal{P}\!{\it ic}}
\def\Picardfunctor{{\rm Pic}}
\def\Deformationcategory{\mathcal{D}\!{\it ef}}



\newcommand{\todo}[1]{\footnote{\textbf{TODO.} #1}}

\begin{document}
\title{Semipositivity}
\maketitle

\section{Introduction}
This section develops the basic properties of semipositivity.


\begin{definition}
Let $G=\prod_i GL(E_i)$ be a product of general linear groups. A representation $\rho:G\rightarrow GL(F)$ is called semipositive (or polynomial) if it extends to a morphism $\overline{\rho}:\prod_i End(E_i)\rightarrow End(F)$.
\end{definition}

\todo{give notation for $H$-bundle you get from a $G$-bundle and a map $G\to H$}.

\begin{definition}
Let $V_1,\dots, V_n$ vector bundles of rank $r_1,\dots, r_n$ over a base space X, and let $\rho:\prod_i GL(r_i)\rightarrow GL(m)$ be any representation. One can define a bundle $\rho(V_1,\dots, V_n)$ of rank $m$ as follows: every $V_i$ is a principal $GL(r_i)$-bundle, hence it corresponds to a point $x_i$ in $H^1(X, GL(r_i))$; the map $\rho$ induces a map $\rho^*:H^1(X,\prod_i GL(r_i)\rightarrow H^1(X, GL(m))$, i.e. $\rho: \prod_i H^1(X,GL(r_i))\rightarrow H^1(X, GL(m))$. We define $\rho(V_1,\dots, V_n)$ to be the principal $GL(m)$-bundle corresponding to $\rho^*(x_1,\dots,x_n)$.
\end{definition}

\begin{lemma}
For $i=1,2,\ldots,n$, let $V_i=\sum_{j}\mathcal{L}_{ij}$ be a vector bundle of rank $r_i$ that splits as a direct sum of line bundles $\mathcal{L}_{ij}$. Let $\rho:\prod_{i}GL(r_i)\to GL(m)$ be a semipositive representation. Then, $\rho(V_1,\ldots,V_n)$ is a a direct sum of line bundles of the form $\oplus_{i,j}\otimes\mathcal{L}_{ij}^{a_{ij}}$, where $a_{ij}\ge0$.
\end{lemma}

\begin{proof}
\todo{add details, maybe}
For each $i$, we have a diagonal embedding $\psi_i:\prod_{j=1}^{r_i}GL(1)\to GL(r_i)$...
Taking the product over $i$ and post-composing with $\rho$ yields a map
\begin{equation}
\prod_{i=1}^{n}\prod_{j=1}^{r_i}GL(1)\to GL(m),
\end{equation}
which factors through a maximal torus $\psi:\prod_{k=1}^{m}GL(1)\to GL(m)$.
\todo{make diagram...}

Thus $\rho(V_1,\ldots,V_n)$ splits as the direct sum of the line bundles you get from $\prod\prod GL(1)\to GL(1)$ after projecting to the $t$-th factor...

This map is a character, so of the form $(z_{ij})\mapsto \prod z_{ij}^{a_{ij,t}}$, and $a_{ij,t}\ge0$ by semipositivity.

Then $\rho(V_1,\ldots,V_n)$ is the direct sum of $\otimes\mathcal{L}_{ij}^{a_{ij,t}}$ over $t\in[1,n]$.

\end{proof}

\begin{lemma}
The following properties hold for semipositive representations:
\item[(i)] Symmetric products are semipositive;
\item[(ii)] Subrepresentations of semipositive representations are semipositive;
\item[(iii)] Let $V_1,\dots, V_n$ be vector bundles such that each $V_i$ is a sum of line bundles $\sum_j L_{ij}$, and $\rho$ a semipositive respresentation. Then the structure group of $\rho(V_1,\dots, V_n)$ is a torus and is a direct sum of line bundles $\otimes L_{ij}^{a_{ij}}$, where each of the $a_{ij}$ are nonnegative.
\item[(iv)] Let $V_1,\dots, V_n$ and $W_1,\dots, W_n$ be vector bundles such that $rk (V_i)=rk(W_i)$, and let $f_i: V_i\rightarrow W_i$ be sheaf homomorphisms. If $\rho$ is semipositive, then there exists a natural map 
$$\rho(f_1,\dots, f_n):\rho(V_1,\dots, V_n)\rightarrow \rho(W_1,\dots,W_n).$$
\item[(v)] Let $V_1,\dots, V_n$ be vector bundles over a base $X$, and $f:X'\rightarrow X$ a morphism. Then $\rho(f^*V_1,\dots,f^*V_n)=f^*(\rho(V_1,\dots,V_n))$.
\end{lemma}


\begin{proof}

\end{proof}


\begin{definition}
A locally free sheaf $V$ on a scheme $X$ is semipositive if for every map from a proper smooth curve $f:C\to X$ every quotient line bundle of $f^*V$ has nonnegative degree.
\end{definition}

\begin{lemma}\label{semipos=nef}
A locally free sheaf $V$ on a scheme $X$ is semipositive if and only if $\mathcal{O}_{Proj V}(1)$ is nef on $Proj_XV$.
\end{lemma}
\begin{proof}
Let $\pi: Proj_X V\to V$ be the projection map. Assume $\mathcal{O}_{\mathbb{P}V}(1)$ be nef. Let $L$ be  any quotient line bundle of $f^*V$ over a smooth proper curve $C$. Then $\deg \mathcal{O}_{\mathbb{P}L}(1)=\deg L$. Since $\mathbb{P}_CL$ is a curve in $\mathbb{P}_CV$, we have $\deg L=\mathcal{O}_{\mathbb{P}V}(1)|\mathbb{P}_CL$. Hence, using that $\mathcal{O}_{\mathbb{P}V}(1)$ be nef is nef, we get $\deg L>0$ and so $V$ is semipositive.

For the other direction, consider any smooth proper $C\subset \mathbb{P}_XV$ curve. By the universal property of relative Proj $\mathcal{O}_{\mathbb{P}V}(1)|_C$ is a quotient of $\pi^*V|_C$. Thus, semipositivity of $V$ implies nefness of $\mathcal{O}_{Proj V}(1)$.
\end{proof}

\begin{lemma}
3.4 Semipositivity is an open condition in flat families.
Let $f:Y\to X$ be a flat map and $V$ be a vector bundle on $Y$. Then the set of points $x\in X$ such that the vector bundle $V_x$ is semipositive on the fiber $Y_x$ is open.
\end{lemma}

\begin{proof}
It follows from lemma \ref{semipos=nef} and nefness is an open condition.
Let $\pi:\mathbb{P}_YV\to Y$ be the projection map.  Note $\pi$ is flat. Given a point $x\in X$, by lemma \ref{semipos=nef} the vector bunlde $V_x$ is semipositive if and only if $\mathcal{O}_{Proj V}(1)_x$ is nef. The compositve morphism $f\circ \pi$ is flat. Since nefness is an open condition in flat families, the set of points $x\in X$ such that $\mathcal{O}_{\mathbb{P}V}(1)$ is nef is open in $X$.
\end{proof}

\begin{lemma}\label{global_generation_of_specific_twist_on_curve}
Let $C$ be a smooth curve of genus $g$ over a field and let $\mathcal{E}$ be a vector bundle on $C$. Let $\mathcal{L}$ be a line bundle on $C$ of degree at least $2g$. Then $\mathcal{E}\otimes\mathcal{L}$ is globally generated.
\end{lemma}

\begin{proof}

\end{proof}


\begin{lemma}
Let $C$ be a smooth curve over a field of characteristic $p>0$. Let $V_1,\ldots,V_n$ be vector bundles on $C$ with no line bundle quotients of negative degree. Then, $\rho(V_1,\ldots,V_n)$ has no line bundle quotients of negative degree.
\end{lemma}
\begin{proof}
Let $g$ be the genus of $C$, and let $\mathcal{L}$ be a line bundle of degree at least $2g$, as in Lemma \ref{apply_rho_still_semipos}. 
For each $i$, there exists a generically surjective map \todo{say why}
\begin{equation}
f_i:(\mathcal{L}^{-1})^{\oplus r_i}\to V_i
\end{equation}
Applying \todo{insert ref}, we thus get a generaically surjective \todo{why?} map
\begin{equation}
\rho(f_1,\ldots,f_n):\bigoplus_{i}\mathcal{L}^{-\otimes b_i}\to\rho(V_1,\ldots,V_n).
\end{equation}

Let $N(g,\rho)$ be the positive integer such that $\min\deg\mathcal{L}^{-b_i}=-N(g,\rho)$; $N$ depends on $g$ and $\rho$ but not on the $V_i$.\todo{why?}

Suppose $\mathcal{M}$ is a line bundle quotient of $\rho(V_1,\ldots,V_n)$. By pre-composing with $\rho(f_1,\ldots,f_n)$, we obtain a nonzero map
\begin{equation}
\bigoplus_{i}\mathcal{L}^{-\otimes b_i}\to\mathcal{M}.
\end{equation}
Thus, we conclude that
\begin{equation*}\label{degree_of_quotient_not_too_small}
\deg\mathcal{M}\ge -N.
\end{equation*}


Finally, suppose that $\rho(V_1,\ldots,V_n)$ has a line bundle quotient $\mathcal{N}$ of degree $d<0$. Let $F:C\to C$ be the absolute Frobenius map on $C$. Then, by pulling back the surjection $\rho(V_1,\ldots,V_n)\to\mathcal{N}$ by $F^t$, we obtain a quotient line bundle of $\rho((F^{t})^{*}V_1,\ldots,(F^{t})^{*}V_n)$ of degree $dp^{t}<-N$ for sufficiently large $t$, contradicting \ref{degree_of_quotient_not_too_small}.

\end{proof}

\begin{lemma}\label{no_negative_quotient_on_curve}
Let $C$ be a smooth curve any field. Let $V_1,\ldots,V_n$ be vector bundles on $C$ with no line bundle quotients of negative degree. Then, $\rho(V_1,\ldots,V_n)$ has no line bundle quotients of negative degree.
\end{lemma}
\begin{proof}
spread out, use positive char case.
\end{proof}


\begin{lemma}\label{apply_rho_still_semipos}
Let $X$ be a scheme, and let $V_i$, $i=1,2,\ldots,n$ be semipositive vector bundles of rank $r_i$ on $X$. 
Let $\rho:\prod_i GL(r_i)\to GL(m)$ be a semipositive representation, and let $W=\rho(V_1,\ldots,V_n)$. 
Then, $W$ is a semipositive vector bundle.
\end{lemma}

\begin{proof}
Immediate from Lemma \ref{no_negative_quotient_on_curve} and the definition of semipositivity.
\end{proof}

\begin{lemma}
Let $X$ be a scheme and $V$ a semipositive vector bundle on $X$. 
Then, $Sym^dV$ is a semipositive vector bundle.
\end{lemma}


\begin{lemma}
Let $X$ be a scheme, and let $V_i$, $i=1,2,\ldots,n$ be semipositive vector bundles of rank $r_i$. 
Let $G=\prod_{i=1}^{n}GL(r_i)$, let $\rho:G\to GL(m)$ be a semipositive representation, and let $W=\rho(V_1,\ldots,V_n)$. Let $x\in X$ be a point, and let $V_x$ be a $G$-invariant subspace of $W_x$, which defines a subbundle $V\subset W$. 
Then, $V$ is a semipositive vector bundle.
\end{lemma}

\begin{proof}
The $G$-invariant subspace $V_x\subset W_x$ defines a subrepresentation $\sigma$ of $\rho$, and we see that $V=\sigma(V_1,\ldots,V_n)$.
 By \todo{ref}, $\sigma$ is semipositive, so by Lemma \ref{apply_rho_still_semipos}, $V$ is semipositive.
\end{proof}



\bibliographystyle{unsrt}
\bibliography{references}

\end{document}

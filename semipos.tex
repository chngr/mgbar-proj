\input{preamble}


\newcommand{\todo}[1]{\footnote{\textbf{TODO.} #1}}

\begin{document}
\title{Semipositivity}
\maketitle


\section{Introduction}
This section develops the basic properties of semipositivity.


\begin{definition}
Let $G=\prod_i GL(E_i)$ be a product of general linear groups. A representation $\rho:G\rightarrow GL(F)$ is called semipositive (or polynomial) if it extends to a morphism $\overline{\rho}:\prod_i End(E_i)\rightarrow End(F)$.
\end{definition}

\begin{lemma}
The following representations are semipositive:
\item[(i)] Symmetric products;
\item[(ii)] Let $V_1, \dots, V_n$ be vector bundles with typical fibers $E_i$ and let $\rho$ be any representation of $G$. One can define a vector bundle $\rho(V_1,\dots,V_n)$ as follows: TO DO.
\item[(iii)] Assume in example (ii) that every $V_i$ is a direct sum of line bundles $\sum_j L_{ij}$. Then the structure group of  $V_i$ can be reduced to a torus. Thus the structure group of $\rho(V_1,\dots, V_n)$ is also a torus and is a direct sum of line bundles of the form $\otimes L_{ij}^{a_{ij}}$. Since $\rho$ is semipositive, each of the $a_{ij}$ is nonnegative.
\end{lemma}


\begin{proof}

\end{proof}


\begin{definition}
A locally free sheaf $V$ on a scheme $X$ is semipositive if for every map from a proper curve $f:C\to X$ every quotient line bundle of $f^*V$ has nonnegative degree.
\end{definition}

\begin{lemma}
A locally free sheaf $V$ on a scheme $X$ is semipositive if and only if $O_{Proj V}(1)$ is nef on $Proj_XV$.
\end{lemma}
\begin{proof}

\end{proof}

\begin{lemma}
3.4 Semipositivity is an open condition in flat families.
\end{lemma}

\begin{lemma}\label{apply_rho_still_semipos}
Let $X$ be a scheme, and let $V_i$, $i=1,2,\ldots,n$ be semipositive vector bundles of rank $r_i$ on $X$. Let $\rho:\prod_i GL(r_i)\to GL(m)$ be a semipositive representation, and let $W=\rho(V_1,\ldots,V_n)$. Then, $W$ is a semipositive vector bundle.
\end{lemma}

\begin{proof}
should probably have auxiliary lemmas
\end{proof}

\begin{lemma}
Let $X$ be a scheme and $V$ a semipositive vector bundle on $X$. Then, $Sym^dV$ is a semipositive vector bundle.
\end{lemma}

\begin{proof}
Immediate from and Lemma \todo{sym is semipos} and Lemma \ref{apply_rho_still_semipos}.
\end{proof}

\begin{lemma}
Let $X$ be a scheme, and let $V_i$, $i=1,2,\ldots,n$ be semipositive vector bundles of rank $r_i$. Let $G=\prod_{i=1}^{n}GL(r_i)$, let $\rho:G\to GL(m)$ be a semipositive representation, and let $W=\rho(V_1,\ldots,V_n)$. Let $x\in X$ be a point, and let $V_x$ be a $G$-invariant subspace of $W_x$, which defines a subbundle $V\subset W$. Then, $V$ is a semipositive vector bundle.
\end{lemma}

\begin{proof}
The $G$-invariant subspace $V_x\subset W_x$ defines a subrepresentation $\sigma$ of $\rho$, and we see that $V=\sigma(V_1,\ldots,V_n)$. By \todo{ref}, $\sigma$ is semipositive, so by Lemma \ref{apply_rho_still_semipos}, $V$ is semipositive.
\end{proof}



\bibliographystyle{unsrt}
\bibliography{references}

\end{document}

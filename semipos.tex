\input{preamble}


\newcommand{\todo}[1]{\footnote{\textbf{TODO.} #1}}

\begin{document}
\title{Semipositivity}
\maketitle

\section{Introduction}
This section develops the basic properties of semipositivity.


\begin{definition}
Let $G=\prod_i GL(E_i)$ be a product of general linear groups. A representation $\rho:G\rightarrow GL(F)$ is called semipositive (or polynomial) if it extends to a morphism $\overline{\rho}:\prod_i End(E_i)\rightarrow End(F)$.
\end{definition}

\begin{lemma}
The following representations are semipositive:
\item[(i)] Symmetric products;
\item[(ii)] Let $V_1, \dots, V_n$ be vector bundles with typical fibers $E_i$ and let $\rho$ be any representation of $G$. One can define a vector bundle $\rho(V_1,\dots,V_n)$ as follows: TO DO.
\item[(iii)] Assume in example (ii) that every $V_i$ is a direct sum of line bundles $\sum_j L_{ij}$. Then the structure group of  $V_i$ can be reduced to a torus. Thus the structure group of $\rho(V_1,\dots, V_n)$ is also a torus and is a direct sum of line bundles of the form $\otimes L_{ij}^{a_{ij}}$. Since $\rho$ is semipositive, each of the $a_{ij}$ is nonnegative.
\end{lemma}


\begin{proof}

\end{proof}


\begin{definition}
A locally free sheaf $V$ on a scheme $X$ is semipositive if for every map from a proper smooth curve $f:C\to X$ every quotient line bundle of $f^*V$ has nonnegative degree.
\end{definition}

\begin{lemma}\label{semipos=nef}
A locally free sheaf $V$ on a scheme $X$ is semipositive if and only if $\mathcal{O}_{Proj V}(1)$ is nef on $Proj_XV$.
\end{lemma}
\begin{proof}
Let $\pi: Proj_X V\to V$ be the projection map. Assume $\mathcal{O}_{\mathbb{P}V}(1)$ be nef. Let $L$ be  any quotient line bundle of $f^*V$ over a smooth proper curve $C$. Then $\deg \mathcal{O}_{\mathbb{P}L}(1)=\deg L$. Since $\mathbb{P}_CL$ is a curve in $\mathbb{P}_CV$, we have $\deg L=\mathcal{O}_{\mathbb{P}V}(1)|\mathbb{P}_CL$. Hence, using that $\mathcal{O}_{\mathbb{P}V}(1)$ be nef is nef, we get $\deg L>0$ and so $V$ is semipositive.

For the other direction, consider any smooth proper $C\subset \mathbb{P}_XV$ curve. By the universal property of relative Proj $\mathcal{O}_{\mathbb{P}V}(1)|_C$ is a quotient of $\pi^*V|_C$. Thus, semipositivity of $V$ implies nefness of $\mathcal{O}_{Proj V}(1)$.
\end{proof}

\begin{lemma}
3.4 Semipositivity is an open condition in flat families.
Let $f:Y\to X$ be a flat map and $V$ be a vector bundle on $Y$. Then the set of points $x\in X$ such that the vector bundle $V_x$ is semipositive on the fiber $Y_x$ is open.
\end{lemma}
\begin{proof}
It follows from lemma \ref{semipos=nef} and nefness is an open condition.
Let $\pi:\mathbb{P}_YV\to Y$ be the projection map.  Note $\pi$ is flat. Given a point $x\in X$, by lemma \ref{semipos=nef} the vector bunlde $V_x$ is semipositive if and only if $\mathcal{O}_{Proj V}(1)_x$ is nef. The compositve morphism $f\circ \pi$ is flat. Since nefness is an open condition in flat families, the set of points $x\in X$ such that $\mathcal{O}_{\mathbb{P}V}(1)$ is nef is open in $X$.
\end{proof}

\begin{lemma}
Let $X$ be a scheme, and let $V_i$, $i=1,2,\ldots,n$ be semipositive vector bundles of rank $r_i$ over a scheme $X$. Let $\rho:\prod_i GL(r_i)\to GL(m)$ be a semipositive representation, and let $W=\rho(V_1,\ldots,V_n)$. 
\end{lemma}

\begin{proof}
should probably have auxiliary lemmas
\end{proof}

\begin{lemma}
Let $X$ be a scheme, and let $V_i$, $i=1,2,\ldots,n$ be semipositive vector bundles of rank $r_i$ over a scheme $X$. Let $\rho:\prod_i GL(r_i)\to GL(m)$ be a semipositive representation, and let $W=\rho(V_1,\ldots,V_n)$. Let $V$ be the subbundle of $W$ associated to a subrepresentation of $\rho$. Then, if $W$ is semipositive, then $V$ is semipositive.
\end{lemma}

\begin{proof}

\end{proof}



\bibliographystyle{unsrt}
\bibliography{references}

\end{document}

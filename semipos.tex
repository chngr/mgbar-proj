\input{preamble}


\newcommand{\todo}[1]{\footnote{\textbf{TODO.} #1}}

\begin{document}
\title{Semipositivity}
\maketitle


\section{Introduction}
This section develops the basic properties of semipositivity.


\begin{definition}
Let $G=\prod_i GL(E_i)$ be a product of general linear groups. A representation $\rho:G\rightarrow GL(F)$ is called semipositive (or polynomial) if it extends to a morphism $\overline{\rho}:\prod_i End(E_i)\rightarrow End(F)$.
\end{definition}

\todo{give notation for $H$-bundle you get from a $G$-bundle and a map $G\to H$}.

\begin{definition}
Let $V_1,\dots, V_n$ vector bundles of rank $r_1,\dots, r_n$ over a base space X, and let $\rho:\prod_i GL(r_i)\rightarrow GL(m)$ be any representation. One can define a bundle $\rho(V_1,\dots, V_n)$ of rank $m$ as follows: every $V_i$ is a principal $GL(r_i)$-bundle, hence it corresponds to a point $x_i$ in $H^1(X, GL(r_i))$; the map $\rho$ induces a map $\rho^*:H^1(X,\prod_i GL(r_i)\rightarrow H^1(X, GL(m))$, i.e. $\rho: \prod_i H^1(X,GL(r_i))\rightarrow H^1(X, GL(m))$. We define $\rho(V_1,\dots, V_n)$ to be the principal $GL(m)$-bundle corresponding to $\rho^*(x_1,\dots,x_n)$.
\end{definition}

\begin{lemma}
For $i=1,2,\ldots,n$, let $V_i=\sum_{j}\mathcal{L}_{ij}$ be a vector bundle of rank $r_i$ that splits as a direct sum of line bundles $\mathcal{L}_{ij}$. Let $\rho:\prod_{i}GL(r_i)\to GL(m)$ be a semipositive representation. Then, $\rho(V_1,\ldots,V_n)$ is a a direct sum of line bundles of the form $\oplus_{i,j}\otimes\mathcal{L}_{ij}^{a_{ij}}$, where $a_{ij}\ge0$.
\end{lemma}

\begin{proof}
\todo{add details, maybe}
For each $i$, we have a diagonal embedding $\psi_i:\prod_{j=1}^{r_i}GL(1)\to GL(r_i)$...
Taking the product over $i$ and post-composing with $\rho$ yields a map
\begin{equation}
\prod_{i=1}^{n}\prod_{j=1}^{r_i}GL(1)\to GL(m),
\end{equation}
which factors through a maximal torus $\psi:\prod_{k=1}^{m}GL(1)\to GL(m)$.
\todo{make diagram...}

Thus $\rho(V_1,\ldots,V_n)$ splits as the direct sum of the line bundles you get from $\prod\prod GL(1)\to GL(1)$ after projecting to the $t$-th factor...

This map is a character, so of the form $(z_{ij})\mapsto \prod z_{ij}^{a_{ij,t}}$, and $a_{ij,t}\ge0$ by semipositivity.

Then $\rho(V_1,\ldots,V_n)$ is the direct sum of $\otimes\mathcal{L}_{ij}^{a_{ij,t}}$ over $t\in[1,n]$.

\end{proof}

\begin{lemma}
The following properties hold for semipositive representations:
\item[(i)] Symmetric products are semipositive;
\item[(ii)] Subrepresentations of semipositive representations are semipositive;
\item[(iii)] Let $V_1,\dots, V_n$ be vector bundles such that each $V_i$ is a sum of line bundles $\sum_j L_{ij}$, and $\rho$ a semipositive respresentation. Then the structure group of $\rho(V_1,\dots, V_n)$ is a torus and is a direct sum of line bundles $\otimes L_{ij}^{a_{ij}}$, where each of the $a_{ij}$ are nonnegative.
\item[(iv)] Let $V_1,\dots, V_n$ and $W_1,\dots, W_n$ be vector bundles such that $rk (V_i)=rk(W_i)$, and let $f_i: V_i\rightarrow W_i$ be sheaf homomorphisms. If $\rho$ is semipositive, then there exists a natural map 
$$\rho(f_1,\dots, f_n):\rho(V_1,\dots, V_n)\rightarrow \rho(W_1,\dots,W_n).$$
\item[(v)] Let $V_1,\dots, V_n$ be vector bundles over a base $X$, and $f:X'\rightarrow X$ a morphism. Then $\rho(f^*V_1,\dots,f^*V_n)=f^*(\rho(V_1,\dots,V_n))$.
\end{lemma}


\begin{proof}

\end{proof}


\begin{definition}
A locally free sheaf $V$ on a scheme $X$ is semipositive if for every map from a proper curve $f:C\to X$ every quotient line bundle of $f^*V$ has nonnegative degree.
\end{definition}

\begin{lemma}
A locally free sheaf $V$ on a scheme $X$ is semipositive if and only if $O_{Proj V}(1)$ is nef on $Proj_XV$.
\end{lemma}
\begin{proof}

\end{proof}

\begin{lemma}
3.4 Semipositivity is an open condition in flat families.
\end{lemma}

\begin{lemma}\label{apply_rho_still_semipos}
Let $X$ be a scheme, and let $V_i$, $i=1,2,\ldots,n$ be semipositive vector bundles of rank $r_i$ on $X$. Let $\rho:\prod_i GL(r_i)\to GL(m)$ be a semipositive representation, and let $W=\rho(V_1,\ldots,V_n)$. Then, $W$ is a semipositive vector bundle.
\end{lemma}

\begin{proof}
should probably have auxiliary lemmas
\end{proof}

\begin{lemma}
Let $X$ be a scheme and $V$ a semipositive vector bundle on $X$. Then, $Sym^dV$ is a semipositive vector bundle.
\end{lemma}

\begin{proof}
Immediate from and Lemma \todo{sym is semipos} and Lemma \ref{apply_rho_still_semipos}.
\end{proof}

\begin{lemma}
Let $X$ be a scheme, and let $V_i$, $i=1,2,\ldots,n$ be semipositive vector bundles of rank $r_i$. Let $G=\prod_{i=1}^{n}GL(r_i)$, let $\rho:G\to GL(m)$ be a semipositive representation, and let $W=\rho(V_1,\ldots,V_n)$. Let $x\in X$ be a point, and let $V_x$ be a $G$-invariant subspace of $W_x$, which defines a subbundle $V\subset W$. Then, $V$ is a semipositive vector bundle.
\end{lemma}

\begin{proof}
The $G$-invariant subspace $V_x\subset W_x$ defines a subrepresentation $\sigma$ of $\rho$, and we see that $V=\sigma(V_1,\ldots,V_n)$. By \todo{ref}, $\sigma$ is semipositive, so by Lemma \ref{apply_rho_still_semipos}, $V$ is semipositive.
\end{proof}



\bibliographystyle{unsrt}
\bibliography{references}

\end{document}

\IfFileExists{stacks-project.cls}{%
\documentclass{stacks-project}
}{%
\documentclass{amsart}
}

% The following AMS packages are automatically loaded with
% the amsart documentclass:
%\usepackage{amsmath}
%\usepackage{amssymb}
%\usepackage{amsthm}

% For dealing with references we use the comment environment
\usepackage{verbatim}
\newenvironment{reference}{\comment}{\endcomment}
%\newenvironment{reference}{}{}
\newenvironment{slogan}{\comment}{\endcomment}
\newenvironment{history}{\comment}{\endcomment}

% For commutative diagrams you can use
% \usepackage{amscd}
\usepackage[all]{xy}

% We use 2cell for 2-commutative diagrams.
\xyoption{2cell}
\UseAllTwocells

% To put source file link in headers.
% Change "template.tex" to "this_filename.tex"
% \usepackage{fancyhdr}
% \pagestyle{fancy}
% \lhead{}
% \chead{}
% \rhead{Source file: \url{template.tex}}
% \lfoot{}
% \cfoot{\thepage}
% \rfoot{}
% \renewcommand{\headrulewidth}{0pt}
% \renewcommand{\footrulewidth}{0pt}
% \renewcommand{\headheight}{12pt}

\usepackage{multicol}

% For cross-file-references
\usepackage{xr-hyper}

% Package for hypertext links:
\usepackage{hyperref}

% For any local file, say "hello.tex" you want to link to please
% use \externaldocument[hello-]{hello}
\externaldocument[introduction-]{introduction}
\externaldocument[conventions-]{conventions}
\externaldocument[sets-]{sets}
\externaldocument[categories-]{categories}
\externaldocument[topology-]{topology}
\externaldocument[sheaves-]{sheaves}
\externaldocument[sites-]{sites}
\externaldocument[stacks-]{stacks}
\externaldocument[fields-]{fields}
\externaldocument[algebra-]{algebra}
\externaldocument[brauer-]{brauer}
\externaldocument[homology-]{homology}
\externaldocument[derived-]{derived}
\externaldocument[simplicial-]{simplicial}
\externaldocument[more-algebra-]{more-algebra}
\externaldocument[smoothing-]{smoothing}
\externaldocument[modules-]{modules}
\externaldocument[sites-modules-]{sites-modules}
\externaldocument[injectives-]{injectives}
\externaldocument[cohomology-]{cohomology}
\externaldocument[sites-cohomology-]{sites-cohomology}
\externaldocument[dga-]{dga}
\externaldocument[dpa-]{dpa}
\externaldocument[hypercovering-]{hypercovering}
\externaldocument[schemes-]{schemes}
\externaldocument[constructions-]{constructions}
\externaldocument[properties-]{properties}
\externaldocument[morphisms-]{morphisms}
\externaldocument[coherent-]{coherent}
\externaldocument[divisors-]{divisors}
\externaldocument[limits-]{limits}
\externaldocument[varieties-]{varieties}
\externaldocument[topologies-]{topologies}
\externaldocument[descent-]{descent}
\externaldocument[perfect-]{perfect}
\externaldocument[more-morphisms-]{more-morphisms}
\externaldocument[flat-]{flat}
\externaldocument[groupoids-]{groupoids}
\externaldocument[more-groupoids-]{more-groupoids}
\externaldocument[etale-]{etale}
\externaldocument[chow-]{chow}
\externaldocument[intersection-]{intersection}
\externaldocument[pic-]{pic}
\externaldocument[adequate-]{adequate}
\externaldocument[dualizing-]{dualizing}
\externaldocument[duality-]{duality}
\externaldocument[discriminant-]{discriminant}
\externaldocument[local-cohomology-]{local-cohomology}
\externaldocument[curves-]{curves}
\externaldocument[resolve-]{resolve}
\externaldocument[models-]{models}
\externaldocument[pione-]{pione}
\externaldocument[etale-cohomology-]{etale-cohomology}
\externaldocument[proetale-]{proetale}
\externaldocument[crystalline-]{crystalline}
\externaldocument[spaces-]{spaces}
\externaldocument[spaces-properties-]{spaces-properties}
\externaldocument[spaces-morphisms-]{spaces-morphisms}
\externaldocument[decent-spaces-]{decent-spaces}
\externaldocument[spaces-cohomology-]{spaces-cohomology}
\externaldocument[spaces-limits-]{spaces-limits}
\externaldocument[spaces-divisors-]{spaces-divisors}
\externaldocument[spaces-over-fields-]{spaces-over-fields}
\externaldocument[spaces-topologies-]{spaces-topologies}
\externaldocument[spaces-descent-]{spaces-descent}
\externaldocument[spaces-perfect-]{spaces-perfect}
\externaldocument[spaces-more-morphisms-]{spaces-more-morphisms}
\externaldocument[spaces-flat-]{spaces-flat}
\externaldocument[spaces-groupoids-]{spaces-groupoids}
\externaldocument[spaces-more-groupoids-]{spaces-more-groupoids}
\externaldocument[bootstrap-]{bootstrap}
\externaldocument[spaces-pushouts-]{spaces-pushouts}
\externaldocument[groupoids-quotients-]{groupoids-quotients}
\externaldocument[spaces-more-cohomology-]{spaces-more-cohomology}
\externaldocument[spaces-simplicial-]{spaces-simplicial}
\externaldocument[spaces-duality-]{spaces-duality}
\externaldocument[formal-spaces-]{formal-spaces}
\externaldocument[restricted-]{restricted}
\externaldocument[spaces-resolve-]{spaces-resolve}
\externaldocument[formal-defos-]{formal-defos}
\externaldocument[defos-]{defos}
\externaldocument[cotangent-]{cotangent}
\externaldocument[examples-defos-]{examples-defos}
\externaldocument[algebraic-]{algebraic}
\externaldocument[examples-stacks-]{examples-stacks}
\externaldocument[stacks-sheaves-]{stacks-sheaves}
\externaldocument[criteria-]{criteria}
\externaldocument[artin-]{artin}
\externaldocument[quot-]{quot}
\externaldocument[stacks-properties-]{stacks-properties}
\externaldocument[stacks-morphisms-]{stacks-morphisms}
\externaldocument[stacks-limits-]{stacks-limits}
\externaldocument[stacks-cohomology-]{stacks-cohomology}
\externaldocument[stacks-perfect-]{stacks-perfect}
\externaldocument[stacks-introduction-]{stacks-introduction}
\externaldocument[stacks-more-morphisms-]{stacks-more-morphisms}
\externaldocument[stacks-geometry-]{stacks-geometry}
\externaldocument[moduli-]{moduli}
\externaldocument[moduli-curves-]{moduli-curves}
\externaldocument[examples-]{examples}
\externaldocument[exercises-]{exercises}
\externaldocument[guide-]{guide}
\externaldocument[desirables-]{desirables}
\externaldocument[coding-]{coding}
\externaldocument[obsolete-]{obsolete}
\externaldocument[fdl-]{fdl}
\externaldocument[index-]{index}

% Theorem environments.
%
\theoremstyle{plain}
\newtheorem{theorem}[subsection]{Theorem}
\newtheorem{proposition}[subsection]{Proposition}
\newtheorem{lemma}[subsection]{Lemma}

\theoremstyle{definition}
\newtheorem{definition}[subsection]{Definition}
\newtheorem{example}[subsection]{Example}
\newtheorem{exercise}[subsection]{Exercise}
\newtheorem{situation}[subsection]{Situation}

\theoremstyle{remark}
\newtheorem{remark}[subsection]{Remark}
\newtheorem{remarks}[subsection]{Remarks}

\numberwithin{equation}{subsection}

% Macros
%
\def\lim{\mathop{\rm lim}\nolimits}
\def\colim{\mathop{\rm colim}\nolimits}
\def\Spec{\mathop{\rm Spec}}
\def\Hom{\mathop{\rm Hom}\nolimits}
\def\Ext{\mathop{\rm Ext}\nolimits}
\def\SheafHom{\mathop{\mathcal{H}\!{\it om}}\nolimits}
\def\SheafExt{\mathop{\mathcal{E}\!{\it xt}}\nolimits}
\def\Sch{\textit{Sch}}
\def\Mor{\mathop{\rm Mor}\nolimits}
\def\Ob{\mathop{\rm Ob}\nolimits}
\def\Sh{\mathop{\textit{Sh}}\nolimits}
\def\NL{\mathop{N\!L}\nolimits}
\def\proetale{{pro\text{-}\acute{e}tale}}
\def\etale{{\acute{e}tale}}
\def\QCoh{\textit{QCoh}}
\def\Ker{\mathop{\rm Ker}}
\def\Im{\mathop{\rm Im}}
\def\Coker{\mathop{\rm Coker}}
\def\Coim{\mathop{\rm Coim}}

%
% Macros for moduli stacks/spaces
%
\def\QCohstack{\mathcal{QC}\!{\it oh}}
\def\Cohstack{\mathcal{C}\!{\it oh}}
\def\Spacesstack{\mathcal{S}\!{\it paces}}
\def\Quotfunctor{{\rm Quot}}
\def\Hilbfunctor{{\rm Hilb}}
\def\Curvesstack{\mathcal{C}\!{\it urves}}
\def\Polarizedstack{\mathcal{P}\!{\it olarized}}
\def\Complexesstack{\mathcal{C}\!{\it omplexes}}
% \Pic is the operator that assigns to X its picard group, usage \Pic(X)
% \Picardstack_{X/B} denotes the Picard stack of X over B
% \Picardfunctor_{X/B} denotes the Picard functor of X over B
\def\Pic{\mathop{\rm Pic}\nolimits}
\def\Picardstack{\mathcal{P}\!{\it ic}}
\def\Picardfunctor{{\rm Pic}}
\def\Deformationcategory{\mathcal{D}\!{\it ef}}



\newcommand{\todo}[1]{\footnote{\textbf{TODO.} #1}}

\begin{document}
\title{Semipositivity}
\maketitle

\section{Introduction}
This section develops the basic properties of semipositivity.

\begin{remark}
We will work over a field throughout. One can extend the discussion to a more general situation, where we have a principal $\mathcal{G}$-bundle over a scheme $X$, where $\mathcal{G}$ is a product of constant group schemes associated to general linear groups, and a representation $\rho:\mathcal{G}\to\mathcal{GL}(m)$, where $\mathcal{GL}(m)$ is the constant group scheme associated to $\mathrm{GL}(m)$. However, we are unaware of applications that need this level of generality, so we stick to $X$ over a field and $\rho$ a constant representation.

It's possible we need finiteness conditions on $X$, for example, to make openness of nef locus work.
\end{remark}

\begin{remark}
This discussion should eventually be extended to algebraic spaces over a field to make the application to $\overline{M}_g$ work, but all of the statements should go through easily.
\end{remark}

\section{Semipositivity for representations}

\begin{definition}\label{semipos_rep_def}
Let $k$ be a field.
Let $V_1,\ldots,V_n,W$ be finite-dimensional vector spaces over $k$.
Let $G=\prod_{i=1}^{n}=\mathrm{GL}(V_i)$.
A representation $\rho:G\rightarrow \mathrm{GL}(W)$ is called \textit{semipositive} if it extends to a morphism of monoids $\overline{\rho}:\prod_i \mathrm{End}(V_i)\rightarrow\mathrm{End}(W)$.
\end{definition}

\begin{lemma}\label{semipos_rep_examples}
\begin{enumerate}
\item\label{sym_and_wedge_semipos} Let $k$ be a field. 
Let $V$ be a finite-dimensional vector space over $k$ and let $d$ be a positive integer. Then, the natural representations $\mathrm{GL}(V)\to \mathrm{GL}(\mathrm{Sym}^dV)$ and $\mathrm{GL}(V)\to \mathrm{GL}(\Lambda^dV)$ are semipositive.
\item\label{subreps_and_quotients_semipos} Subrepresentations and quotients of semipositive representations are semipositive.
\item The direct sum of finitely many semipositive representations is semipositive.
\item 
Let $k$ be a field. 
Let $V_1,\ldots,V_n,W$ be finite-dimensional vector spaces over $k$.
Let $\rho:\prod_{i=1}^{n}\mathrm{GL}(V_i)\rightarrow \mathrm{GL}(W)$ be a semipositive representation. 
Let $T_V$ denote the antihomomorphism of monoids $\prod_i\mathrm{GL}(V_i^*)\to\prod\mathrm{GL}(V_i)$ taking a automorphism of $V_i$ to the its dual.
Similarly, let $T_W$ denote the antihomomorphism of monoids $\mathrm{GL}(W)\to\mathrm{GL}(W^*)$.
Then, $T_W\rho T_V$ is a semipositive representation.
\end{enumerate}
\end{lemma}

\begin{proof}
\begin{enumerate}
\item Omitted.
\item Let $\rho:G\to \mathrm{GL}(W)$ be a semipositive representation and let $W'\subset W$ be a subrepresentation. 
We may easily reduce to the case $G=\mathrm{GL}(V)$.
Then, $\rho$ extends to a morphism of monoids $\overline{\rho}:\mathrm{End}(V)\to \mathrm{End}(W)$. 
The representation $\rho$ also gives a morphism of schemes $\phi:\mathrm{GL}(V)\times W\to W$, where $W$ has been given the obvious scheme structure. 
Similarly, $\overline{\rho}$ gives rise to an extension $\overline{\phi}:\mathrm{End}(V)\times W\to W$. 

The fact that $W'\subset W$ is a subrepresentation is equivalent to the fact that the image of $\phi$ when restricted to the closed subscheme $\mathrm{GL}(V)\times W'$ is $W'$. 
It is sufficient to prove the same statement for the closed subscheme $\mathrm{End}(V)\times W'$ of $\mathrm{End}(V)\times W$. 
However, by assumption the dense open subscheme $\mathrm{GL}(V)\times W'$ of $\mathrm{End}(V)\times W'$ maps to $W'$, so $\mathrm{End}(V)\times W'$ must as well, as desired. One should check that the resulting map on $W'$ is linear; details omitted.                                                                                                                                                                                                                                  

The case of quotients of semipositive representations is proved similarly.
\item Omitted.
\item Omitted.
\end{enumerate}
\end{proof}


\begin{definition}
Let $k$ be a field.
Let $X$ be a $k$-scheme.
Let $f:G\to H$ be a morphism of $k$-groups, and let $\pi:P\to X$ be a principal $G$-bundle corresponding to a class $\pi\in H^1(X,G)$. 
Let $f^{*}:H^1(X,G)\to H^1(X,H)$ be the induced map on cohomology. 
Then, we denote the principal $H$-bundle corresponding to the class $f^{*}(\pi)$ by $\pi_f:P\times_{G}H\to X$.
\end{definition}

\begin{definition}
Let $k$ be a field.
Let $X$ be a $k$-scheme. 
Let $V_1,\ldots,V_n$ locally free sheaves of finite ranks $r_1,\dots, r_n$, respectively.
Let $G=\prod_i \mathrm{GL}(r_i)$.
Let $\rho:G\rightarrow \mathrm{GL}(m)$ be a representation. 
Let $\pi':V'_1\times\cdots\times V'_n\to X$ be the principal $G$-bundle obtained by taking the product of the principal $\mathrm{GL}(r_i)$-bundles associated to the $V_i$.
Then, we denote the locally free sheaf associated to the principal $\mathrm{GL}(m)$-bundle $\pi'_{\rho}$ by $\rho(V_1,\dots, V_n)$.
\end{definition}

\begin{lemma}\label{pullback_and_rho_commute}
Let $X$ be a scheme.
Let $\mathcal{V}_1,\ldots,\mathcal{V}_n$ be locally free sheaves on $X$. Let $f:X'\rightarrow X$ a morphism of schemes.
Then $\rho(f^*\mathcal{V}_1,\dots,f^*\mathcal{V}_n)$ and $f^*(\rho(\mathcal{V}'_1,\dots,\mathcal{V}'_n))$ are canonically isomorphic as locally free sheaves.
\end{lemma}

\begin{proof}
Clear from the construction. Details omitted.
\end{proof}


\begin{lemma}\label{functoriality_rho}
Let $k$ be a field.
Let $X$ be a $k$-scheme.
Let $\mathcal{V}_1,\dots, \mathcal{V}_n$ and $\mathcal{V}'_1,\dots,\mathcal{V}'_n$ be locally free sheaves of finite rank on $X$.
Assume that for each $i$, $\mathrm{rank}(\mathcal{V}_i)=\mathrm{rank}(\mathcal{V}'_i)=r_i$.
For each $i$, let $f_i:\mathcal{V}_i\rightarrow \mathcal{V}'_i$ be a morphism of sheaves.
Let $\rho:\prod\mathrm{GL}(r_i)\to\mathrm{GL}(m)$ be a representation.
If $\rho$ is semipositive, then there exists a morphism of sheaves
$$
\rho(f_1,\dots, f_n):\rho(\mathcal{V}_1,\ldots,\mathcal{V}_n)\rightarrow \rho(\mathcal{V}'_1,\dots,\mathcal{V}'_n)
$$
\end{lemma}

\begin{proof}
After choosing an open cover of $X$ trivializing all of the locally free sheaves in question, the maps $f_i$ are described by $r_i\times r_i$ matrices. The images of these matrices under a choice of $\overline{\rho}$ as in Definition \ref{semipos_ref_def} define the map $\rho(f_1,\ldots,f_n)$, and the fact that $\rho$ is a monoid homomorphism proves that it is a morphism of sheaves. Details omitted.
\end{proof}








\begin{lemma}\label{rho_of_split_bundle}
Let $k$ be a field.
Let $X$ be a $k$-scheme.
For $i=1,2,\ldots,n$, let $V_i=\oplus_{j}\mathcal{L}_{ij}$ be a locally free sheaf of rank $r_i$ that splits as a direct sum of line bundles $\mathcal{L}_{ij}$.
Let $k$ be a field.
Let $\rho:\prod_{i}\mathrm{GL}(r_i)\to \mathrm{GL}(m)$ be a semipositive representation. 
Then, $\rho(\mathcal{V}_1,\ldots,V_n)$ is a a direct sum of line bundles of the form $\oplus_{i,j}\otimes\mathcal{L}_{ij}^{a_{ij}}$, where $a_{ij}\ge0$ for each $i,j$.
\end{lemma}

\begin{proof}
For each $i$, we have a diagonal embedding $\psi_i:\prod_{j=1}^{r_i}\mathrm{GL}(1)\to \mathrm{GL}(r_i)$.
Taking the product over $i$ and post-composing with $\rho$ yields a map
$$
\psi:\prod_{i=1}^{n}\prod_{j=1}^{r_i}\mathrm{GL}(1)\to \mathrm{GL}(m),
$$
which factors through a maximal torus $\prod_{k=1}^{m}\mathrm{GL}(1)\subset \mathrm{GL}(m)$, as in the following diagram:
$$
\xymatrix{
\prod_{i=1}^{n}\prod_{j=1}^{r_i} \mathrm{GL}(1)\ar[r]^{\psi} \ar[rd] &\prod_{i=1}^{n}\mathrm{GL}(r_i)\ar[r]^{\rho} &\mathrm{GL}(m)\\
&\prod_{k=1}^{m}\mathrm{GL}(1)\ar[ru]\ar[d]^{\mathrm{pr}_t}&\\
&\mathrm{GL}(1)&
}
$$


Now consider the projection on the $t$-th factor $\mathrm{pr}_t:\prod_{k=1}^m\mathrm{GL}(1)\to \mathrm{GL}(1)$. The composition $\prod_{i=1}^n\prod_{j=1}^{r_i} \mathrm{GL}(1)\to \mathrm{GL}(1)$ is a character, is of the form $(z_{ij})_{i,j}\mapsto \prod z_{ij}^{a_{ij,t}}$. On a trivializing cover of $X$, the $z_{ij}$ correspond to the transition functions of $\mathcal{L}_{ij}$, hence $\prod z_{ij}^{a_{ij,t}}$ correspond to $\otimes\mathcal{L}_{ij}^{a_{ij,t}}$.

Pulling back to $\prod_{k=1}^{m} \mathrm{GL}(m)$, we see that $\rho(V_1,\ldots,V_n)$ splits as the direct sum of the line bundles $\oplus_t \otimes_{i,j}\mathcal{L}_{ij}^{a_{ij,t}}$.
\end{proof}

\section{Semipositivity for locally free sheaves}

\begin{definition}\label{def_semipos_bundle}
Let $k$ be a field.
Let $X$ be a $k$-scheme.
Let $V$ be a locally free sheaf of finite rank on $X$.
We say that $V$ is \textit{semipositive} if for every map from a proper and smooth $k$-scheme of dimension 1 $C$, invertible sheaf $\mathcal{L}$ on $C$, $k$-morphism $f:C\to X$, and surjection $f^{*}V\to\mathcal{L}$, we have $\deg_C\mathcal{L}\ge0$.\end{definition}


\begin{lemma}
Quotients and extensions of semipositive vector bundles are semipositive.
\end{lemma}

\begin{proof}
Immediate from the right-exactness of pullback, details omitted.
\end{proof}


\begin{lemma}\label{pullback of semipos}
Let $V$ be a semipositive vector bundle on a scheme $X$. For any morphism $g:Y\to X$, the pullback $f^*V$ is semipositive vector bundle on $Y$.
\end{lemma}
\begin{proof}
Consider any morphism $f:C\to Y$, as in Definition \ref{def_semipos_bundle}. 
Then, as $f^*(g^*V)=(g\circ f)^*V$ and $V$ is semipositive, any quotient invertible sheaf of $g^*(f^*V)$ has non-negative degree. 
Hence, $f^*V$ is semipositive.
\end{proof}

\begin{lemma}\label{semipos=nef}
Let $X$ be a scheme.
Let $\mathcal{V}$ be a locally free sheaf on $X$.
Then, $\mathcal{V}$ is semipositive if and only if $\mathcal{O}_{\mathbb{P}_X(V)}(1)$ is nef on $\mathbb{P}_X(V)$.
\end{lemma}

\begin{proof}
Let $\pi:\mathbb{P}_X(V)\to V$ be the projection map. 
Assume $\mathcal{O}_{\mathbb{P}_X(V)}(1)$ is nef.
Let $f:C\to X$ be a morphism as in Definition \label{def_semipos_bundle}.
Let $\mathcal{L}$ be an invertible quotient of $f^*V$ on $C$. 
Then, $\deg \mathcal{O}_{\mathbb{P}L}(1)=\deg L\ge0$, so $V$ is semipositive.

For the other direction, consider a $k$-morphism $g:C\to\mathbb{P}_X(V)$, where $C$ is a smooth proper $k$-scheme of dimension 1.
By the universal property of relative Proj, $\mathcal{O}_{\mathbb{P}V}(1)|_C$ is a quotient of $\pi^*V|_C$. 
By Lemma \ref{pullback of semipos}, $\pi^*V$ is semipositive, so $\deg\mathcal{O}_{\mathbb{P}V}(1)|_C\geq 0$.
It follows that $\mathcal{O}_{\mathrm{Proj} V}(1)$ is nef.
\end{proof}



\begin{lemma}\label{semipos_locus_open}
Let $X$ and $Y$ be schemes.
Let $f:Y\to X$ be a flat morphism of schemes.
Let $\mathcal{V}$ be a locally free sheaf of finite rank on $Y$.
Then, the set of points $x\in X$ such that the locally free sheaf $\mathcal{V}_x$ is semipositive on the fiber $Y_x$ is open.
\end{lemma}

\begin{proof}
Let $\pi:\mathbb{P}_Y(V)\to Y$ be the projection map.
Given a point $x\in X$, by Lemma \ref{semipos=nef} the vector bundle $\mathcal{V}_x$ is semipositive if and only if $\mathcal{O}_{\mathbb{P}_Y(V)}(1)_x$ is nef. 
The compositve morphism $f\circ \pi$ is flat, because $\pi$ is flat.
The claim now follows from the fact that nefness is an open condition in flat families.
\end{proof}

\begin{lemma}\label{global_generation_of_twist_on_curve}]
Let $k$ be a field.
Let $C$ be a smooth, proper, and connected $k$-scheme of of genus $g$.
Let $\mathcal{E}$ be a vector bundle on $C$. 
Let $\mathcal{L}$ be a line bundle on $C$ of degree at least $2g$. 
Then, $\mathcal{E}\otimes\mathcal{L}$ is globally generated.
\end{lemma}

\begin{proof}
By Tag OB57, it suffices to assume that $k$ is algebraically closed. 
Let $x\in C$ be a closed point with ideal sheaf $\mathcal{O}(-x)$. 
We have a short exact sequence
\begin{equation}
0\to\mathcal{E}\otimes\mathcal{L}(-x)\to\mathcal{E}\otimes\mathcal{L}\to\mathcal{E}\otimes\mathcal{L}\mid_x\to0
\end{equation}
It suffices to show that the latter map $\mathcal{E}\otimes\mathcal{L}\to\mathcal{E}\otimes\mathcal{L}\mid_x$ remains surjective after applying $H^0$, and thus to show that $H^1(C,\mathcal{E}\otimes\mathcal{L}(-x))=0$.

By Serre Duality, it suffices to show that $\Hom_{\mathcal{O}_C}(\mathcal{E}\otimes\mathcal{L}(-x),\omega_C)=0$, which is equivalent to $\Hom(\mathcal{E},\omega_C\otimes\mathcal{L}^{-1}(x))=0$.

The target of such a nonzero map is a locally free sheaf of negative degree, so the image subsheaf is torsion-free and hence locally free sheaf of negative degree. However, $\mathcal{E}$ has no negative line bundle quotients, so no such map can exist.
\end{proof}



\begin{lemma}\label{no_negative_quotient_on_curve_p}
Let $k$ be a field of characteristic $p>0$.
Let $C$ be a smooth, proper, and connected $k$-scheme of dimension 1. 
Let $\mathcal{V}_1,\ldots,\mathcal{V}_n$ be semipositive vector bundles of ranks $r_1,\ldots,r_n$ on $C$. 
Let $\rho$ be a semipositive representation $\prod \mathrm{GL}(r_i)\to \mathrm{GL}(m)$. 
Then, $\rho(\mathcal{V}_1,\ldots,\mathcal{V}_n)$ is a semipositive vector bundle.
\end{lemma}
\begin{proof}
Let $g$ be the genus of $C$, and fix a line bundle $\mathcal{L}$ of degree at least $2g$.
By Lemma \ref{global_generation_of_twist_on_curve}, $\mathcal{V}_i\otimes\mathcal{L}$ is generated by global sections for each $i$, so there exist maps
\begin{equation}
f_i:(\mathcal{L}^{-1})^{\oplus r_i}\to \mathcal{V}_i,
\end{equation}
which are surjective at a given closed point of $x\in C$, and hence generically surjective.

Applying Lemma \ref{pullback_and_rho_commute}, Lemma \ref{rho_of_split_bundle}, and Lemma \ref{functoriality_rho}, we get a generaically map
\begin{equation}
\rho(f_1,\ldots,f_n):\bigoplus_{i}\mathcal{L}^{-\otimes b_i}\to\rho(\mathcal{V}_1,\ldots,\mathcal{V}_n),
\end{equation}
where the source is $\rho(\mathcal{L}^{-\otimes r_1},\ldots,\mathcal{L}^{-\otimes r_n})$.
Thus, the $b_i$ are positive integers depending on the ranks of the $V_i$ but not the $V_i$ themselves.
In particular, there exists a positive integer $N$ such that $\min\deg\mathcal{L}^{-b_i}=-N$, where $N$ does not depend on the $\mathcal{V}_i$. Moreover, $\rho(f_1,\ldots,f_n)$ is surjective at $x$ by the construction in Lemma \ref{functoriality_rho}, as the image of an $r$-tuple of invertible matrices is invertible under $\overline{\rho}$.

Suppose $\mathcal{M}$ is an invertible quotient of $\rho(V_1,\ldots,V_n)$. By pre-composing with $\rho(f_1,\ldots,f_n)$, we obtain a nonzero map
$$
\bigoplus_{i}\mathcal{L}^{-\otimes b_i}\to\mathcal{M}.
$$
Thus, we conclude that
\begin{equation}\label{degree_of_quotient_not_too_small}
\deg\mathcal{M}\ge -N.
\end{equation}

Finally, suppose that $\rho(V_1,\ldots,V_n)$ has a line bundle quotient $\mathcal{N}$ of degree $d<0$. 
Let $F:C\to C$ be the absolute Frobenius map on $C$. 
Then, by pulling back the surjection $\rho(V_1,\ldots,V_n)\to\mathcal{N}$ by $F^t$, we obtain a quotient line bundle of $\rho((F^{t})^{*}V_1,\ldots,(F^{t})^{*}V_n)$ of degree $dp^{t}<-N$ for sufficiently large $t$, by Lemma \ref{pullback_and_rho_commute}
However, as $(F^{t})^*V_i$ is semipositive for each $i$, this contradicts (\ref{degree_of_quotient_not_too_small}), so hence $\mathcal{N}$ cannot exist.
\end{proof}

\begin{lemma}\label{no_negative_quotient_on_curve}
Let $k$ be a field.
Let $C$ be a smooth, proper, and connected $k$-scheme of dimension 1. 
Let $\mathcal{V}_1,\ldots,\mathcal{V}_n$ be semipositive vector bundles of ranks $r_1,\ldots,r_n$ on $C$. 
Let $\rho$ be a semipositive representation $\prod \mathrm{GL}(r_i)\to \mathrm{GL}(m)$. 
Then, $\rho(\mathcal{V}_1,\ldots,\mathcal{V}_n)$ is a semipositive vector bundle.
\end{lemma}
\begin{proof}
If the characteristic of $k$ is positive, this is Lemma \ref{no_negative_quotient_on_curve_p}.
If the characteristic of $k$ is zero, after possibly enlarging $k$, we may find an integral and finite type $\mathbb{Z}$-algebra $A$ over which $C$, $\mathcal{V}_1,\ldots,\mathcal{V}_n$, and $\rho$ are all defined, and such that we have a smooth morphism of schemes $\pi:\mathcal{C}\to\Spec A$. 
By Lemma \ref{semipos_locus_open}, we may replace $\Spec A$ with an open subset so that each $\mathcal{V}_i$ is semipositive when restricted to any fiber of $\pi$. 
By Lemma \label{no_negative_quotient_on_curve_p}, $\mathcal{W}=\rho(\mathcal{V}_1,\ldots,\mathcal{V}_n)$ is semipositive when restricted to any fiber of $\pi$ above a point of $\Spec A$ with residue field of positive characteristic.
This locus on $\Spec A$ is non-empty, so by another application of Lemma \ref{semipos_locus_open}, so $\mathcal{W}=\rho(\mathcal{V}_1,\ldots,\mathcal{V}_n)$ is semipositive when restricted to the generic fiber, which is what was needed.
Some additional details may be necessary.\todo{Does this even work??}


 we can construct a flat morphism of schemes $\pi:\mathcal{C}\to\Spec A$, where $A$ is a finite-type $\mathbb{Z}$-algebra with $K(A)\cong k$, such that $\mathcal{C}_k\cong C$, along with vector bundles 
\end{proof}


\begin{lemma}\label{apply_rho_still_semipos}
Let $X$ be a scheme
Let $\mathcal{V}_1,\ldots,\mathcal{V}_n$ be semipositive vector bundles of finite ranks $r_1,\ldots,r_n$ on $X$. 
Let $\rho:\prod_i \mathrm{GL}(r_i)\to \mathrm{GL}(m)$ be a semipositive representation, and let $W=\rho(V_1,\ldots,V_n)$. 
Then, $W$ is a semipositive vector bundle.
\end{lemma}

\begin{proof}
Immediate from Lemma \ref{no_negative_quotient_on_curve} and the definition of semipositivity.
\end{proof}

	
%\begin{lemma}\label{semipos_of_symd_curve_charp}
%Let $C$ be a smooth curve over a field of characteristic $p>0$. Let $V$ be a semipositive vector bundle on $C$. Then, for any positive integer $d$, $Sym^dV$ is a semipositive vector bundle.
%\end{lemma}
%\begin{proof}
%Let $g$ be the genus of $C$, and let $\mathcal{L}$ be a line bundle on $C$ of degree at least $2g$. 
%By Lemma \ref{global_generation_of_twist_on_curve}, $V\otimes\mathcal{L}^{-1}$ is globally generated, so there exists a generically surjective map
%\begin{equation}
%f:(\mathcal{L}^{-1})^{\oplus r_i}\to V
%\end{equation}
%coming from a collection of $r$ global sections generating the fiber at some closed point of $C$.
%Applying $Sym^d$, we thus get a generically surjective map
%\begin{equation}
%Sym^df:(\mathcal{L}^{-\otimes d})^{\binom{d+r}{r}}\to Sym^{d}V
%\end{equation}
%
%Suppose $\mathcal{M}$ is a line bundle quotient of $Sym^dV$. By pre-composing with $\rho(f_1,\ldots,f_n)$, we obtain a nonzero map
%\begin{equation}
%\bigoplus_{i}\mathcal{L}^{-\otimes d}\to\mathcal{M}.
%\end{equation}
%Thus, we conclude that
%\begin{equation}\label{degree_of_quotient_not_too_small}
%\deg\mathcal{M}\ge d\cdot\deg(\mathcal{L}).
%\end{equation}
%
%Finally, suppose that $\rho(V_1,\ldots,V_n)$ has a line bundle quotient $\mathcal{N}$ of degree $e<0$. Let $F:C\to C$ be the absolute Frobenius map on $C$. Then, by pulling back the surjection $Sym^dV\to\mathcal{N}$ by $F^t$, where $t$ is a positive integer, we obtain a quotient line bundle of $(F^{t})^*Sym^dV\cong Sym^d(F^{*}V)$ of degree $dp^{t}<-N$ for sufficiently large $t$, which contradicts (\ref{degree_of_quotient_not_too_small}) due to Lemma \todo{pullback is still positive}.
%
%\end{proof}
%
%
%\begin{lemma}\label{semipos_of_symd_curve_all_char}
%Let $C$ be a smooth curve over a field of arbitrary characteristic. Let $V$ be a semipositive vector bundle on $C$. Then, for any positive integer $d$, $Sym^dV$ is a semipositive vector bundle.
%\end{lemma}
%\begin{proof}
%If the characteristic of the ground field is positive, this is Lemma \ref{semipos_of_symd_curve_charp}. Suppose that the characteristic of the ground field is zero.
%\end{proof}

\begin{lemma}\label{symd_semipos}
Let $k$ be a field.
Let $X$ be a scheme and $\mathcal{V}$ a semipositive vector bundle on $X$. 
Then, $\mathrm{Sym}^d\mathcal{V}$ is a semipositive vector bundle.
\end{lemma}
\begin{proof}
This is Lemma \ref{apply_rho_still_semipos} applied to the semipositive representation $\mathrm{Sym}^d$, see Lemma \ref{semipos_rep_examples}(\ref{sym_and_wedge_semipos}).
\end{proof}


\begin{lemma}
Let $k$ be a field.
Let $X$ be a $k$-scheme.
Let $\mathcal{V}_1,\ldots,\mathcal{V}_n$ be semipositive vector bundles on $X$ of ranks $r_1,\ldots,r_n$, respectively. 
Let $G=\prod_{i=1}^{n}\mathrm{GL}(r_i)$.
Let $\rho:G\to \mathrm{GL}(m)$ be a semipositive representation.
Let $\mathcal{W}=\rho(\mathcal{V}_1,\ldots,\mathcal{V}_n)$.
Let $\mathcal{V}\subset\mathcal{W}$ be a $G$-invariant subsheaf of $\mathcal{W}$.
Then, $\mathcal{V}$ is semipositive.
\end{lemma}

\begin{proof}
Let $x\in X$ be any point.
The $G$-invariant subspace $\mathcal{V}_x\subset\mathcal{W}_x$ defines a subrepresentation $\sigma$ of $\rho$, and we see that $\mathcal{V}=\sigma(\mathcal{V}_1,\ldots,\mathcal{V}_n)$.
By Lemma \ref{semipos_rep_examples}(\ref{subreps_and_quotients_semipos}), $\sigma$ is a semipositive representation, so by Lemma \ref{apply_rho_still_semipos}, $\mathcal{V}$ is a semipositive locally free sheaf.
\end{proof}



\bibliographystyle{unsrt}
\bibliography{references}

\end{document}
